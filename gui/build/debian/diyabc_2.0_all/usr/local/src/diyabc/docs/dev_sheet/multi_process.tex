\documentclass[12pt,a4paper]{article}
\usepackage[utf8]{inputenc}
\usepackage[francais]{babel}
\usepackage[T1]{fontenc}
\usepackage{graphicx}
\usepackage{fancyhdr}
\usepackage{geometry}
\usepackage{hyperref}
\author{Veyssier Julien}
%\institute{Université de Montpellier 2}

\title{Multi processing managers}
\date\today
%\pagestyle{headings}
\pagestyle{fancy}
\fancyhf{}
\fancyhead[L]{\thepage}
\fancyhead[R]{Biblio multi processing managers | Veyssier}

\geometry{hmargin=2.5cm,vmargin=2cm}

\begin{document}

\maketitle
\newpage

\tableofcontents

\newpage
 

\section{Introduction}
Ce document est la synthèse de la recherche bibliographique effectu\'ee sur les gestionnaires d'ex\'ecution
de processus parallèles.

\section{DIANE}
Comporte un job scheduler et un worker. 

\section{Ganga}
C'est la partie de DIANE qui gère la sp\'ecification des jobs : ex\'ecutable, destination, i/o, splitter/merger.

\section{ProActive}
Solution tout en un pour le lancement d'ex\'ecutions parallèles. Comporte une API pour cr\'eer des ``runs'', des job schedulers,
des ressource managers, des clients multiplateforme de visualisation de l'activit\'e des ressources et des agents capables de recevoir des jobs sur
des machines personnelles.
ProActive propose une API en java pour \ldots

\section{Wisdom}
Centr\'e sur la notion de grille et sur la formation d'un système pour produire collaborativement des donn\'ees utiles
au combat contre la malaria.

\section{OurGrid}
http://en.wikipedia.org/wiki/OurGrid

OurGrid est un projet br\'esilien d\'ejà mis en place dans certaines universit\'es ou laboratoires. C'est un système pour
cr\'eer des grilles de calcul pair à pair avec les ressources offertes par tout type de machine se d\'eclarant comme disponible.
On distingue trois cat\'gories de machines dans OurGrid :
\begin{itemize}
    \item Les brokers qui font des requêtes de calcul à la grille. Ils s'occupent de l'ordonnancement eux même.
    \item Les peers (composants reli\'es en P2P) qui sont la face visible du potentiel de calcul. Ils organisent dynamiquement
        la liste de ressources qu'ils tiennent à disposition.
    \item Les Workers qui attendent les parties ind\'ependantes de calcul.
        
\end{itemize}
\end{document}
