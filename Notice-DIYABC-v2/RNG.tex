By nature, a random number generator (RNG) is a sequential algorithm,
as described in Figure~\ref{fig:rng1} below. Indeed, we shall describe
a RNG by its updating function $f$ changing deterministically the
internal state. Each time the user requires a new realization of the
uniform distribution over $[0;1)$, the algorithm derives a value
$u_{k}$ from the current internal state $i_{k}$ and then updates
this state with $f$. Hence a first and important issue for parallel
Monte Carlo computations is to design independent RNGs that might
run in parallel while minimizing the communications between processors.
It is quite standard to use as many RNGs as computing cores in the
computer or in the cluster of computers.

\begin{figure}[htb]
\centering \textit{}%
\begin{minipage}[c]{0.6\linewidth}%
 \includegraphics[width=1\textwidth]{DCMT_rng} \caption[width=.6\textwidth]{\label{fig:rng1} \textbf{\textit{\footnotesize Random Number Generator.}}\textit{\footnotesize{}
A RNG is an algorithm that produces a sequence of floating numbers,
says $u_{0},u_{1},\ldots$, that resembles a sequence of independant
random numbers, uniformly distributed over $[0;1)$. It uses a sequence
of internal states, say $i_{0},i_{1},\ldots$, which are computed
by reccurence, namely, $i_{k+1}=f(i_{k})$. The first internal state
$i_{0}$ is often named the seed.}}
%
\end{minipage}
\end{figure}


The second version of DIYABC uses the Dynamic Creator (DCMT) of \citet{DCMT}
to look for a set of independent Mersenne-Twister generators. Actually,
the updating function $f$ of a Mersenne-Twister generator is parametrized
by a few integer numbers. The output of the DCMT is a set of $N$
updating functions, say $\{f^{(1)},\ldots,f^{(N)}\}$, producing independent
streams. That is, the $n$-th RNG is a sequence of iternal states
$i_{0}^{(n)},i_{1}^{(n)},i_{2}^{(n)},\ldots$ satisfying $i_{k+1}^{(n)}=f^{(n)}(i_{k}^{(n)})$
that gives rise to a sequence of independent, uniformly distributed
numbers $u_{0}^{(n)},u_{1}^{(n)},u_{2}^{(n)},\ldots$. We found that
the DCMT was simple to use and gave good results. There is no limitation
on the number $N$ of RNGs it produces. Once initialized, the different
RNGs do not require any communication between them and each of them
runs as quickly as a single Mersenne-Twister generator. But an important
limitation is that it is impossible to add a new RNG to the set produced
by the DCMT. Practically, this means that we have to know \textit{a
priori} a bound on the number of jobs working together in parallel.
See section \ref{cluster}

%%% Local Variables: 
%%% mode: latex
%%% TeX-master: "Notice_DIYABC_principal"
%%% End: 

