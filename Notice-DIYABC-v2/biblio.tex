\begin{thebibliography}{a}
\bibitem[Beaumont \emph{et al.}, 2002]{B2002} Beaumont, M. A., W. Zhang and D. J. Balding, 2002. Approximate Bayesian Computation in Population Genetics. \emph{Genetics} \textbf{162}, 2025-2035.
\bibitem[Beaumont, 2008]{B2008}Beaumont, M.A., 2008. Joint determination of topology, divergence time, and immigration in population trees. In Simulation, Genetics, and Human Prehistory, eds. S. Matsumura, P. Forster,  C. Renfrew. McDonald Institute Press, University of Cambridge (\emph{in press}).
\bibitem[Begg and Gray, 1984]{BG1984} Begg, C.B. and R. Gray, 1984. Calculation of polychotomous logistic regression parameters using individualized regressions. \emph{Biometrika},
\textbf{71}, 11-18.
\bibitem[Belkhir \emph{et al.}, 1996-2004]{BB1996} Belkhir K., Borsa P., Chikhi L., Raufaste N. and F. Bonhomme, 1996-2004 GENETIX 4.05, logiciel sous Windows TM pour la g�n�tique des populations. Laboratoire G�nome, Populations, Interactions, CNRS UMR 5171, Universit� de Montpellier II, Montpellier (France).
\bibitem[Bertorelle and Excoffier, 1998]{BE1998} Bertorelle, G. and L. Excoffier, 1998. Inferring admixture proportion from molecular data. \emph{Mol. Biol. Evol.} \textbf{15}, 1298-1311.
\bibitem[Choisy  \emph{et al.}, 2004]{CF2004} Choisy, M., P. Franck and J.M. Cornuet, 2004. Estimating admixture proportions with microsatellites : comparison of methods based on simulated data. \emph{Mol. Ecol.} \textbf{13}, 955-968.
\bibitem[Chakraborty and Jin, 1993]{CJ1993}Chakraborty R and L Jin, 1993. A unified approach to study hypervariable polymorphisms: statistical considerations of determining relatedness and population distances. EXS. \emph{67}, 153�175.
\bibitem[Cornuet \emph{et al.}, 2006]{C2006} Cornuet, J. M.,
M. A. Beaumont, A. Estoup and M. Solignac, 2006. Inference on
microsatellite mutation processes in the invasive mite, \emph{Varroa
destructor}, using reversible jump Markov chain Monte Carlo.
\emph{Theoret. Pop. Biol.} \textbf{69}, 129-144.
\bibitem[Cornuet \emph{et al.}, 2010]{C2010}Cornuet J.M., V. Ravign\'e and A. Estoup, 2010. Inference on population history and model checking using DNA sequence and microsatellite data with the sofware DIYABC (v1.0). \emph{submitted}. 
\bibitem[Cornuet \emph{et al.}, 2008]{C2008}Cornuet J.M., F. Santos, M.A. Beaumont, C.P. Robert, J.M. Marin, D.J. Balding, T. Guillemaud and A. Estoup, 2008. Infering population history with DIYABC: a user-friendly approach to Approximate Bayesian Computations. \emph{Bioinformatics}, \textbf{24} (23), 2713-2719.

\bibitem[Estoup \emph{et al.}, 1993]{E1993} Estoup, A., M. Solignac, M. Harry and J.M. Cornuet, 1993. Characterization of $(GT)_n$ and $(CT)_n$ microsatellites in two insect species: \emph{Apis mellifera} and \emph{Bombus terrestris}. \emph{Nucl. Ac. Res.}, \textbf{21}, 1427-1431.
\bibitem[Estoup \emph{et al.}, 2001]{E2001} Estoup, A., I. J. Wilson, C. Sullivan, J. M. Cornuet and C. Moritz, 2001 Inferring population history from microsatellite et enzyme data in serially introduced cane toads, \emph{Bufo marinus}. \emph{Genetics}, \textbf{159}, 1671-1687.
\bibitem[Estoup \emph{et al.}, 2002]{E2002}Estoup, A., P. Jarne and J.M. Cornuet, 2002.
 Homoplasy and mutation model at microsatellite loci and their consequences for population
 genetics analysis. \emph{Mol. Ecol.}, \textbf{11}, 1591-1604.
\bibitem[Estoup and Clegg, 2003]{EC2003} Estoup, A. and S. M. Clegg, 2003. Bayesian inferences on the recent islet colonization history by the bird Zosterops lateralis lateralis. \emph{Mol. Ecol.} \textbf{12}: 657-674.
\bibitem[Estoup \emph{et al.}, 2004]{EB2004}Estoup, A., M.A. Beaumont, F. Sennedot, C. Moritz and J.M. Cornuet, 2004. Genetic analysis of complex demographic scenarios : spatially expanding populations of the cane toad, \emph{Bufo marinus}. \emph{Evolution},\textbf{58}, 2021-2036.
\bibitem[Estoup \emph{et al.}, 2012]{EL2012}Estoup, A., E. Lombaert, J.M. Marin, T. Guillemaud, P. Pudlo, C.P. Robert and J.M. Cornuet, 2012. Estimation of demo-genetic model probabilities with Approximate Bayesian Computation using linear discriminant analysis on summary statistics. \emph{Molecular Ecology Resources}, \textbf{12}, 846-855.

\bibitem[Excoffier \emph{et al.}, 2005]{Ex2005} Excoffier, L., A. Estoup and J.M. Cornuet, 2005.
 Bayesian analysis of an admixture model with mutations and arbitrarily linked markers. \emph{Genetics} \textbf{169}, 1727-1738.
\bibitem[Fagundes \emph{et al.}, 2007]{FR2007} \textsc{Fagundes, N.J.R., N. Ray, M.A. Beaumont, S. Neuenschwander, F. Salzano, S.L. Bonatto and L. Excoffier}, 2007. Statistical evaluation of alternative models of human evolution. \emph{Proc. Natl. Acad. Sc.}, \textbf{104} : 17614-17619.
\bibitem[Fu and Chakraborty, 1998]{F1998} Fu, Y.X. and
Chakraborty, R., 1998. Simultaneous estimation of all the parameters
of a stepwise mutation model. \emph{Genetics}, \textbf{150},
487-497.
\bibitem[Garza and Williamson, 2001]{GW2001} Garza JC and E Williamson, 2001. Detection of reduction in population size using data from microsatellite DNA. \emph{Mol. Ecol.}   \textbf{10},305-318.
\bibitem[Gelman \emph{et al.}, 1995]{GCSR1995} Gelman, A., J.B. Carlin, H.S. Stern and D.B. Rubin, 1995. \emph{Bayesian Data Analysis}. Chapman et Hall, London, 526p. 
\bibitem[Golstein \emph{et al.}, 1995]{GL1995} Goldstein DB, Linares AR, Cavalli-Sforza LL, and Feldman MW, 1995. An evaluation of genetic distances for use with microsatellite loci. \emph{Genetics} \textbf{139}, 463-471.
 \bibitem[Goudet, 1995]{G1995} Goudet, J. ,1995. FSTAT (Version 1.2): A computer program to calculate F- statistics. \emph{J. Hered.} \textbf{86},  485-486.
 \bibitem[Griffiths and Tavar\'e, 1994]{GT1994} Griffiths, R.C. and S. Tavar\'e, 1994. Simulating probability distributions in the coalescent. \emph{Theor. Pop. Biol.} \textbf{46}, 131-159.
\bibitem[Guillemaud  \emph{et al.}, 2010]{G2010}Guillemaud T., M.A. Beaumont, M. Ciosi, J.M. Cornuet and A. Estoup, 2010. Inferring introduction routes of invasive species using approximate Bayesian computation on microsatellite data. \emph{Heredity}, \textbf{104}, 88-99.
\bibitem[Haag-Liautard \emph{et al.}, 2008]{HL2008} Haag-Liautard C., N. Coffey, D. Houle, M. Lynch, B. Charlesworth and P.D. Keightley, 2008. Direct estimation of the mitochondrial DNA mutation rate in Drosophila melanogaster. \emph{Plos Biol}, \textbf{6}, e204.
 \bibitem[Hamilton \emph{et al.}, 2005]{HS2005} Hamilton, G., M. Stoneking and L. Excoffier, 2005. Molecular analysis reveals tighter social regulation of immigration in patrilocal populations than in matrilocal populations. \emph{Proc. Natl. Acad. Sci. USA}, \textbf{102}, 7476-7480.
\bibitem[Hasegawa  \emph{et al.}, 1985]{HKY1985} Hasegawa, M., Kishino, H and Yano, T., 1985. Dating the human-ape splitting by a molecular clock of mitochondrial DNA. \emph{Journal of Molecular Evolution} 22:160-174.
\bibitem[Hudson \emph{et al.}, 1992]{H1992} Hudson,R. R., M. Slatkin and W.P. Maddison, 1992. Estimation of levels of gene flow fom DNA sequence data. \emph{Genetics}, 132, 583-589.
\bibitem[Ihaka and Gentleman, 1996]{IG1996}Ihaka R. and R. Gentleman, 1996. $R$: a language for data analysis and graphics. \emph{J.  Comput. Graph. Stat.}, \textbf{5}, 299-314
\bibitem[Ingvarsson, 2008]{I2008} Ingvarsson P.K., 2008. Multilocus patterns of nucleotide polymorphism and the demographic history of \emph{Populus tremula. Genetics}, 180: 329-340.
\bibitem[Jukes and Cantor, 1969]{JK1969}Jukes, TH and Cantor, CR., 1969. Evolution of protein molecules. Pp. 21-123 in H. N. Munro, ed. \emph{Mammalian protein metabolism}. Academic Press, New York.
\bibitem[Kimura, 1980]{K1980}Kimura, M., 1980. A simple method for estimating evolutionary rate of base substitution through comparative studies of nucleotide sequences. \emph{Journal of Molecular Evolution} 16:111-120.
\bibitem[Lombaert \emph{et al.}, 2010]{L2010}Lombaert E., T. Guillemaud, J.M. Cornuet, T. Malausa, B. Facon and A. Estoup, 2010.  Bridgehead effect in the worldwide invasion of the biocontrol harlequin ladybird. \emph{PLoS ONE}, http://dx.plos.org/10.1371/journal.pone.0009743.
\bibitem[Matsumoto and Nishimura, 2000]{DCMT} Matsumoto M and T Nishimura, 2000. Dynamic Creation of Pseudorandom Number Generators. \emph{Monte Carlo and Quasi-Monte Carlo Methods 1998}, Springer, pp 56--69.
\bibitem[Miller \emph{et al.}, 2005]{ME2005} Miller N, A. Estoup, S. Toepfer, D Bourguet, L. Lapchin, S. Derridj, K.S. Kim, P Reynaud, F. Furlan and T. Guillemaud,  2005. Multiple Transatlantic Introductions of the Western Corn Rootworm. \emph{Science}, \textbf{310}, p. 992
\bibitem[Nei, 1972]{N1972} Nei M., 1972. Genetic distance between populations. \emph{Am. Nat.} 106:283-292
\bibitem[Nei, 1987]{N1987} Nei M., 1987. \emph{Molecular Evolutionary Genetics}. Columbia University Press, New York, 512 pp.
\bibitem[Ohta and Kimura, 1973]{O1973} Ohta, T. and Kimura, M.,
1973. A model of mutation appropriate to estimate the number of
electrophoretically detectable alleles in a finite population.
\bibitem[Pascual \emph{et al.}, 2007]{PC2007}Pascual, M., M.P. Chapuis, F. Mestres, J. Balany\'a, R.B. Huey, G.W. Gilchrist, L. Serra and A. Estoup, 2007. Introduction history of \emph{Drosophila subobscura} in the New World : a microsatellite based survey using ABC methods. \emph{Mol. Ecol.}, \textbf{16}, 3069-3083.
\bibitem[Pollock DD \emph{et al.}, 1998]{PDD1998}Pollock DD, Bergman A, Feldman MW, Goldstein DB, 1998. Microsatellite behavior with range constraints: parameter
estimation and improved distances for use in phylogenetic reconstruction. \emph{Theoretical Population Biology}, \textbf{53}, 256–271.
\bibitem[Pritchard \emph{et al.}, 1999]{P1999} Pritchard, J., M. Seielstad,
A. Perez-Lezaun and M. Feldman, 1999. Population growth of human Y
chromosomes: a study of Y chromosome microsatellites. \emph{Mol.
Biol. Evol.} \textbf{16}, 1791-1798.
\bibitem[Rannala and Moutain, 1997]{RM1997}Rannala, B., and J. L. Mountain, 1997. Detecting immigration by using multilocus genotypes. \emph{Pro. Nat. Acad. Sci. USA} \textbf{94}, 9197-9201.
\bibitem[Raymond and Rousset, 1995]{RR1995} Raymond M., and F. Rousset, 1995. Genepop (version 1.2), population genetics software for exact tests and ecumenicism. \emph{J. Hered.}, \textbf{86}, 248-249
\bibitem[Schug  \emph{et al.}, 1997]{SMA1997} Schug M.D., T.F. Mackay and C.F. Aquadro, 1997. Low mutation rates of microsatellite loci in \emph{Drosophila melanogaster}. \emph{Nat Genet.} \textbf{15}, 99-102.
\bibitem[Stephens and Donnelly, 2000]{SD2000} Stephens, M. and P. Donnelly,
 2000, Inference in molecular population genetics (with discussion).
 J. R. Stat. Soc. B \textbf{62}, 605-655.
 \bibitem[Tajima, 1989]{TA1989}Tajima, F., 1989. Statistical method for testing the neutral mutationhypothesis by DNA polymorphism. \emph{Genetics} 
123: 585-595
 \bibitem[Tamura and Nei, 1993]{TN1993} Tamura, K., and M. Nei., 1993. Estimation of the number of nucleotide substitutions in the control region of mitochondrial DNA in humans and chimpanzees. \emph{Molecular Biology and Evolution} 10:512-526.
\bibitem[V\'azquez  \emph{et al.}, 2000]{VPAD2000} V\'azquez F.J., T. P�rez, J. Albornoz and A. Dom\'inguez, 2000.Estimation of microsatellite mutation rates in \emph{Drosophila melanogaster}. \emph{Genet Res.}, \textbf{76}, 323-6.
\bibitem[Weir and Cockerham, 1984]{WC1984}Weir BS and CC Cockerham , 1984. Estimating F-statistics for the analysis of population structure. \emph{Evolution} \textbf{38}: 1358-1370.
\bibitem[Whittaker \emph{et al.},
2003]{W2003} Whittaker, J.C., Harbord, R.M., Boxall, N., Mackay, I.,
Dawson, G. and Sibly, R.M. 2003. Likelihood-based estimation of
microsatellite mutation rates, \emph{Genetics}, \textbf{164},
781-787.
\end{thebibliography}
