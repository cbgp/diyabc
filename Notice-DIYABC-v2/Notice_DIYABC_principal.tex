\documentclass [a4paper]{report}
\usepackage{graphicx,array,natbib,amsmath,bm}
\usepackage{geometry}
\usepackage{soul,color}
\usepackage[pagewise]{lineno}
\usepackage{setspace}
\usepackage{listings}
\usepackage{multirow}

\geometry{ hmargin=2.5cm, vmargin=2.5cm }

% \usepackage[latin1]{inputenc}
 %\usepackage[T1]{fontenc}
 %\usepackage[normalem]{ulem}
% \usepackage[french]{babel}
% \usepackage{verbatim}
% \usepackage{graphicx}

\title{\Huge \emph{\textbf{DIYABC}} \\ version 2.0\\------\\ A user-friendly software \\  for inferring population history  through \\ Approximate Bayesian Computations}

\author{\vspace{1.0 cm}\Large J.M. Cornuet, P. Pudlo, J. Veyssier,\\ A. Dehne-Garcia, A. Estoup and V. Ravign\'e\\
\vspace{6.0cm}
Centre de Biologie et de Gestion des Populations\\
Institut National de la Recherche Agronomique\\
Campus International de Baillarguet, CS 30016 Montferrier-sur-Lez\\
34988 Saint-G\'ely-du-Fesc Cedex, France\\
\texttt{(diyabc@cbgp.supagro.fr)}}


\date{\today}
\normalsize

% Ce qui suit sert à numéroter les section, sous-sections, ... � partir de 1 au lieu de 0
\setcounter{secnumdepth}{4} 

\renewcommand{\thesection}{\arabic{section}.} 
\renewcommand{\thesubsection}{\thesection \arabic{subsection}} 
\renewcommand{\theparagraph}{\thesubsection.\arabic{paragraph}} 

\let\sectionv\section 
\renewcommand{\section}[1]{\sectionv{#1} \setcounter{paragraph}{0}} 

\let\subsectionv\subsection 
\renewcommand{\subsection}[1]{\subsectionv{#1} \setcounter{paragraph}{0}} 
% Fin de la modif de renum�rotation

\linenumbers
\begin{document}
\maketitle %\clearpage
\pagestyle{myheadings}
\markboth{DIYABC v2.0}{DIYABC v2.0}
\begin{doublespacing}
\tableofcontents
\end{doublespacing}
\newpage
\section{Preface}
In less than 10 years, Approximate Bayesian Computations (ABC) have developed in the Population Genetics community as a new tool for inference on the past history of populations and species. Compared to other approaches based on the computation of the likelihood which are still restrained to a very narrow range of evolutionary scenarios and mutation models, the ABC approach has demonstrated its ability to stick to biological situations that are much more complex and hence realistic. However, this approach still requires numerous computations to be performed so that it has been used mostly by specialists (i.e. statisticians and programmers). This has almost certainly restrained the possible impact of ABC in population genetic studies. We believe that this situation must be improved and therefore we have developed a computer program for the large community of experimental biologists. We therefore designed DIYABC as a user-friendly program allowing non specialist biologists to achieve their own analysis.\

The first version ($DIYABC v0.x$) had been written especially for microsatellite data. There were at least two reasons for that. The first one is that we have been among the first to develop and use this class of markers in population genetic studies \citep[e.g.][]{E1993}. Since then, we have developed microsatellites in numerous species as well as we have published theoretical studies and reviews on these markers \citep[e.g.][]{E2002}. The second reason is that microsatellites have been and still are very popular markers in the population geneticist community and there is now a large quantity of data that might benefit of an ABC approach.\

The second version of our software ($DIYABC v1.x$) has been designed to make use of DNA sequence data.This has several immediate consequences. For instance, the standard Genepop data file format has been extended to incorporate sequence data. This has been done in collaboration with the authors of $Genepop$ and explained in  subsection 4.1.1. In this version, sequence loci are considered in the same way as microsatellite loci, i.e. they are considered as genetically independent and intra-locus recombination is not (yet) available. Concerning mutation models for DNA sequences, we used the same philosophy as for microsatellites, i.e. the program considers only simple and widely used models, keeping in mind that a higher-dimensional parameter space will be less well explored than a lower-dimensional space. Note that none of these mutation models includes insertion-deletions. 
Also five categories of loci (either microsatellites or DNA sequences) were considered in this second version : autosomal diploid, autosomal haploid, X-linked, Y-linked and mitochondrial. Note that X-linked loci can be used for an haplo-diploid species in which both sexes have been sampled. If non-autosomal loci have been typed in population samples, the sex-ratio of the species will have to be provided (see subsection 4.1.1).\\ 

%Also for DNA sequence loci, one can choose among two possibilities : either each locus has its own mutation model or they all share the same mutation model (but not the same mutation parameters). Going back to microsatellites, there is an additional mutation parameter added to this version : the mean rate of single nucleotide insertion-deletion mutations.\\

%Given that mitochondrial DNA sequences are commonly used by population geneticists and that different effective population sizes correspond to such monoparental loci, a specific code has also been added to the genepop format to comply with this new requirement (see 4.1.1 for details). 

Other improvements over version 0 included :
\begin{enumerate}
\item the use of multithread technology in order to exploit multicore/multiprocessor computers. This is especially useful when building the reference table and for several other intensive computation steps, such as the multinomial logistic regression,
\item a new option which helps the detection of "bad" prior modelisation of the data, 
\item another new option which helps evaluate the goodness of fit of a given model-parameter posterior combination (i.e. Model checking),   
\item many new screens implemented not only to treat sequence data, but also to cope with the new options described above, as well as to offer useful complementary information on the current run. 

\end{enumerate}

 The third version of $DIYABC$ ($DIYABCv2.x$)  has been entirely recoded in order to be used under the usual three OS (Windows, Mac and Linux). Also the code for computations has been separated from that of the graphic user interface (GUI). The former has been rewritten in C++ and the latter is a mixture of Python and Qt (PyQt). The user can then launch computations with or without using the GUI. The GUI 's uses are : 

\begin{enumerate}
\item the management of projects
\item the input of the historical and genetical models
\item the parameterization of analyses
\item the launch of computations  of the reference table and of the various required analyses
\item the visualization of results
\end{enumerate}

Also, as DNA sequences have been added in the second version, a new category of markers has been added to the third version : Single Nucleotide Polymorphisms (SNPs). Instead of extending once more the Genepop format, a new data simple format has been designed for these markers.  

This version includes all improvements of version 1.x and a few new improvements such as :
\begin{itemize}
\item loci of the same type (i.e. microsatellites on one hand or DNA sequences on the other hand) can be associated in one or more groups. This allows for instance to define different mutation models for microsatellites with motifs of different lengths.
\item  the model checking option is now presented as a direct option (not a suboption of the ABC estimation of parameters) which largely simplifies its use.
\item the logistic regression can be performed on factorial discriminant analysis components instead of all summary statistics. This reduces the number of dependent variables, thus allowing to run large "confidence in scenario choice" analyses including many summary statistics and scenarios.
\item ascertainment bias in the design of SNPs can be tentatively corrected by considering ``reference'' samples in which the loci need to be polymorphic in order to belong to the SNP set. These reference samples are not necessarily included of the actual samples.
\end{itemize}

\subsection{Acknowledgements}
We thank Mark Beaumont who has been at the origin of our interest for ABC. He offered us constant help and inspiration since the beginning. We also thank David Balding who welcomed one of us (JMC) in his team during the whole writing of the program and who organized several workshops on ABC during the same period. We are indebted to Christian Robert, Jean-Michel Marin, Stuart Baird, Thomas Guillemaud, Renaud Vitalis, Gael Kergoat, Gilles Guillot and David Welsh with whom we discussed many theoretical and practical aspect of DIYABC in the numerous meetings financed by a grant from the French Research National Agency (project $MISGEPOP$ ANR-05-BLAN-196). The same grant is also aknowledged for having paid for the 2-year salary of FS. This research was also supported by an EU grant awarded to JMC as an EIF Marie-Curie fellowship (project $StatInfPopGen$) and which allowed him to come to David Balding's place at Imperial College (London, UK). Current and future developments of DIYABC are financed by a new grant from the French Research National Agency (project $EMILE$ ANR-09-BLAN-0145) awarded in september 2009.
\subsection{References to cite}
\begin{itemize}
\item \textbf{version 0 :} Cornuet J.M., F. Santos, M.A. Beaumont, C.P. Robert, J.M. Marin, D.J. Balding, T. Guillemaud and A. Estoup. Inferring population history with DIYABC: a user-friendly approach to Approximate Bayesian Computations (2008) \emph{Bioinformatics}, \textbf{24} (23), 2713-2719.
\item  \textbf{version 1 :}  Cornuet J.M., V. Ravign\'e and A. Estoup, 2010. Inference on population history and model checking using DNA sequence and microsatellite data with the sofware DIYABC (v1.0) (2010) \emph{BMC Bioinformatics}, \textbf{11}, 401.
\item  \textbf{version 2 :}  Cornuet J.M., Pudlo P., Veyssier J., Dehne-Garcia A., V. Ravign\'e and A. Estoup. Improved inference on population history using SNP's and DIYABC (v2). \emph{submitted}
\end{itemize}
\subsection{Web site}
http://www.montpellier.inra.fr/CBGP/diyabc\\



\newpage
\section{Methodology}
\subsection{Basic notions on ABC} Approximate Bayesian Computation or
ABC is a bayesian approach in which the posterior distributions of
the model parameters are determined by replacing the computation of
the likelihood (probability of observed data given the values of the
model parameters) by a measure of similarity between observed and
simulated data. The posterior distributions are estimated from
parameter values providing simulated data that are the most similar
to observed data. Historically, different ways of estimating this
similarity have been proposed, but all have been based on statistics
summarizing information conveyed by the data set. In population
genetics, data most often relate to individuals that have been
genotyped at a given set of loci, these individuals being representative of the
studied populations. The summary statistics are for instance the mean
number of alleles per population or genetic distances between pairs
of populations. It is much easier to measure the
similarity between small sets of summary statistics than between
large sets of multilocus genotype data. When the number of summary
statistics is low, it is possible to select simulated data for which
\emph{all} the summary statistics are close to those of the observed
data \citep{P1999,E2001,EC2003}. However, for more complex scenarios
necessitating a larger number of summary statistics, it becomes
almost impossible to find such simulated data sets. \citet{B2002}
have hence proposed to measure similarity through the Euclidian
distance between observed and simulated summary statistics, after normalization by standard
deviations of simulated statistics. In addition, these authors introduced a
step of weighted local linear regression aimed at favoring simulated data
sets that are closer to the observed one.\\
 In practice, the ABC approach can be summarized in three successive
 steps \citep{Ex2005} : i) generating simulated data sets, ii)
 selecting simulated data sets closest to observed data set and iii)
 estimating posterior distributions of parameters through a local
 linear regression procedure.\\ 
 In addition, this approach provides a way of comparing different
 models (hereafter named scenarios) that can explain observed data. Two measures of posterior probabilities of scenarios are proposed. The first measure is simply the relative proportion of each scenario in the simulated data sets closest to observed data sets \citep{ME2005,PC2007}. The second measure  is obtained by a logistic regression of each scenario probability on the deviations between simulated and observed summary statistics \citep{FR2007,B2008}. \\
 
 
 In order to simulate data, one has first to define one (or possibly
several) scenario(s). Each scenario includes a historical model describing
how the sampled populations  are connected to their common ancestor
and a mutational model describing how allelic states of the studied
genes are changing along their genealogical trees.\\

\subsection{Historical model parameterization}
The evolutionary scenario, which is characterized by the historical
model, can be described as a succession in time of "events" and
"inter event periods". The events considered in the program are a
restricted set of possible events affecting populations evolution.
In the current version of the program, we consider only 4 categories
of events : population divergence, 
discrete change of effective population size, admixture and sampling (the last one has been added
to allow considering samples taken at different times). Between two successive
events affecting a population, we assume that populations evolve
independently (e.g. without migration) and with a fixed effective
size. The usual parameters of the historical model are the times of
occurrence of the various events (counted in generations), the
effective sizes of populations and the admixture rates. When writing
the scenario, events have to be coded  sequentially backward in time 
(see section \emph{2.5  Prior Distribution} when time priors are overlapping).
Although this choice may not be natural at first sight, it is coherent with
coalescence theory on which are based all data simulations in the
program. For that reason, the keywords for a divergence or an
admixture event are \texttt{merge} and \texttt{split}, respectively.
Two other keywords, \texttt{varNe} and \texttt{sample},
correspond to a discrete change in effective population size and a gene
sampling, respectively. 
%Eventually, only for SNPs, we have added the keyword \texttt{refsample} to control the ascertainment bias (see below).
\\
A scenario takes the form of a succession of lines (one line per
event), each line starting with the time of the event, then the
nature of the event, and ending with several other data depending on
the nature of the event. Following is the syntax used for each
category of event :
\begin{description}
\item[population sample] : $\langle time \rangle$ \texttt{sample} $\langle pop \rangle$  %$[nmales$   $nfemales]$  \\ 
$\langle time \rangle$ is the time (always counted in number of generations) at which the sample was taken and\\
 $\langle pop \rangle$ is the population number from which is taken the sample. It is worth stressing here that \textbf{samples are considered
in the same order as they appear in the data file}.\\
%$[nmales$   $nfemales]$ is only used for SNP loci to indicate the number of males (respectively females) from the sample that have been used to detect SNPs. Thes males and females appear in the corresponding sample of the data file.
 
%\item[``reference'' sample] : $\langle time \rangle$ \texttt{refsample} $\langle pop \rangle$  $nmales$   $nfemales$  \textbf{ONLY FOR SNP loci}\\ 
%$\langle time \rangle$ is the time (always counted in number of generations) at which the ``reference'' sample was taken and\\
% $\langle pop \rangle$ is the population number from which is taken the sample.\\
%$nmales$   $nfemales$ indicate the number of males (respectively females) that have been used to detect SNPs in the reference sample. These males and females \textbf{do not appear} in the data file.

\item[population size variation] : $\langle time \rangle$ \texttt{varNe}
 $\langle pop \rangle$ $\langle Ne \rangle$\\
 From time $\langle time \rangle$, looking backward in time,
 population $\langle pop \rangle$ will have an effective size $\langle Ne
 \rangle$.
\item[population divergence] :
$\langle time \rangle$ \texttt{merge} $\langle pop0 \rangle$
$\langle pop1 \rangle$\\
At time $\langle time \rangle$, looking backward in time, population
$\langle pop1 \rangle$ "merges" with population $\langle pop0
\rangle$. Hereafter, only $\langle pop0 \rangle$ "survives".
\item[population admixture] :
$\langle time \rangle$ \texttt{split} $\langle pop0 \rangle$
$\langle pop1 \rangle$ $\langle pop2 \rangle$ $\langle rate \rangle$\\
At time $\langle time \rangle$, looking backward in time, population
$\langle pop0 \rangle$ "splits" between populations $\langle pop1
\rangle$ and $\langle pop2 \rangle$. A gene lineage from population
$\langle pop0 \rangle$ joins population $\langle pop1 \rangle$
(respectively $\langle pop2 \rangle$) with probability $\langle rate
\rangle$ (respectively 1-$\langle rate \rangle$).
\end{description}

A historical model is a succession of lines as described above. However, in
order to cope with special situations (see explanations in Note
 9 below), we added a first line giving the effective sizes of sampled
populations before the first event described, looking backward in time.
Expressions between arrows, other than population numbers, can be
either a numeric value (e.g. 25) or a character string (e.g.
\texttt{t0}). In the latter case, it is considered as a parameter of
the model. So the only possible parameters of the historical model
are times of events, effective population sizes and admixture
rates.\\
The program offers the possibility to add or remove scenarios, by just clicking on the corresponding buttons. The usual shortcuts (CTRL+C, CTRL+V and CTRL+X) can be used to edit the different scenarios. Some or all parameters can be in common among scenarios.\\

\textbf{Notes}
\begin{enumerate}
\item There are two ways of giving a fixed value to effective population sizes, times and admixture rates. Either the fixed value appears as a numeric value in the scenario windows or it  is given as a string value like any parameter. In the latter case, one gives this parameter a fixed value by choosing a Unifom distribution and setting the minimum and maximum to that value in the prior setting (see section 2.4).
\item All expressions must be separated by at least one space.
\item All expressions relative to parameters can include sums or
differences. For instance, it is possible to write :\\
\texttt{t0 merge 2 3}\\
\texttt{t0+t1 merge 1 2}\\
This means that \texttt{t1} is the time elapsed between the two
events. By imposing \texttt{t1>0} (as explained in section
\textbf{prior and posterior distributions}), this implies that the
divergence of populations 1 and 2 is always more ancient than the
divergence of populations 2 and 3. \
However, one cannot mix a parameter and a numeric value (e.g. \texttt{t1+150} will result in an error). This can be done by writing \texttt{t1+t2} and fixing \texttt{t2} by choosing a uniform distribution with lower and upper bounds both equal to 150. 
\item Time is always given in generations. Since we look backward,
time increases towards past.
\item Negative times are allowed (e.g. the example given in section 3), but not recommended.
\item Population numbers must be consecutive natural integers
starting at 1. The number of population can exceed the number of
samples and vice versa : in other words, unsampled populations can be considered in
the scenario on one hand, and the same population can be sampled
more than once on the other hand.
\item Multi-furcating population trees can be considered, by writing
several divergence events occurring at the same time. However, one
has to be careful to the order of the \texttt{merge} events. For
instance, the following piece of scenario will fail :\\
\texttt{100 merge 1 2} \\
\texttt{100 merge 2 3} \\
This is because, after the first line, population 2, which has
merged with population 1, does not "exist" anymore (the surviving population is population 1). So, it cannot
receive lineages of population 3 as it should as a result of the
second line. The correct ways are either to put line 2 before line
1, or to change line 2 to :\\ \texttt{100 merge 1 3}.
\item Since times of events can be parameters, the order of events
can change according to the values taken by the time parameters. In
any case, before simulating a data set, the program sorts out events
by increasing times \footnote{ Sorting events by increasing times can only be done when all time values are known, i.e. when simulating datasets. When checking scenarios, all time values are not yet defined, so that when visualizing a scenario, events are represented in the same order as they appear in the window used to define the scenario.}. If two or more events occur at the same time,
the order is that of the scenario as it is written by the the user.
\item Most scenarios begin with sampling events. We then need to
know the effective size of the populations to perform the simulation
of coalescences until the next event concerning each population. One
way would have been to provide the population size on the same line
of the scenario description. However, in some scenarios with varying
population sizes, it can not be determined what is the effective
size at the sampling time before the set of time parameter values is
generated. For that reason, we decided to provide the effective size
and the sampling description on two distinct lines.
\end{enumerate}

\textbf{Examples} Below are some usual scenarios with increasing complexity. Each scenario is coded on the left side and a graphic representation given by DIYABC is printed on the right side
\begin{enumerate}
\item One population from which several samples have been taken at
various generations : 0, 3 and 10. The only unknown parameter of the scenario\footnote{Of course, there are also one or more parameter(s) for the mutation model.} is the
 effective population size. \\
\begin{center} 
\begin{tabular}{cc}
\includegraphics[scale=0.5]{code_scenario_01.pdf} & \includegraphics[scale=0.35]{scenario_01.pdf} \\
\end{tabular}
\end{center}

\item Two populations of size \texttt{N1} and \texttt{N2} have diverged \texttt{t} generations
in the past from an ancestral population of size \texttt{N1+N2}.\\
\begin{center} 
\begin{tabular}{cc}
\includegraphics[scale=0.5]{code_scenario_02.pdf} & \includegraphics[scale=0.35]{scenario_02.pdf} \\
\end{tabular}
\end{center}


\item Two parental populations (1 and 2) with constant effective populations sizes
\texttt{N1} and \texttt{N2} have diverged at time \texttt{td} from
an ancestral population of size \texttt{NA}. At time \texttt{ta},
there has been an admixture event between the two populations giving
birth to an admixed population (3) with effective size \texttt{N3}
and with an admixture rate \texttt{ra} relative to population 1.\\
\begin{center} 
\begin{tabular}{cc}
\includegraphics[scale=0.5]{code_scenario_03.pdf} & \includegraphics[scale=0.35]{scenario_03.pdf} \\
\end{tabular}
\end{center}

\item The next scenario is slightly more complicated. It includes four population samples and two admixture events.
For simplicity sake, all populations are assumed to have identical effective sizes (\texttt{Ne}).\\
\begin{center} 
\begin{tabular}{cc}
\includegraphics[scale=0.5]{code_scenario_04.pdf} & \includegraphics[scale=0.35]{scenario_04.pdf} \\
\end{tabular}
\end{center}



Note that although there are only four samples, the scenario
includes a fifth unsampled population. This unsampled population which diverged from
population 1 at time \texttt{t3} was a parent in the admixture event
occurring at  time \texttt{t2}. Note also that the first line must include the effective sizes of the \emph{five} populations.\\

\item The following three scenarii correspond to a classic invasion history from an ancestral population (population 1). In scenario 1, population 3 is derived from population 2, itself derived from population 1. In scenario 2, population 2 derived from population 3, itself derived from population 1. In scenario 3, both populations 2 and 3 derived independently from population 1. The same trio of scenarii will be taken later in a fully described example.
Note that when a new population is created from its ancestral population, there is an initial size reduction (noted here \texttt{N2b} for population 2 and  \texttt{N3b} for population 3) since the invasive population generally starts with a few immigrants. 
\end{enumerate}
\medskip \ \ \ \ \ \ \ \  Scenario 1
\begin{center} 
\begin{tabular}{cc}
\includegraphics[scale=0.5]{code_scenario_05-1.pdf} & \includegraphics[scale=0.4]{test3pop_scenario_1.png} \\
\end{tabular}
\end{center}

 \ \ \ \ \ \ \ \  Scenario 2
\begin{center} 
\begin{tabular}{cc}
\includegraphics[scale=0.5]{code_scenario_05-2.pdf} & \includegraphics[scale=0.4]{test3pop_scenario_2.pdf} \\
\end{tabular}
\end{center}

\medskip \ \ \ \ \ \ \ \  Scenario 3
\begin{center} 
\begin{tabular}{cc}
\includegraphics[scale=0.5]{code_scenario_05-3.pdf} & \includegraphics[scale=0.4]{test3pop_scenario_3.pdf} \\
\end{tabular}
\end{center}


\newpage
\subsection{Mutation model parameterization (microsatellite and DNA sequence loci)}
The program can analyse microsatellite data and DNA sequence data altogether as well as separately. In the current version, there  are still two restrictions. First, all loci in an analysis must be genetically independent. Second, for DNA sequence loci, intralocus recombination is not considered.\\

Loci are grouped by the user according to its needs (this an improvement of the current version which imposed all loci of a given category to follow the same mutation model). A different mutation model can be defined for each group. For instance, one group can include all microsatellites with motifs that are 2 bp long and another group those with a 4 bp long motif. Also, with DNA sequence loci, nuclear loci can be grouped together and a mitochondrial locus form a separate group.\\ 
 
The parameterization of the two categories of markers is now described below. 

\subsubsection{Microsatellite loci}
 Although a variety of mutation models have been
proposed for microsatellite loci \citep{W2003}, it is usually
sufficient to consider only the simplest models \citep{C2006}. This
has the non-negligible advantage of reducing the number of parameters, which
can be a real issue when complex scenarios are considered. This is why we chose the Generalized Stepwise Mutation model \citep{E2002}. Under this model, a mutation increases or decreases the length of the microsatellite by a number of repeated motifs  following a geometric
distribution. This model necessitates only two parameters : the mutation rate  (\texttt{$\mu$}) and the parameter of the
geometric distribution (\texttt{$P$}). The same mutation model is imposed to all loci of a given group. However, each locus has its own parameters   (\texttt{$\mu_i$} and \texttt{$P_i$}) and, following a hierarchical scheme, each locus parameter is drawn from a gamma distribution with mean equal to the mean parameter value. Note also that :
\begin{enumerate}
\item individual loci parameters (\texttt{$\mu_i$} and \texttt{$P_i$}) are considered as nuisance parameters and hence are never recorded. Only mean parameters are recorded.
\item The variance or shape parameter of the gamma distributions are set by the user and are NOT considered as parameters.
\item The SMM or Stepwise Mutation Model is a special case of the GSM in which the number of repeats involved in a mutation is always one. Such a model can be easily achieved by setting the maximum value of mean P ($\bar{P}$) to 0. In this case, all loci have their $P_i$ set equal to 0 whatever the shape of the gamma distribution.
\item All loci can be given the same value of a parameter by setting the shape of the corresponding gamma distribution to 0 (this is NOT a limiting case of the gamma, but only a way of telling the program). 
\end{enumerate}
Eventually, to give more flexibility to the mutation model, the program offers the possibility to consider mutations that insert or delete a single nucleotide to the microsatellite sequence.  
In the previous version, this option was considered as marginal, and was not treated in the same way as the motif size stepwise mutational process, i.e. there was no associated parameter that could be adjusted to the data. This has been changed in this version :  it is now possible to use a mean parameter (named $ \mu_{(SNI)}$) with a prior to be defined and individual loci having either values identical to the mean parameter or drawn from a Gamma distribution.

\subsubsection{DNA sequence loci}

Note first that this version of the program does not consider insertion-deletion mutations, mainly because there does not seem to be much consensus on this topic. Concerning substitutions, only the simplest models are considered. We chose the Jukes-Cantor (1969) one parameter model, the Kimura (1980) two parameter model, the Hasegawa-Kishino-Yano (1985) and the Tamura-Nei (1993) models. The last two models include the ratios of each nucleotide  as parameters. However, in order to reduce the number of parameters, these ratios have been fixed to the values calculated from the observed data set for each DNA sequence locus. Consequently, this leaves two and three parameters for the Hasegawa-Kishino-Yano (HKY) and Tamura-Nei (TN), respectively.
Also, two adjustments are possible : one can fix the fraction of constant sites (those that cannot mutate) on the one hand and the shape of the Gamma distribution of mutations among sites on the other hand.\\
As for microsatellites, all sequence loci of the same group are given the same mutation model with mean parameter(s) drawn from priors and each locus has its own parameter(s) drawn from a Gamma distribution (same hierarchical scheme). Notes 1, 2 and 4 of previous subsection (2.3.1) apply also for sequence loci.

\subsection{SNPs do not require mutation model parameterization}

SNPs have two characteristics that allow to get rid of mutation models : they are polymorphic and they present only two allelic (ancestral and derived) states. In order to be sure that all analyzed SNP loci have the two characteristics, non polymorphic loci are disgarded right from the beginning of analyses. Consequently, no matter \emph{how} it occurred, we can assume that there occured one and only one mutation in the coalescence tree of sampled genes. We will see below that this  largely simplifies (and speeds up) SNP data simulation. Also, this advantageously reduces the dimension of the parameter space (as mutation parameters are not needed in this case). There is however a potential drawback which is the absence of any calibration generally brought by priors on mutation parameters. Consequently, (time/effective size) ratios rather than original time parameters will be informative.  

\subsection{Prior distributions}
The Bayesian aspect of the ABC approach implies that parameter
estimations use prior knowledge about these parameters, prior knowledge given by prior distributions of parameters. The program
offers a choice among usual probability distributions, i.e.
Uniform, Log-Uniform, Normal or Log-Normal for historical parameters and Uniform, Log-Uniform or Gamma
 for mutation parameters. Extremum values and other parameters (e. g. mean and
standard deviation) must be filled in by the
user.  \\
In addition, one can impose some simple conditions on historical
parameters. For instance, there can be two times parameters with
overlapping prior distributions. However, we want that the first
one, say \texttt{t1}, to always be larger than the second one, say
\texttt{t2}. For that, we just need to set \texttt{t1} $>$
\texttt{t2} in the corresponding edit-windows. Such a condition
needs to be between two parameters (not a parameter and a number,
though this can be set up by giving a minimum and a maximum to the
prior distribution) and more precisely between two parameters of the same category (i.e. two effective sizes, two times or two admixture rates). The limit to the number of conditions is
imposed by the logics, not by the program. The only binary
relationships accepted here are $>, <, >= and <=$.

\subsection{Algorithms for data simulation : main features}
Data simulation is based on the Wright-Fisher model. It consists in generating the genealogy of all sampled genes until their most recent common ancestor using coalescence theory.\\ 
This begins by randomly drawing a complete set of
parameters from their own prior distributions and that satisfy all imposed conditions. 
Then, once events have been ordered by increasing times, a sequence
of \emph{actions} is constructed. If there are more than one locus,
the same sequence of actions is used for all successive loci.
Possible \emph{actions} fall into four categories :
\begin{description}
  \item[adding a sample to a population] :\\
Add as many gene lineages to the population as there are genes in
the sample.
  \item[merge two populations] : \\
Move the lineages of the second population into the first
population.
  \item[split between two populations]: \\ Distribute the lineages of the admixed population among the two
parental populations according to the admixture rate.
  \item[coalesce and mutate lineages within a population]: \\
There are two possibilities here, depending on whether the
population is \emph{terminal} or not. We call \emph{terminal} the
population including the most recent common ancestor of the whole
genealogy. In a terminal population, coalescences and mutations stop
when the MRCA is reached whereas in a non terminal population,
coalescence and mutations stop when the upper (most ancient) limit
is reached. In the latter case, coalescences can stop before the
upper limit is
reached because there remains a single lineage, but this single remaining lineage can still mutate.\\
Two different algorithms are implemented : a
generation by generation simulation or a continuous time simulation.
The choice, automatically performed by the program, is based on an empirical criterion which ensures that the (approximate\footnote{The terms \emph{approximate} and \emph{exact} are relative to the basic assumptions of the Wright-Fisher model, not to the biological reality of the process.}) continuous time algorithm is chosen whenever it is faster than the (exact\addtocounter{footnote}{-1}\footnotemark) generation by generation while keeping the relative error on the coalescence rate below 5\% (see  \citet{C2008} for a description of this criterion).\\
 In any case, a coalescent tree is generated over all
 sampled genes.\\
Then the simulation process diverges depending on the type of markers : 
for microsatellite or DNA sequence loci, mutations are distributed over the branches according to a Poisson process whereas for SNP loci, one mutation is applied to a single branch of the coalescent tree, this branch being drawn at random with probability proportional to its length.\\
Eventually, starting from an ancestral allelic state (established as explained below), all allelic states of the
 genealogy are deduced forward in time according to the mutation
 process. For microsatellite loci, the ancestral allelic state is taken at random in the stationary distribution of the mutation model (not considering potential single nucleotide indel mutations). For DNA sequence loci, the procedure is slightly more complicated. First, the total number of mutations over the entire tree is evaluated. Then according to the proportion of constant sites and the gamma distribution of individual site mutation rates, the number and position of mutated sites are generated. Finally, these mutated sites are given 'A', 'T', 'G' or 'C' states according to the selected mutation model. For SNP loci, the ancestral allelic state is arbitrarily set to 0 and it becomes equal to 1 after le the mutation.\\
 Each category of loci has its own coalescence rate deduced from male and female effective population sizes . In order to combine different categories (e.g. autosomal and mitochondrial), we have to take into account the relationships among the corresponding effective population sizes. This can be achieved by linking the different effective population sizes to the effective number of males ( $N_M$ ) and females  ($N_F$) through the sum $N_T=N_F+N_M$ and the ratio $r = N_M/(N_F+N_M)$.  We use the following formulae for the probability of coalescence of two lineages within this population :
 \begin{description}
 \item [autosomal diploid loci :] $p=\frac{1}{8r(1-r)N_T}$
 \item [autosomal haploid loci :] $p=\frac{1}{4r(1-r)N_T}$
 \item [X-linked loci / haplo-diploid loci :] $p=\frac{1+r}{9r(1-r)N_T}$
 \item [Y-linked loci :] $p=\frac{1}{rN_T}$
 \item [Mitochondrial loci :]  $p=\frac{1}{(1-r)N_T}$
 \end{description}
Users have to provide a (total) effective size $N_T$ (on which inferences will be made) and a sex-ratio $r$.
 If no sex ratio is provided, the default value of $r$ is taken as 0.5.  
   

\end{description}
\subsection{Summary statistics}
For each category (microsatellite, DNA sequences or SNP) of loci, the program proposes a series of  summary statistics among those used by population geneticists. These summary statistics are mean values or variances over loci of the same group and 
characterize a single, a pair or a trio of population samples. These are :
\subsubsection{for microsatellite loci}
\begin{description}
\item[Single sample statistics] : 
\begin{enumerate}
  \item mean number of alleles across loci 
  \item mean gene diversity across loci \citep{N1987}
  \item mean allele size variance across loci 
  \item mean M index across loci \citep{GW2001, Ex2005}
 \end{enumerate}
\item[Two sample statistics] :
\begin{enumerate}
  \item mean number of alleles across loci (two samples)
  \item mean gene diversity across loci (two samples)
  \item mean allele size variance across loci (two samples)
  \item $F_{ST}$ between two samples \citep{WC1984}
  \item mean index of classification (two samples) \citep{RM1997,PC2007}
  \item shared allele distance between two samples \citep{CJ1993}
  \item $(\delta\mu)^2$ distance between two samples \citep{GL1995}
  \end{enumerate}
\item[Three sample statistics] :
\begin{enumerate}
  \item Maximum likelihood coefficient of admixture  \citep{CF2004} 
\end{enumerate}
\end{description}

\subsubsection{for DNA sequence loci}

\begin{description}
\item[Single sample statistics] : 
\begin{enumerate}
  \item number of distinct haplotypes 
  \item number of segregating sites
  \item mean pairwise difference
  \item variance of the number of pairwise differences
  \item Tajima's D statistics \citep{TA1989}
  \item Number of private segregating sites (=number of segregating sites if there is only one sample)
  \item Mean of the numbers of the rarest nucleotide at segregating sites\footnote{This statistics can provide information in case of recent demographic variation : a recent expansion increases the number of singletons (nucleotides occuring just once at a segregating site) resulting in a low value of this statistics, whereas a recent decline will produce an opposite result.}
  \item Variance of the numbers of the rarest nucleotide at segregating sites
 \end{enumerate}
\item[Two sample statistics] :
\begin{enumerate}
  \item number of distinct haplotypes in the pooled sample
  \item number of segregating sites in the pooled sample
  \item mean of within sample pairwise differences
  \item mean of between sample pairwise differences
  \item $F_{ST}$ between two samples \citep{H1992}
  \end{enumerate}
\item[Three sample statistics] :
\begin{enumerate}
  \item Maximum likelihood coefficient of admixture  \citep[adapted from][]{CF2004}
\end{enumerate}
\end{description}

\subsubsection{for SNP loci}
\begin{description}
\item[Single sample statistics] : 
\begin{enumerate}
  \item proportion of loci with null gene diverty (= proportion of monomorphic loci) 
  \item mean gene diversity across polymorphic loci \citep{N1987}
  \item variance of gene diversity across polymorphic loci 
  \item mean gene diversity across all loci
 \end{enumerate}
\item[Two sample statistics] :
\begin{enumerate}
  \item proportion of loci with null $F_{ST}$ distance between the two samples \citep{WC1984}
  \item mean across loci of non null $F_{ST}$ distances between the two samples 
  \item variance across loci of non null $F_{ST}$ distances between the two samples
  \item mean across loci of $F_{ST}$ distances between the two samples
  \item proportion of loci with null Nei's distance between the two samples \citep{N1972}
  \item mean across loci of non null Nei's distances between the two samples 
  \item variance across loci of non null Nei's distances between the two samples
  \item mean across loci of Nei's distances between the two samples
  \end{enumerate}
\item[Three sample statistics] :
\begin{enumerate}
  \item proportion of loci with null admixture estimate
  \item mean across loci of non null admixture estimate
  \item variance across loci of non null admixture estimated
  \item mean across all locus admixture estimates
\end{enumerate}
\end{description}

\subsection{Pre-evaluation of scenarios and prior distributions}
This option is proposed to users since version 1.0. The purpose is to check that at least one combination of scenarios and priors can produce simulated data sets that are close enough to the observed data set. This is performed through two kinds of analyses. In the first one, a principal component analysis is performed in the space of summary statistics on at most 100,000 simulated data set and the observed data is added on each plane of the analysis in order to evaluate how the latter is surrounded by simulated data sets. In addition to this global approach, there is a second one in which each summary statistic of the observed data set is ranked against those of the simulated data set. This second analysis helps finding which aspects of the model (including prior) have been mistated. For instance, a grossly overestimated genetic distance (in simulated data sets compared to the observed one) may suggest a mispecification of the prior distribution of the time of divergence of the two involved populations or of the mean mutation rate of the markers. Using this new option before running a full ABC treatment is a convenient way to reveal mispecification of models (scenarios) and/or prior distributions of parameters \citep[see][for an illustration]{C2010}

\subsection{Estimation of posterior distributions of parameters}
Several steps are necessary to get posterior distributions of
parameters. First, the normalized Euclidian distance between the observed data set and
each simulated data set is computed as the sum of squared
differences of summary statistics weighted by the inverse of their
variance in the entire set of simulated data. For the $i$-th data set,
the distance is :
\begin{equation}\label{eq1}
    d_i=\sqrt{\sum_{j=1}^{nstat}\frac{(s_{ij}-s_j^{obs})^2}{V_j}}
\end{equation}
in which $s_{ij}$ is the $j$-th summary statistics from the $i$-th
data set, $s_j^{obs}$ is the $j$-th summary statistics from the
\emph{obs}erved data set and $V_j$ is the variance of the the $j$-th
summary statistics across all simulated data sets.
 Only the closest data sets are selected for further
treatments. The latter includes a weighted local linear regression step aimed
at improving the posterior distributions of the parameters \citep{B2002}. Basically, a
multiple linear regression is performed in which summary statistics
are the independent variables and parameters the dependent
variables. But this regression is also \emph{local} in the sense
that more weight in the regression is given to data sets that are
closest to the observed data set. This is performed by using a
kernel function (the Epanechnikov kernel following \citet{B2002} :
\begin{equation}
\operatorname{K_{\delta}}(d) = \left\{
\begin{array}{ll}
(1.5/\delta)(1-(d/\delta)^2), & t \leq \delta \\
0, & t > \delta \\
\end{array}\right.
\end{equation}
Eventually, parameters are adjusted through this process as :
\begin{equation}\label{eq2}
    \phi_{ik}^*=\phi_{ik}-(\textbf{s}_i-\textbf{s}^{obs})\bm{\beta}_k
\end{equation}
in which $\phi_{ik}$ is the $k$-th parameter of the $i$-th selected
data set, $\phi_{ik}^*$ is the adjusted corresponding parameter,
$\textbf{s}_i$ is the row vector of summary statistics of the $i$-th
selected data set, $\textbf{s}^{obs}$ is the row vector of summary
statistics of the observed data set and $\bm{\beta}_k$ is the
transposed $k$-th row vector of the regression coefficient
matrix.\par The adjusted $\phi_{ik}^*$ of the selected data sets are
an approximate sample of the posterior distribution of parameters \citep{B2002}.

\subsection{Model checking}
\emph{Checking the model is crucial to statistical analysis}  \citep[p161 in][]{GCSR1995}. Model checking (i.e. the assessment of the �goodness-of-fit� of a model � parameter posterior combination) is a facet of ABC analysis that has been so far neglected (but see Ingvarsson,  2008). Following Gelman et al. (1995; pp 159-163), we already implemented this option in $DIYABCv1.0$,  to measure the discrepancy between a model � parameter posterior combination and a �real� data set by considering various sets of test quantities. These test quantities can be chosen among the large set of ABC summary statistics proposed in the program. This option is based on the same kinds of analysis as section 2.7. The main difference is the set of simulated data. Whereas in section 2.7, prior distributions of parameters have been used to simulate data sets, here we use posterior distributions of the same parameters, hence simulating data from the \emph{posterior predictive distribution}. \\
The first analysis is a principal component analysis in the space of summary statistics using data sets simulated with the \textbf{prior} distributions of parameters (exactly as in section 2.7) and the observed data as well as \textbf{data sets from the posterior predictive distribution} are represented on each plane of the PCA. If the model fits well the data, one should see on each PCA plane a wide cloud of data sets simulated from the prior, with the observed data set in the middle of a small cluster of datasets from the posterior predictive distribution.\\
In the second analysis, each summary statistics of the observed data set is ranked against the distribution of the corresponding summary statistics from the posterior predictive distribution. Summary statistics play here the role of \emph{test statistics} \citep[p169 in][]{GCSR1995}.\\
 Since summary statistics are generally not sufficient, it is advised to use different sets of summary statistics to compute the posterior distribution of parameters on one hand and to check the model on the other hand \citep[see][]{C2010}. This has been implemented in DIYABC. 

\subsection{Measures of performances}
As stressed in previous studies \citep[e.g.][]{Ex2005}, the ABC appproach provides an efficient way of assessing its own performances for estimating posterior distributions of parameters. The reference table, the building of which represents generally 95 to 99\% of the computing time, can be reused to analyse pseudo-observed (test) data sets obtained through simulation with known values of parameters. It is then rather quick and easy to evaluate the performance of the method for parameter estimation by computing statistics such as estimation biases or mean square errors.\\
These measures of performance have been fully integrated into DIYABC. The performance measures computed by DIYABC are :
\begin{description}
\item[the average relative bias] : the difference between the point estimate ($e$) and the true value ($v$)divided by the true value, 
 $\frac{1}{n}\sum_{i=1}^n \frac{e_i - v_i}{v_i}$, averaged over the $n$ test data sets,\\
  \item[the square Root of the Relative Mean Square Error (RRMSE)]: the square root of the average square difference between the point estimate and the true value, divided by the true value,
 $\sqrt{\frac{1}{n}\sum_{i=1}^n(\frac{e_i-v_i}{v_i})^2}$\\
 \item[the square Root of the Relative Mean Integrated Square Error (RRMISE)] : the square root of the average (over test data sets) of the integrated square error (measured on each test data set) divided by the true value,
 $\sqrt{\frac{1}{n}\sum_{i=1}^n(\frac{\sum_{j=1}^{m_i} (x_{ij}-v_i)^2}{m_iv_i^2})}$,
  $x_{ij}$ and $m_i$ being the sampled values and the sample size of the posterior distribution in the $i$-th test data set, respectively.\\
 \item[the Relative Mean Absolute Deviation (RMAD)] : the average (over test data sets) of the mean absolute deviation (measured on each data set), divided by the true value,
 $\frac{1}{n}\sum_{i=1}^n(\frac{\sum_{j=1}^{m_i} \arrowvert x_{ij}-v_i \arrowvert}{m_i \arrowvert v_i\arrowvert}$\\

 \item[the 50\% and 95\% coverages] : the proportion of test data sets for which the 50\% and 95\% credibility intervals respectively include the true value.\\
 \item[the factor 2] :the proportion of test data sets for which the point estimate is at least half and at most twice the true value.
 \item[the Relative Median Bias (RMB)] : the 50\% quantile of the  bias (measured on each test data set) divided by the true value. The bias is computed respectively for each point estimate 
 \item[the Relative Median Absolute Deviation (RMedAD)] : the 50\% quantile (over test data sets) of the median (over each data set) of the absolute difference between each value of the posterior distribution sample and the true value divided by the true value.
 \item[the Relative Median of the Absolute Error (RMAE)] :  the 50\% quantile (over test data sets) of the absolute value of the difference between the point estimate (in each data set) and the true value divided by the true value.
\end{description} 
~\
DIYABC considers the following three point estimates : mean, median and mode of the  $\phi_{ik}^*$  (sample of the posterior distribution of each parameter), as defined in subsection 1.7.
~\\
 Concerning the true value ($v$) appearing in the above formulae, DIYABC offers two possibilities :
 \begin{enumerate}
 \item All values $v$ are fixed by the user. If any one of these values is outside the limits given to the prior for the corresponding parameter, a warning message is issued but the analysis can proceed if needed.
 \item All values $v$ are drawn from distributions. These distributions can be different from those of priors. They may even not be overlapping (no warning message is issued whatever the user's choice). 
 \end{enumerate}
 If you want to fix some parameter values and draw the other from distributions, choose the second option and give the same desired values as minimum and maximum for those fixed parameter values.\\

In order to better assess the information brought by genetic data, DIYABC provides a double estimate of all these bias/precision statistics. As expected, the first one is based on genetic data given in the data file. The second one is computed as if there was no genetic information, $i.e.$ estimates are based only on parameter priors. Technically, a sample of parameter values is drawn at random from the reference table. This sample of the same size of the sample of posterior values is used in place of the latter in all computations.

\subsection{Comparison of scenarios}
The ABC approach can also be used to compare possible scenarios for the same data file through the computation of the posterior probabilities of each scenario and this option is naturally implemented in DIYABC.\\

\subsubsection{Reference table} 
First, the reference table can include as many scenarios as desired. By default, the prior probability of each scenario is uniform, that is each scenario will have approximately the same number of simulated data sets. But, if for any reason, one wants a different prior probability for each scenario, there is the possibility to do so.\\

Scenarios are drawn according to their own prior probability and then only parameters that are defined for the drawn scenario are generated from their respective prior distribution. Scenarios may or may not share parameters.\\
When conditions apply to some parameters (see subsection 2.4), the program provides the possibility of choosing between two options :
\begin{enumerate}
\item parameter sets are drawn in their respective prior distributions until all conditions are fulfilled.
\item a single parameter set is drawn and only if all condition are fulfilled, the simulation is performed and the data set is recorded in the reference table.
\end{enumerate}
When there is only one scenario, both options are equivalent, although in option 2, there might be less simulated data sets that are recorded than one asked. When there is more than one scenario, the second option can be viewed as a way to set prior probabilities on scenario that result from imposed conditions on parameters (see \citet{ME2005} for an example).  

\subsubsection{Posterior probability of scenarios}
The program DIYABC provides two estimates of the posterior probability of each scenario :
\begin{description} 
\item[a \emph{direct}  estimate :] This is simply the number of times that a given scenario is found in the first $n_{\delta}$ simulated data sets once the latter, produced under several scenarios, have been sorted by ascending distances to the observed data set ($i.e.$ the ``closest'' simulated data sets).\\
\item[a logistic regression estimate :] Following M.A. Beaumont's suggestion \citep{FR2007,B2008}, a polychotomic weighted logistic regression is performed on the first  $n_{\delta}$ data sets with the proportion of the scenario as the dependent variable and the differences between observed and simulated data set summary statistics as the independent variables. The intercept of the regression (corresponding to an identity between simulated and observed summary statistics) is taken as the point estimate. In addition, 95\% confidence intervals are computed \citep{C2008}. 
\end{description} 

Since both estimates are dependent upon the chosen threshold ($\delta$), the program provides a range of 100 estimates for the direct approach (for each one 100-th of $n_{\delta}$ between 0 and $n_{\delta}$) and up to 10 estimates for the logistic regression estimates (e.g. one estimate for $kn_{\delta}/10$ with $k \in[1,2,...10]$ when the number of analyses is set to 10). These estimates are represented in two graphs, one for each kind of estimate. These two graphs can be printed and/or saved (in \emph{svg}, \emph{jpg}, \emph{png} or \emph{pdf} format). Values can also be output as a text file.\

In $DIYABCv2.0$, a new possibility is offered to the user that may be useful when dealing with many summary statistics and many scenarios. In this particular case, the logistic regression has to deal with large matrices and the amount of needed memory on one hand and the computation time on the other hand can become problematically large. An approximate solution is to replace summary statistics by the components of a linear discriminant analysis which reduces the number of independent variables to the smallest of number of summary statistics and scenarios. Although the result is only approximate, it can be a useful guide in some specific cases. The gain in time can be large. For instance, the time can be reduced by a 100X factor \citep{EL2012}. \\

\subsubsection{Confidence in scenario choice}
The program DIYABC offers a last option that allows one to evaluate the confidence in a scenario choice. Suppose that we compare 3 scenarios for a given data set and that e.g. scenario 2 had maximum posterior probability. By using this option, we can estimate type I and type II errors when choosing scenario 2 as the true scenario. To do so, we simulate a given number of data sets according to scenario 1, 2 and 3. Then we count the proportion of times that scenario 2 has not the highest posterior probability among the three competing scenarios when it is the true scenario (type I error, estimated from test data sets simulated under scenario 2) or the proportion of times that scenario 2 has highest posterior probability when it $not$ the true scenario (type II error, estimated from test data sets simulated under scenarios 1 and 3); in this case, we have two type II error values, one associated to scenario 2 and one to scenario 3.\\

As for the bias/precision analysis, parameter values can be fixed to given values or drawn from given distributions (not necessarily the same as those used as priors for the reference table).

\clearpage

\section{The Graphic User Interface}

When launching the GUI, the home screen appears like this :\\


\includegraphics[scale=0.4]{gui_pictures/Capture-DIYABC-1.png} 

You can already notice that DIYABC works with projects. This notion is new to version 2 of DIYABC. It is explained in subsection 3.1.

\subsection{What is a DIYABC Project ?}

\label{doc_openProjectButton}
A $ DIYABC $ project is a unit of work materialized by a specific and unique directory. A project is defined by at least one observed data set and one reference table header file. These files are located in the \emph{Project directory} which name includes an identifier, the date of creation and a number (between 1 and 100).\\

The header file, always named \texttt{header.txt}, contains all information necessary to compute a reference table associated with the data : i.e. the scenarios, the scenario parameter priors, the characteristics of loci, the loci parameter priors and the summary statistics to compute.
As soon as the first records of the reference table have been saved in the reference table file,  always named \texttt{reftable.bin} and also included in the project directory, the project is "locked". This means that the header file can not be changed anymore. If one needs to change a scenario or a parameter prior, or a summary statistics, a new project needs to be defined. This is to guarantee that all subsequent actions performed on the project are in coherence with the current data and header files. It is of course strongly advised NOT to move files among projects.
Incidentally, the \texttt{header.txt} file is only built when the project has been saved, the information progressively input by the user being saved in a series of temporary files.\\

Once a sufficiently large reference table has been simulated, analyses can be performed. Their different output files are copied to the \emph{analysis} directory included in the project directory, and containing as many directories as analyses performed. Hence, it is now much easier to know with certainty the conditions of each analysis.    

\subsection{Options of the home screen}
\label{doc_newSNPProjectButton}
The home screen above has two menus and several buttons.\\ 
Let's start with the menus. Below are shown all submenus :

\begin{center} 
\begin{tabular}{cc}
\includegraphics[scale=0.5]{gui_pictures/Capture-DIYABC-2.png} & \includegraphics[scale=0.5]{gui_pictures/Capture-DIYABC-3.png}\\
\end{tabular}
\end{center}

The \texttt{File} menu has seven options, namely \texttt{New project}, \texttt{Open project}, \texttt{Open recent projects}, \texttt{Save all projects}, \texttt{Settings}, \texttt{Simulate data set(s)} and \texttt{Quit}. All are self explanatory.\\
The \texttt{Help} menu has two options : \texttt{About DIYABC} which opens up a small window providing the names and address of the authors and \texttt{Show logfile} which gives access to a logfile viewer in which are recorded all actions and messages about the execution of the GUI.\\ 
Just below the menu are five shortcuts to main \texttt{File} menu options.\\

\includegraphics[scale=0.4]{gui_pictures/Capture-DIYABC-4.png}\\
 
On the right, the field \texttt{What's this ?} is an another way to get help on a specific GUI object :\\
 
\includegraphics[scale=0.4]{gui_pictures/Capture-DIYABC-5.png}\\ 

Eventually, below the logo, there are three buttons which are duplicate shortcuts :\\
 
\includegraphics[scale=0.4]{gui_pictures/Capture-DIYABC-6.png}\\ 


  
%Clicking on the \fbox{\textsf{Open project}} button opens up the following frame:\\

%\includegraphics[scale=0.4]{gui_pictures/Capture-DIYABC-5.png} 

%To select a project, you just double click on the corresponding directory.
%\newpage
% The following screen then appears :\\

%\includegraphics[scale=0.35]{gui_pictures/Capture-DIYABC-6.png} 

%We will go back later to the description of this screen.
%\begin{itemize}
% \item 
%  Cliking on the \fbox{\textsf{New project}} opens up the screen below requiring a name for the new project :\\
%\begin{center}
%\includegraphics[scale=0.35]{gui_pictures/Capture-DIYABC-7.png} 
%\end{center}
%\item
%After giving a name to new project and cliking on \fbox{\textsf{OK}}, the following screen appears :\\ 

%\includegraphics[scale=0.35]{gui_pictures/Capture-DIYABC-6.png} 

%\end{itemize}

\subsection{Defining a new project}
Defining a new project requires different steps which are not the same whether the data are SNPs or microsatellites/DNA sequences (MSS). 
Let start with an MSS project : click on one of the following :
\begin{itemize}
 \item \texttt{File} menu $>$ \texttt{New project} $>$ \texttt{Microsatellites and/or sequences}
 \item the menu shortcut \texttt{New MSS}
 \item the bottom left button \fbox{\textsf{New Microsat/Sequence project}}
\end{itemize}
or press simultaneously the \texttt{Control} and \texttt{M} keys.\\
A new window appears in which the user can choose a location and a name for the new project as shown below :\\

\includegraphics[scale=0.35]{gui_pictures/Capture-DIYABC-7.png} 

Let's enter \textbf{demo1} as the project name and click on the \fbox{\textsf{Create project}} button.\\
The following screen appears :\\

\includegraphics[scale=0.35]{gui_pictures/Capture-DIYABC-8.png} 

The \textbf{demo1} project and all its future files will be located in the directory \texttt{demo1\_2012\_5\_31-1}.

\subsubsection{Step 1 : choosing the data file}
We next need to choose the data file of the project. This is performed by clicking on the corresponding \fbox{\textsf{Browse}} button (previous screen).  The usual file browsing screen appears (below) and one has to select a Genepop format data file, here \texttt{data1.mss}. \\

\includegraphics[scale=0.35]{gui_pictures/Capture-DIYABC-9.png} 
\\
Clicking on the \fbox{\textsf{Open}} button leads  to the following screen with the edit field filled with the name of the data file and some characteristics of this data file appearing on the screen (number of loci, individuals and samples).\\
Below these fields are two panels indicating that we need to provide information about the Historical model (left panel) and about the Genetic data and associated Summary statistics (right panel). The red crosses on both panels will change to green checks once the corresponding information will be completed.\\ 
\\
\\
\includegraphics[scale=0.35]{gui_pictures/Capture-DIYABC-10.png} 
\newpage
\subsubsection{Inform the Historical model}
Click on the corresponding \fbox{\textsf{Set}} button. The following screen, familiar to users of previous versions, appears:\\

\includegraphics[scale=0.35]{gui_pictures/Capture-DIYABC-11.png} 

Let's enter a simple scenario in scenario 1 edit window and click on the \fbox{\textsf{Define priors}} button. We get this :\\

\includegraphics[scale=0.35]{gui_pictures/Capture-DIYABC-12.png} 

The parameter prior frame allows to choose the prior density of each parameter. A parameter is anything in the scenario that is not a keyword (here \texttt{sample} and \texttt{merge}), nor a numeric value. In our example scenario, parameters are hence : \texttt{N1, N2, N3, t1} and \texttt{t2}. In our example, we need to set the priors on \texttt{t1} and \texttt{t2} such that \texttt{t2}$>$  \texttt{t1}. We can do it either by using the \texttt{set condition} button or by playing with the minimum and maximum values of the two parameters.\\

If we click on the \fbox{\textsf{Check scenario}} button, the logic of the scenario is checked and if it is found OK, and if the scenario is drawable, the drawing appears on a new frame : \\

\includegraphics[scale=0.35]{gui_pictures/Capture-DIYABC-13.png} 

The scenario can be saved by clicking on the \fbox{\textsf{SAVE}} button. The frame can be close by clicking on the \fbox{\textsf{CLOSE}} button.\\

Since the scenario has been checked, we can validate and save the historical model by clicking on the  \fbox{\textsf{VALIDATE AND SAVE}} button (bottom screen of p 21). We then go back to the project screen in which the historical model has now received the green check sign.\\

\includegraphics[scale=0.35]{gui_pictures/Capture-DIYABC-14.png} 

 \newpage
\subsubsection{Inform the Genetic model}

Click on the corresponding \fbox{\textsf{Set}} button. We get the following screen : \\

\includegraphics[scale=0.35]{gui_pictures/Capture-DIYABC-15.png} 

On the left part of the screen, there is the list of loci, with their type (M for microsatellites or S for DNA sequences) and the motif size and allelic range for microsatellite loci only. Actually, the values for motif size and allelic range are just default values and do not necessarily correspond to the actual data. The user who knows the real values for its data is required to set the correct values at this stage. If the range is too short to include all values observed in the analysed dataset, a message appears in a box asking to enlarge the corresponding allelic range. Note that the allelic range is measured in number of motifs, so that a range of 40 for a motif length of 2 bp means that the difference between the smallest and the longest alleles should not exceed 80 bp. It is worth stressing that the indicated allelic range (expressed in number of continuous allelic states) corresponds to a potential range which is usually larger than the range observed from the analyzed dataset (cf. all possible allelic states have usually not been sampled). In practice it is difficult to assess the actual microsatellite constraints on the allelic range; to do that one needs allelic data from several distantly related populations/sub-species as well as related species which is rarely the case \citep[see][]{PDD1998}; \citep{E2002}. We achieved a meta-analysis from numerous primer notes documenting the microsatellite allelic ranges of many (i.e. \textgreater 100) different species (and related species). We used the corrective statistical treatment on such data proposed by \citep{PDD1998}. Our results pointed to a mean microsatellite allelic range of 40 continuous states (hence the default allelic range value of 40 mentioned in the program). We also found, however, that range values greatly varied among species and among loci within species (unpublished results). We therefore recommend to use the following pragmatic behaviour when considering the allelic range of your analysed microsatellite dataset: (i) if the difference in number of motif of your locus is \textless 40 motifs in the analysed  dataset then leave the default allelic range value of 40. (ii) if the difference in number of motif of your locus is \textgreater 40 motifs in your dataset then take Max\_allele\_size $-$ Min\_allele\_size)/motif size + say 10 additional motifs to re-define the allelic range of the locus in the corresponding DIYABC panel (e.g. (200 nu $-$ 100 nu)/2 + 10 = 50 + 10 = 60 as allelic range).\\
We then need to define at least one group of loci by clicking on the \fbox{\textsf{Add group}} button. We get this :\\

\includegraphics[scale=0.35]{gui_pictures/Capture-DIYABC-16.png} 

Suppose we want all loci in the same group because we consider that they all have similar mutational modalities. We select them like in any table, extending the selection with the \texttt{Shift} and \texttt{Control} keys (see below) : \\

\includegraphics[scale=0.35]{gui_pictures/Capture-DIYABC-17.png} 

and then pressing the \fbox{\textsf{$ >> $}} button : \\

\includegraphics[scale=0.35]{gui_pictures/Capture-DIYABC-18.png}

Note that the \fbox{\textsf{Auto group}} button would have produced the same result of putting all the microsatellite loci in the same group.\\  

We then need to define the mutation model and the summary statistics of the locus group. Clicking on the \fbox{\textsf{Set mutation model}} button, the following screen appears :\\

\includegraphics[scale=0.35]{gui_pictures/Capture-DIYABC-19.png} 

Once the mutation model of Group 1 is defined, we click on the \fbox{\textsf{VALIDATE}} button to go back to the previous screen. Clicking on the \fbox{\textsf{Set Summary statistics}} button, we get the following screen :\\ 

\includegraphics[scale=0.35]{gui_pictures/Capture-DIYABC-20.png} 

We define summary statistics by checking the corresponding boxes :\\ 

\includegraphics[scale=0.35]{gui_pictures/Capture-DIYABC-21.png} 

Once finished, we click on the \fbox{\textsf{VALIDATE}} button to go back to the screen of p24. Now, we can validate also this screen which brings us back to the screen of p22. The latter looks now like this : \\

\includegraphics[scale=0.35]{gui_pictures/Capture-DIYABC-22.png} 

At that moment, the project directory includes the following files : a copy of the data file, and four configuration files : \texttt{conf.analysis}, \texttt{conf.gen.tmp}, \texttt{conf.hist.tmp}, \texttt{conf.tmp}. Note that the project is not yet saved. To save the project, we need either to save it explicitly by using the \texttt{File} menu (see below) or to start simulating data sets (next section). Saving the project results in saving the \texttt{header.txt} file in the project directory.


\includegraphics[scale=0.35]{gui_pictures/Capture-DIYABC-23.png} 

\subsection{Building the reference table}

Keeping on the current screen, indicate the required number of data sets to simulate for the reference table : \\ 

\includegraphics[scale=0.35]{gui_pictures/Capture-DIYABC-24.png} 


Then click on the \fbox{\textsf{Run computations}} button. If things go well, you will soon see the progress both into the edit window "Number of simulated data sets in the reference table" and in the progress bar below. Also, you have an estimate of the remaining time (at the left of the  \fbox{\textsf{Run computations}} button):\\

\includegraphics[scale=0.35]{gui_pictures/Capture-DIYABC-25.png} 

When the computation is finished, the screen looks like this :\\

\includegraphics[scale=0.35]{gui_pictures/Capture-DIYABC-26.png} 

\subsection{Performing analyses}

We have now eveything necessary to perform analyses. The current screen shows two tabs : \texttt{Reference table} and \texttt{Analyses}. Let's click on the \texttt{Analyses} tab. We get this new screen :\\

\includegraphics[scale=0.35]{gui_pictures/Capture-DIYABC-27.png} 

First, we need to define the analysis we want to perform. So we click on the \fbox{\textsf{Define new analysis}} button and get this new screen :\\

\includegraphics[scale=0.35]{gui_pictures/Capture-DIYABC-28.png} 

We need to choose among the six possible types of analyses (actually, only four of them are possible, since the reference table includes a single scenario). We decide to first check whether the model (scenario and parameter prior definition) is off the target or not. This can be appreciated through the analysis denominated \texttt{Pre-evaluate scenario prior combination}. To illustrate the result, we also ask for a principal component analysis by checking the corresponding square. Eventually, we give the name of \textbf{pre-eval1} to this first analysis. The screen now looks like this :\\

\includegraphics[scale=0.35]{gui_pictures/Capture-DIYABC-29.png}\\ 

After clicking on the \fbox{\textsf{VALIDATE}} button, we go back to the previous screen. However, the new analysis now appears on top of the analysis panel. For each analysis, this panel provides its name and type, the list of parameters that will be transmitted (in a coded way) to the computation program, a progress bar that approximates the progress of the analysis run, and four buttons. The right button has to be clicked to launch the analysis. The three left buttons provide a way to copy an analysis (\fbox{\textsf{Copy}} button), to make some modifications (\fbox{\textsf{Edit}} button) before launching it or to delete the analysis (\fbox{\textsf{Del}} button).\\
 

\includegraphics[scale=0.35]{gui_pictures/Capture-DIYABC-30.png} \\

Let's click on the \fbox{\textsf{Launch}} button. This analysis is very fast (ca 1 second) so that the progress bar shows almost immediately a 100\% value : \\

\includegraphics[scale=0.35]{gui_pictures/Capture-DIYABC-31.png} \\

To view results, just click on the \fbox{\textsf{View results}} button. After some seconds (while the program reads the PCA result file), we can see this : \\

\includegraphics[scale=0.35]{gui_pictures/Capture-DIYABC-32.png} \\

The results are shown PCA plane by PCA plane. Each (small) dot represents a simulated dataset from the reference table and the large yellow dot represents the observed data set. The initial components of datasets are the values of the summary statistics from which are computed the principal components. The four drop-lists (\texttt{Scenario to draw}, \texttt{Horizontal axis component}, \texttt{Vertical axis component}, \texttt{Number of prior plots per scenario}) can be used to explore further the results of the PCA.\\
The graphic can be printed or saved (\fbox{\textsf{PRINT}} and \fbox{\textsf{SAVE}} buttons, respectively). Clicking on the \fbox{\textsf{CLOSE}} button closes the result window. Eventually, clicking on the \fbox{\textsf{View numerical results}} opens up another screen as shown below :\\

\includegraphics[scale=0.35]{gui_pictures/Capture-DIYABC-33.png} \\

This screen is obtained by computing for each summary statistics the proportion of simulated data (considering the total reference table) that have a value below the value of the observed dataset. A star indicates proportions lower than 5\% or greater than 95\% (two stars, $<$1\% or $>$1\%; three stars, $<$0.1\% or $>$0.1\%).\\ 

 As usual, results can be printed (\fbox{\textsf{PRINT}}) and/or saved (\fbox{\textsf{SAVE}}). Click on \fbox{\textsf{OK}} to leave this screen.\\

Although we get one star for a few summary statistics, we conclude that our model is suitable enough to proceed to other ABC analyses.

\subsubsection{ABC parameter estimation}

Back on the screen of page 30, we click on the \fbox{\textsf{Define new analysis}} button. We choose the \texttt{Estimate posterior distribution of parameters} option and we call \texttt{estim1} this second analysis :\\

\includegraphics[scale=0.35]{gui_pictures/Capture-DIYABC-34.png} \\
 
We click on the \fbox{\textsf{VALIDATE}} button and get the following screen in which we can choose the scenario to use for this estimation. Since a single scenario has been defined, there is nothing else to do than to click on the \fbox{\textsf{VALIDATE}} button : \\

\includegraphics[scale=0.35]{gui_pictures/Capture-DIYABC-35.png} \\
 
We get then the following screen in which we can make several choices :
\begin{itemize}
 \item on the left hand side, we can choose the number of closest simulated datasets that will be used for the local linear regression (cf section 2.1).
 \item below, we can select the transformation of parameter values that can generally improve the results (default = logit transformation).
 \item on the right hand, we can truncate the reference table to a specified number of datasets.
 \item eventually, estimations can be performed either on original ($i.e.$ raw) parameters, and/or combinations of parameters that are generally more estimable. \emph{Composite} parameters are products of effective population sizes or times by mean mutation rate whereas \emph{Scaled} parameters are ratios of effective population sizes or times by mean effective population size (computed from all terminal populations, $i.e.$ $N_1$, $N_2$ and $N_3$ in the present example). 
\end{itemize}

\includegraphics[scale=0.35]{gui_pictures/Capture-DIYABC-36.png} \\
 
Apart from the number of closest datasets that we set at 10,000 (although 1,000 would be also a correct choice), we keep all other default values and click on the \fbox{\textsf{VALIDATE}} button. We get back to the Analysis control panel which now looks like this:\\

\includegraphics[scale=0.35]{gui_pictures/Capture-DIYABC-38.png} \\

%\newpage
We click on the \fbox{\textsf{Launch}} button. The analysis progress is now visible :\\

\includegraphics[scale=0.35]{gui_pictures/Capture-DIYABC-38a.png} \\

As long as the analysis is not terminated, we could stop it by clicking on the  \fbox{\textsf{Stop}} button. Once this second analysis is finished, we can view its results by clicking on the  \fbox{\textsf{View results}} button :\\

\includegraphics[scale=0.35]{gui_pictures/Capture-DIYABC-39.png} \\

Let's have a look :\\

\includegraphics[scale=0.35]{gui_pictures/Capture-DIYABC-40.png} \\

In the scrolling window, we get graphics showing the prior (red curve) and posterior (green curve) distributions of all parameters. Below each graphics are statistics (mean, median, mode and quantiles) of the posterior distribution. The latter are grouped in a table that appears when clicking on the upper left \fbox{\textsf{view numerical results}} button, showing this :\\

\includegraphics[scale=0.35]{gui_pictures/Capture-DIYABC-41.png} \\

We go back to the previous screen by cliking the \fbox{\textsf{OK}} button.
\newpage
 We can also have results for \emph{Composite} or \emph{Scaled} parameters. Below is an example of \emph{Scaled} time parameters obtained by clicking on the \emph{Scaled} radio button and scrolling the graphs window to the right:\\
 
\includegraphics[scale=0.35]{gui_pictures/Capture-DIYABC-42.png} \\


\subsubsection{Bias and precision}
Let's define a new analysis (click on the \fbox{\textsf{Define new analysis}} button) and choose the option \texttt{Compute bias and precision on parameter estimations}. We give it the name \texttt{bias1} :\\

\includegraphics[scale=0.35]{gui_pictures/Capture-DIYABC-43.png} \\

In this kind of analysis, peudo-observed data are simulated with known values of parameters copying the exact configuration of the observed dataset in terms of sample sizes (taking into account missing data) and are submitted to the same ABC estimation process. If we assume that the evolutionary scenario is correct, the comparison of real and estimated values of parameters provide some information of the precision of the estimation process.\\  
We validate and get this screen :\\

\includegraphics[scale=0.35]{gui_pictures/Capture-DIYABC-44.png} \\

We choose to draw parameter values from distributions :\\

\includegraphics[scale=0.35]{gui_pictures/Capture-DIYABC-45.png} \\

Clicking on the \fbox{\textsf{VALIDATE}} button, we get this screen which allows us to choose distributions. \\

\includegraphics[scale=0.35]{gui_pictures/Capture-DIYABC-46.png} \\

By default, the screen suggests the prior distributions that have been used to build the reference table. However, these distributions can be edited if necessary. We decide not to change them and click on \fbox{\textsf{VALIDATE}} which brings us to the following screen :\\

\includegraphics[scale=0.35]{gui_pictures/Capture-DIYABC-47.png} \\

If we want to keep the same distributions for mutation parameters as when builduing the reference table, we just click on \fbox{\textsf{VALIDATE}}. If we need to change them, we click on \fbox{\textsf{Set mutation model}} which would bring the following screen :\\

\includegraphics[scale=0.35]{gui_pictures/Capture-DIYABC-48.png} \\

After validating twice, we get the last screen necessary to define this king of analysis :\\

\includegraphics[scale=0.35]{gui_pictures/Capture-DIYABC-49.png} \\
 
This screen is similar to that for parameter estimation (see section 3.5.1). After validating, we get back to the analysis panel with a third analysis defined :\\

\includegraphics[scale=0.35]{gui_pictures/Capture-DIYABC-50.png} \\

The analysis takes some time to run compared to the previous one, because it simulates hundreds datasets and on each one, a full ABC estimation is performed. Then after some time, the analysis is finished: \\

\includegraphics[scale=0.35]{gui_pictures/Capture-DIYABC-51.png} \\

To view results, we click on the  \fbox{\textsf{View results}} button. 

\newpage

The results are visible in a scrolling window :\\

\includegraphics[scale=0.32]{gui_pictures/Capture-DIYABC-52a.png} \\

Note that there are two values given for each statistics. The upper value is that of the statistics computed from the \emph{posterior} distribution of parameters, $i.e.$ \textbf{using the genetic information provided by data}. The lower value, noted between parentheses, is that of the statistics computed from the \emph{prior} distribution of parameters, $i.e.$ \textbf{NOT using the genetic information provided by data but only that contained in prior distributions}.  

\subsubsection{Model Checking}
We now define another type of analysis called \texttt{Model Checking} which is used to evaluate how well the scenario and priors of parameters fit the data summarized by summary statistics. This is the last option on the following screen :\\

\includegraphics[scale=0.32]{gui_pictures/Capture-DIYABC-53.png} \\

We call this analysis \texttt{mc1} and check the box to get a PCA performed. This PCA is computed in the same way compared to that of the first option (\texttt{Pre-evaluate scenario prior combinations}). However, new datasets simulated with parameters drawn from the posterior distributions of parameters are also represented on the different planes of the PCA (but not taken in the PCA  computation). 

We validate the above screen and get the usual next screen :\\

\includegraphics[scale=0.35]{gui_pictures/Capture-DIYABC-54.png} \\

that we just validate to get the last screen :\\

\includegraphics[scale=0.35]{gui_pictures/Capture-DIYABC-55.png} \\

In this screen which we have already seen, there is a new panel (bottom right) in which we can choose the number of datasets that we want to simulate from the posterior distributions of parameters. There is also a button  \fbox{\textsf{Redefine summary statistics of group:}} shown by the pointer. This button allows to change the set of summary statistics (for a given group of loci chosen through the drop list on the right). Clicking on this button opens up the usual following screen in which, by default, are checked the summary statistics in the reference table.\\

\includegraphics[scale=0.35]{gui_pictures/Capture-DIYABC-56.png} \\

We decide to use all one-sample and two-sample summary stats :\\

\includegraphics[scale=0.35]{gui_pictures/Capture-DIYABC-57.png} \\
 

Note that when the set of summary statistics is changed (as here), it is necessary to also simulate a large number of datasets using parameter priors to get corresponding values of the newly introduced summary statistics. We validate twice and launch the analysis. When it is finished, we click on the  \fbox{\textsf{View results}} button and get this screen:\\

\includegraphics[scale=0.33]{gui_pictures/Capture-DIYABC-58.png} \\

Clicking on the \fbox{\textsf{View numerical results}} leads to the following screen which provides, for each individual summary statistics, the value in the observed dataset as well as the proportion of data sets (simulated from the posterior) that have a value lower than the observed data set.\\
 
\includegraphics[scale=0.33]{gui_pictures/Capture-DIYABC-60.png} \\


Notice that in this computation, values that are in the interval $[s_{obs}-0.001,s_{obs}+0.001]$ are counted for one half those that are outside the interval. This explains why the fourth digit of the proportion can be 0 or 5 while having simulated 1000 data sets.\\
Here the conclusion is that the chosen model/posterior explain correctly the observed dataset (see \cite{C2010} for further illustrations). 

\subsubsection{Posterior probabilities of scenarios}
Consider a new example dataset in which three populations have been sampled. We want to decide which scenario is the best supported by data, a divergence scenario (scenario 1) or a split scenario (scenario 2) :\\

\includegraphics[scale=0.33]{gui_pictures/Capture-DIYABC-113.png} \\

Suppose that we have already built a reference table with these two scenarios. We define a new analysis that we call \texttt{comp1} :\\

\includegraphics[scale=0.33]{gui_pictures/Capture-DIYABC-105.png} \\
 
Note here that it is possible to replace original summary statistics (SS) by discriminant scores by checking the box \texttt{Linear discriminant analysis on SS}. This option is useful when there are numerous scenarios and many summary statistics \citep{EL2012}. However, in the present case, this is not necessary since the analysis with original summary statistics takes only a few seconds. After clicking on the \fbox{\textsf{VALIDATE}} button, we fill in the required fields, taking default values except for the number of local linear regression (on the second screen) that we set to 10:\\

\includegraphics[scale=0.3]{gui_pictures/Capture-DIYABC-109.png} \\

So that we get the following screen :\\

\includegraphics[scale=0.3]{gui_pictures/Capture-DIYABC-110.png} \\

After validating the screen above, we launch the analysis which lasts a few seconds and press on the \fbox{\textsf{View results}} button. The following screen appears:\\

 \includegraphics[scale=0.3]{gui_pictures/Capture-DIYABC-111.png} \\

Both analyses agree that scenario 1 is the best supported scenario in this comparison. If we click on the \fbox{\textsf{View numerical results}}, the program shows a subset of numerical values used in the previous screen (i.e. the probability values with their 95\% confidence intervals for the 10 subsets of closest simulated data). Note that the 95\% CIs can be used to ensure that probabilities are significantly different among scenarios.\\

 \includegraphics[scale=0.3]{gui_pictures/Capture-DIYABC-112.png} \\

\subsubsection{Confidence in scenario choice}

This last type of analysis is aimed at evaluating with which level of confidence we can trust the previous analysis. To do so, we simulate new datasets with each scenario, apply the same procedure for estimating their respective posterior probabilities and measure the proportion of times the right scenario has the highest posterior probability.\\
We then define a new analysis, \texttt{conf1} as below:\\   

\includegraphics[scale=0.33]{gui_pictures/Capture-DIYABC-115.png} \\

As for the previous analysis, it is possible to replace original summary statistics by discriminant scores. This is still more useful here since this type of analysis can last hours. However, with our current example, the analysis will not last more than a few minutes.\\
The next screen (below) needs some editing :\\

\includegraphics[scale=0.33]{gui_pictures/Capture-DIYABC-116.png} \\

We check the two scenario boxes on the right and decide to simulate datasets according to priors :\\

\includegraphics[scale=0.38]{gui_pictures/Capture-DIYABC-117.png} \\

After validating default values for historical and mutational parameters, we launch the analysis. When it is done, we click on the \fbox{\textsf{View results}} button and get the following screen:\\

\includegraphics[scale=0.38]{gui_pictures/Capture-DIYABC-119.png} \\

At the bottom, there is a summary of results, $i.e.$ the number of times each scenario has the highest posterior probability under each approach:\\

\includegraphics[scale=0.33]{gui_pictures/Capture-DIYABC-120.png} \\

We can deduce the type I error for scenario 1,  which is the probability with
which it is rejected although it is the true scenario : 20 using the direct
method (or 19 using the regression method) over 500, $i.e.$ 0.04 (0.038). To
have access to the type II error (probability of deciding for scenario 1 when it
is not the true scenario), we need to run the same analysis but simulating
according to all other scenarios (only scenario 2 in the present example) and
counting decisions in favor of scenario 1. Running the example analysis with
scenario 2 gives 77 (or 36) over 500 in favor of scenario 1. This gives an
estimate of type II error of 0.154 (0.072) for scenario 1.

\subsection{Simulating data sets}

The DIYABC program can also be used to simulate data sets, either microsatellite and/or DNA sequence data sets using our Genepop format, or SNP data sets using our specific format. This option is reachable through the main \textsf{File} menu as shown below :\\

\includegraphics[scale=0.33]{gui_pictures/Capture-DIYABC-62.png} \\

Clicking on e.g. the \textsf{Microsatellites and/or sequences (Genepop format)} opens up a dialog window in which one can choose the directory into which will be located the project and the future data files :\\

\includegraphics[scale=0.33]{gui_pictures/Capture-DIYABC-64.png} \\

Above, we decided to call \textsf{demo2} this new directory and to locate it in the \textsf{home/diyabc/demo} directory.

Clicking on \fbox{\textsf{OK}} leads to usual screen:\\

\includegraphics[scale=0.33]{gui_pictures/Capture-DIYABC-65.png} \\

We first inform the historical model clicking on the  \fbox{\textsf{Set}} button under \textsf{Historical model}. We edit the scenario box as below:\\

\includegraphics[scale=0.33]{gui_pictures/Capture-DIYABC-66.png} \\

We click on the  \fbox{\textsf{Set parameter values}} button. Arbitrary default values appear :\\

\includegraphics[scale=0.33]{gui_pictures/Capture-DIYABC-67.png} \\

We change these values according to our needs and we click the \fbox{\textsf{Set sample size}} button, getting this screen:\\

\includegraphics[scale=0.33]{gui_pictures/Capture-DIYABC-68.png} \\

We input the needed sample sizes as below :\\

\includegraphics[scale=0.33]{gui_pictures/Capture-DIYABC-69.png} \\

Clicking on the \fbox{\textsf{VALIDATE}} button, we get back to the previous screen showing that the Historical model is now completed:\\

\includegraphics[scale=0.33]{gui_pictures/Capture-DIYABC-70.png} \\


We have now to complete the Genetic data (click on the \fbox{\textsf{Set}} button under \textsf{Genetic data}). The following screen appears:\\

\includegraphics[scale=0.33]{gui_pictures/Capture-DIYABC-71.png} \\

We want a data set including three autosomal, two X-linked and one Y-linked diploid microsatellite loci and one mitochondrial sequence. We also need a sex ratio of one male for four females :\\

\includegraphics[scale=0.33]{gui_pictures/Capture-DIYABC-72.png} \\

We click on the \fbox{\textsf{OK}} button and get the following screen :\\

\includegraphics[scale=0.33]{gui_pictures/Capture-DIYABC-73.png} \\

Our mitochondrial DNA sequence is only 500 nucleotides long and there is a slight excess of A+T (60\%). We edit the corresponding cells : \\

\includegraphics[scale=0.33]{gui_pictures/Capture-DIYABC-74.png} \\

Since mutation models are different for microsatellites and DNA sequences, we define two groups by clicking twice on the \fbox{\textsf{Add group}} button :\\

\includegraphics[scale=0.33]{gui_pictures/Capture-DIYABC-75.png} \\

We select the 6 microsatellite loci by clicking on the first locus name cell and shift-clicking on the sixth locus name cell :\\

\includegraphics[scale=0.33]{gui_pictures/Capture-DIYABC-76.png} \\

The six locus names are transferred into group 1 by clicking on the \fbox{\textsf{$>>$}} button :\\

\includegraphics[scale=0.33]{gui_pictures/Capture-DIYABC-77.png} \\

Then the DNA sequence locus is selected :\\

\includegraphics[scale=0.33]{gui_pictures/Capture-DIYABC-78.png} \\

and transferred into group 2 in the same way :\\

\includegraphics[scale=0.33]{gui_pictures/Capture-DIYABC-79.png} \\

We need now to define the mutation model of each group. Let's click on the \fbox{\textsf{Set Mutation Model}} button of group 1:\\

\includegraphics[scale=0.33]{gui_pictures/Capture-DIYABC-80.png} \\

The usual default values appear. We want to exclude single nucleotide insertions/deletions (SNI mutations). So we set to 0 the Mean SNI rate and Minimum, Maximum and Shape of individual loci SNI rates :\\

\includegraphics[scale=0.33]{gui_pictures/Capture-DIYABC-81.png} \\

Once this done, we go back to the previous screen by clicking on the \fbox{\textsf{VALIDATE}} button. Then we set the mutation model of the mitochondrial DNA sequence. The default values are as follows :\\

\includegraphics[scale=0.33]{gui_pictures/Capture-DIYABC-82.png} \\

The default mean mutation rate is not suited to mitochondrial DNA which generally evolves at a faster rate than nuclear DNA \citep{HL2008}. So we set its value to $10^{-8}$. For all other parameters, we just keep the default values:\\

\includegraphics[scale=0.33]{gui_pictures/Capture-DIYABC-83.png} \\

After validating twice, we get back to the main screen :\\ 

\includegraphics[scale=0.33]{gui_pictures/Capture-DIYABC-84.png} \\

We require 10 simulated data sets :\\

\includegraphics[scale=0.33]{gui_pictures/Capture-DIYABC-85.png} \\

\newpage
We then click on the \fbox{\textsf{Run computation}} button. In a matter of seconds, the computation ends up:\\

\includegraphics[scale=0.33]{gui_pictures/Capture-DIYABC-86.png} \\

Using the file manager, we can check that ten new files (\textsf{demo2\_001.mss} to \textsf{demo2\_010.mss}) have been added to new directory :\\

\includegraphics[scale=0.33]{gui_pictures/Capture-DIYABC-87.png} \\

Opening e.g. the second one with a text editor, we can have a partial view of the simulated genotypes of the first population sample : \\

\includegraphics[scale=0.33]{gui_pictures/Capture-DIYABC-88.png} \\

We can check that the sex ratio is correct : the number of males is one fourth the number of females. The type of each locus given after the name is also correct. All microsatellite allelic values are odd, in agreement with the motif length (2) and the absence of single nucleotide insertion/deletion. More interestingly, it gives an example of how X- and Y-linked microsatellite loci must be written for each sex (here 15 females and 18 males) in our Genepop format. 


\subsection{The \textsf{Settings} option of the \textsf{File} menu}

Let us now detail what is under the \textsf{Settings} option of the \textsf{File} menu shown below :\\

\includegraphics[scale=0.33]{gui_pictures/Capture-DIYABC-91.png} \\

\newpage

Clicking on the \textsf{Settings} option opens up the following multitab window :\\

\subsubsection{Tab \textsf{``various''}}
The first tab \textsf{``various''} contains the following settings :

\begin{enumerate}
 \item \textsf{What's this} is a help functionnality that allows the user to obtain a help message when pointing towards a specific feature of the graphic interface such as a button or an edit field. This help functionnality can be activated by checking the corresponding box.
 \item Checking this box is mainly for debugging purpose or signalling a bug. 
 \item DIYABC is made of two programs : the graphic interface and a computation program. When the user clicks on buttons such as  \fbox{\textsf{Run computation}} or \fbox{\textsf{Launch}}, the graphic interface programs sends a command that launches the computation program. To issue this command, the graphic interface needs to know where the computation program executable is located. There is a default location which depends on the operating system. Clicking on the box \textsf{Use default executable check} will direct the graphic interface to use the executable located in this default directory.
 \item You can also choose another location (e.g. if you want to use a distinct version of the executable) by clicking on the \fbox{\textsf{browse}} button.\\
\includegraphics[scale=0.33]{gui_pictures/Capture-DIYABC-94.png} \\
 \item The next setting \textsf{Particle loop size} defines the number of data sets ($n$) that are simulated in a single block when building the reference table. The computation program proceeds as follows : it first simulate and compute summary statistics of $n$ data sets. When this is done, it writes the results to the reference table file. The reason of doing like this is that computation can be multithreaded but not the file writing.
 \item The graphic interface can detect the number of cores of the computer processor. By default, it sets the number of threads of the computation program to this core number. However, if the user wants to keep some cores for other purposes, the number of threads can be reduced by on the corresponding button (drop down menu shown below).\\
\includegraphics[scale=0.33]{gui_pictures/Capture-DIYABC-95.png} \\
 \item The next setting (\texttt{Maximum log level}) is for debugging pupose and/or signalling a bug (from 1=low information level to 4=high information level).
 \item The graphic interface memorizes recently opened projects. The edit field is used to set the maximum of memorized recent projects.
 \item The last setting concerns the format of graphic files output by different analyses. Choice is shown below:\\
\includegraphics[scale=0.33]{gui_pictures/Capture-DIYABC-97.png} \\
   
\end{enumerate}

Eventually, if changes have been made, they can be either saved or cancelled (two bottom buttons).

\subsubsection{Tab \textsf{``appearance''}}

Clicking on this tab results in the following screen :\\

\includegraphics[scale=0.33]{gui_pictures/Capture-DIYABC-98.png} \\

The window style can be chosen among the following (click on the upper drop down menu) :\\

\includegraphics[scale=0.33]{gui_pictures/Capture-DIYABC-99.png} \\

Likewise, the background color can be chosen among the following colours :\\

\includegraphics[scale=0.33]{gui_pictures/Capture-DIYABC-100.png} \\

Eventually, one can change the font of texts appearing in the different windows by clicking on the corresponding button. A usual font menu then appears allowing the desired change :\\

\includegraphics[scale=0.33]{gui_pictures/Capture-DIYABC-101.png} \\

\subsubsection{Tab \textsf{``cluster''}}\label{clustergui}
The third tab  is related to the use of a computer cluster to perform computations of the reference table. If you have access to a computer cluster and if the computer cluster runs a scheduler queuing system, then you can use it to generate the reference table (detailed in section 5). You will need to :
\begin{enumerate}
 \item check the box \textsf{Use a cluster (...)}
 \item indicate the number of data sets produced by each single job of the queue
 \item indicate the number of cores used by each single job of the queue
 \item indicate the number of concurrent jobs running at the same time on the cluster
 \item indicate the seed to start the generation of RNG files. Leave it blank or write \texttt{None} to use a random seed
\end{enumerate}

The next two text frames deals with first and last parts of the main script running on the cluster. This bash script will submit jobs to the scheduler queuing system :
\begin{enumerate}
 \item the first part is not editable as it include the variables used by DIYABC GUI frontend
 \item the last part deals with the jobs submission. You can edit it to match the specification of your scheduler : submission syntax, queue, ... By default, the code targets a Grid Engine cluster. Please ask for help to your cluster system administration.
\end{enumerate}


Clicking on the \fbox{\textsf{Run computation}} button generates a bundle ($i.e.$ a set of zipped files) including all you need to generate your reference table. You need to transfer the bundle in your cluster account and run it. Once all conputations are done and all the reference table parts are merged in one, you have to transfert the merged reference table back to your DIYABC project on your own computer to proceed subsequent analyses with the DIYABC GUI. All above steps are further detailed in section 5..

\includegraphics[scale=0.33]{gui_pictures/Capture-DIYABC-cluster.png} \\

\subsubsection{Tabs \textsf{``MM Microsats''} and \textsf{``MM Sequences''}}

These two tabs are used to modify the default values of mutation parameters (MM means Mutation Model), for microsatellites and DNA sequences respectively. As an example, here is the screen corresponding to the tab  \textsf{``MM Sequences''} : \\

\includegraphics[scale=0.33]{gui_pictures/Capture-DIYABC-102.png} \\

The initial default values have been obtained through literature compilation and are valid for a large number of species. However, some species may have values that differ substantially from most species. For instance, the mutation rate of some \emph{Drosophila} species are much lower than the values encountered in many other species \citep{SMA1997,VPAD2000} and is outside the range indicated in the initial default values. 



 

\clearpage
 \section{Implementation details}
\subsection{Software design}
$DIYABC$ v2 has been designed in a very different way compared to version 1. Version 1 was a single executable file were the GUI \footnote{Graphic User Interface} and computation codes were highly intricated and both written in the same language (\emph{Delphi}). In version 2, the GUI and the computation codes have been completely separated. Actually, the GUI is a script written in \emph{python} and all computations are included in a program written in \emph{C++}. In opposition to \emph{Delphi} which is restricted to a single OS (\texttt{Windows}), \textit{python} and \textit{C++} can be used with the main three OS (\texttt{Linux}, \texttt{Mac} and \texttt{Windows}), allowing version 2  to be operated under all three OS.\\
The GUI uses the \textit{Qt} graphic library. The computation code is linked to the \textit{openmp} library allowing a better use of multicore/multiprocessor computers.\\
The GUI can launch the computation program with the right parameters and keeps track of the progress of the latter through small log files. The GUI can launch as many computation programs as there are open projects, but no more than one computation program per project. A \textit{lock} file located in the project directory is created when the computation program is launched by the GUI and removed when the computation program has normally terminated. When the computation program has exited anormaly, the GUI issues an error message trying to explain where the programm failed.    
\subsection{Files}
The program uses and produces various files which we will describe now.
\subsubsection{data files}
Data files are text files that contain information about the samples : number and names of microsatellite markers, multilocus genotypes of individuals. The basic  format is that of the Genepop software \citep{RR1995} and data files produced by DIYABC are under this format.  \textbf{Microsatellite genotypes must be noted with 3 (haploid) or 6 (diploid) digits, these three digit numbers being the length in nucleotides of the corresponding PCR products}. In addition, we have added some features to this basic format in order to use sequence data. All these additions are explained in section 4.4.\\ 
Any extension is accepted for datafile names, including no extension at all. If the data file is simulated with $DIYABC$, the extension is \texttt{mss} for microsatellite/DNA sequence data and \texttt{snp} for SNP data. The next page shows examples of data sets saved.

\subsubsection{reference table files} 
Reference table files are binary files which include two successive parts :
\begin{itemize}
 \item The first part is a header which contains information necessary to read the second part, such as the number of scenarios, or the number of parameters of each scenario.
 \item The second part contains simulated data set records, each record containing the scenario number, the parameter and summary statistics values.
\end{itemize}
 
 Each time a reference table is created or increased (each time the \fbox{\textsf{Run computation}} button is pressed), a text file is created in the project directory with the name \texttt{first\_records\_of\_the\_reference\_table\_X.txt} in which \texttt{X} is an integer number starting at 0 and increasing each time the \fbox{\textsf{Run computation}} button is pressed. This file provides a text version of the first $n$ newly created records of the reference table ($n$ being equal to the \emph{Particle loop size}, see section 3.7.3).\\

\subsubsection{output files}
As already seen, DIYABC achieves different analyses : comparison of scenarios, estimation of posterior distribution of parameters, model checking,  computation of bias and mean square errors and evaluation of confidence in scenario choice. Each analysis has its own output which can be printed and saved. Graphs are saved under the chosen format and non-graphic output are saved in text files.\\

We now describe all the files produced by each type of analysis. These files are located in directories (one directory per analysis) gathered in the \texttt{analysis} subdirectory of the project directory. Below is an example of the \texttt{TOYTEST2\_2012\_9\_26-1} project directory substructure:\\

\includegraphics[scale=0.5]{gui_pictures/Capture-DIYABC-103.png}\\

Note that each directory name starts with the name of analysis followed by the type of analysis, $e.g.$ \texttt{bias} for a bias/precision analysis or \texttt{comparison} for a comparison of scenarios. In addition, when a picture has been saved, the corresponding file is located under a subdirectory named \texttt{pictures} ($e.g.$ at the bottom of the figure above).
\begin{description}
 \item [Pre-evaluate scenario prior combinations :] This analysis can produce two output files named \texttt{ACP.txt} and \texttt{locate.txt}. The former is the output of the Principal Component Analysis and the latter that of the analysis giving the proportion of simulated data sets which have a value below the observed value for every summary statistics. This latter file is exactly what appears in the GUI. The structure of the  \texttt{ACP.txt} file is the following. The first line indicates the number of points of the PCA, the number of PCA components (axes) and the inertia of each component, all values are separated by a single space. The second line provides the components of the observed data. It starts with a zero which corresponds to the scenario number in the following lines. Each subsequent line provides the components of data simulated according to a given scenario which number is at the beginning of the line. If one or more PCA figures have been saved, the corresponding files are saved in the \texttt{pictures} subdirectory. They are named as \texttt{refTable\_PCA\_X\_Y\_N.pdf}, with \texttt{X} and  \texttt{Y} giving the axis numbers and \texttt{N} being the number of represented points. 
 \item [Compute posterior probabilities of scenarios :] This analysis produces three output text files : \texttt{compdirect.txt}, \texttt{complogreg.txt} and \texttt{compdirlog.txt}. The latter is directly visualized in the GUI when clicking the \fbox{\textsf{view numerical results}} button. The first two files are used by the GUI to elaborate the two graphics (Direct approach and Logistic regression). Again, if graphics have been saved, the corresponding file(s) is(are) in the \texttt{pictures} subdirectory of the analysis directory.  
 \item [Evaluate confidence in scenario choice :] This analysis produces a single output file, \texttt{confidence.txt}, the content of which is visualized in the GUI.
 \item [Estimate posterior distributions of parameter :] Nine files are written as output of this type of analysis :
  \begin{itemize}
   \item three files \texttt{mmmq\_original.txt}, \texttt{mmmq\_composite.txt} and \texttt{mmmq\_scaled.txt} contain the statistics (mean, median, mode and quantiles) for the original, composite and scaled parameters, respectively. They are visualized in the GUI when clicking the \fbox{\textsf{view numerical results}} button. 
   \item three files \texttt{paramstatdens\_original.txt}, \texttt{paramstatdens\_composite.txt} and \texttt{paramstatdens\_scaled.txt} are used by the GUI to produce the graphics showing prior/posterior distribution.
   \item three files \texttt{phistar\_original.txt}, \texttt{phistar\_composite.txt} and \texttt{phistar\_scaled.txt} contains the $\phi^*$ values of the original, composite and scaled parameters, respectively.
  \end{itemize}
As already mentionned, saved graphics are located in a \texttt{pictures} subdirectory.
 \item [Compute bias and precision of parameter estimations :] Three files \texttt{bias\_original.txt}, \texttt{bias\_composite.txt} and \texttt{bias\_scaled.txt} are produced by this type of analysis. All three files are visualized in the GUI. 
 \item [Perform model-checking] The output files of this type of analysis are the same as those of the \textit{Pre-evaluate scenario prior combinations} analysis (see above). The only difference is in the names of the two text files which start with \texttt{mc} for \texttt{model checking}. 
\end{description}
~\\

In addition, the GUI program writes several files in the project directory :
\begin{description}
 \item [command.txt :] this text file contains the history of commands issued by the GUI to be achieved by the computation program.
 \item [conf.analysis :] this text file contains information about analyses.
 \item [conf.gen.tmp :] this text file contains information about the loci, the genetic parameters and the summary statistics.
 \item [conf.hist.tmp :] this text file contains information about the scenario and the historical parameters.
 \item [conf.th.tmp :] This text file contains the title line of the reference table.
 \item [conf.tmp :] This text file contains the name of the dataset and the number of parameters and summary statistics.
 \item [header.txt :] This text file is a concatenation of the previous four files and is red by the computation program.
 \item [xxx.diyabcproject :] This text file contains the path to the \texttt{xxx} project.
 \item [RNG\_state\_0000.bin :] This binary file contains the current state of the random generator.
 \item [init\_rng.out :] This text file contains information about the initialization of the random generator.
\end{description}
~\\

The computation program writes the following files in the project directory :
\begin{description}
 \item [reftable.log :] This text file is produced when a reftable is increased. It provides the GUI with information about the progress of computations : achieved number of records, time left.
 \item [statobs.txt :] This text file is written every time an analysis is performed. It contains the values of summary statistrics for the observed data set.
\end{description}

The following files are output by the computation program everytime it has been launched by a specific command of the GUI (their use is only for debugging purposes and they are all in the project directory) :
\begin{description}
 \item [general.out :] when computing a reftable.
 \item [pre-ev.out :] when performing a \textit{Pre-evaluate scenario prior combinations} analysis.
 \item [compare.out :] when performing a \textit{Compare scenarios} analysis.
 \item [confidence.out :] when performing a \textit{Confidence in scenario choice} analysis.
 \item [estimate.out :] when performing a \textit{ABC parameter estimation} analysis.
 \item [bias.out :] when performing a \textit{bias-precision} analysis.
 \item [modelChecking.out :] when performing a \textit{model checking} analysis.
\end{description}

When performing a \textit{Bias-precision} or a \textit{Confidence in scenario choice} analysis, the computation program simulates what we call \textit{pseudo-observed datasets}. The parameter and summary statistics values of these pseudo-observed datasets are written in a text file named \textbf{pseudo-observed\_datasets\_xxx.txt} in which \textbf{xxx} is the name given to the analysis.


\subsection{Missing data}
Missing or undetermined genotypes should be coded as \texttt{000} (haploid microsatellites, \texttt{000000}  (diploid microsatellites), \texttt{$<[]>$} (haploid sequences) or \texttt{$<[][]>$} (diploid sequences) and \texttt{9} (SNP) in the data file. \\
Missing data are taken into account in the following way. For each appearance of a missing genotype in the observed data set, the programs records the individual and the locus. When simulating data sets, the program replaces the simulated genotype (obtained through the coalescence process algorithm) by the missing data code at all corresponding locations. All summary statistics are thus computed with the same missing data as for the observed data set. 

\subsection{Data files}
There are two different incompatible formats for data files, one for SNP loci and the other for microsatellite/DNA sequence data.\\ For the latter, the format already presented in version 1 of DIYABC is an extended Genepop format. The additional features are :
\begin{enumerate}
\item In the title line appears the sex ratio noted between \textsf{$<$} and \textsf{$>$} under the form \textsf{$<NM=rNF>$}, in which $r$ is the ratio of the number of females per male ($e.g.$ \textsf{$<NM=2.5NF>$} means that the number of males is 2.5 times the number of females). Since the title is generally only copied, this addition should not interfere with other programs using  Genepop datafiles. Also if there is no such sex ratio addition, DIYABC will consider by default that NM=NF.
\item After the locus name, there is an indication for the category of the locus which is $<A>$ for autosomal diploid loci, $<H>$ for autosomal haploid loci, $<X>$ for X-linked (or haplo-diploid) loci, $<Y>$ for Y-linked loci and $<M>$ for mitochondrial loci. If no category is noted, DIYABC will consider the locus as autosomal diploid or autosomal haploid depending on the corresponding genotype of the first typed individual.
\item Genotypes of microsatellite loci are noted with six digit numbers (e.g. 190188) if diploid and by three digit numbers (e.g. 190) if haploid.
\item Sequence locus are noted between  \textsf{$<$} and \textsf{$>$} . In addition each sequence allele is noted between brackets. For instance, a haploid sequence locus  will be noted $<[$GTCTA$]>$ and a diploid sequence locus $<[$GTCTA$][$GTCTT$]>$.
\item Missing microsatellite genotypes are noted \textsf{000} if haploid or \textsf{000000} if diploid.
\item Missing sequence genotypes are noted $<[\ ]>$ if haploid or $<[\ ][\ ]>$ if diploid.
\end{enumerate}

For SNP data, the format includes:
\begin{itemize}
 \item a first line providing the sex-ratio as above
 \item a second line starting with the three keywords \texttt{IND  SEX  POP}, separated by at least one space, followed by as many letters as SNP loci, the letter giving the location of the locus as above ($<A>$ for autosomal diploid loci, $<H>$ for autosomal haploid loci, $<X>$ for X-linked (or haplo-diploid) loci, $<Y>$ for Y-linked loci and $<M>$ for mitochondrial loci). Letters are separated by a single space.
 \item as many lines as there are genotyped individuals, with the code-name of the individual, a letter ($M$ or $F$) indicating its sex, a code-name for its population and the values (0, 1 or 2) of the number of the reference allele at each SNP locus. 
\end{itemize}


 
Below are three examples of data sets simulated by DIYABC.\\ In the first example, this data set includes two population samples, each of 12 diploid individuals (8 females and 4 males in the first sample and 5 females and 7 males in the second sample). As deduced from the letter between $<$ and $>$ on the locus name lines (see page 25), these individuals have been genotyped at 3 microsatellite loci (1 autosomal $<A>$, 1 X-linked $<X>$ and 1 Y-linked $<Y>$) and 3 DNA sequence loci (1 autosomal. 1 X-linked and 1 mitochondrial $<M>$). The species sex-ratio, given in the title line, is of three males for one female ($<NM=3NF>$) or in other words, the number of males equals three times the number of females. 
\begin{figure}[h]
\includegraphics[scale=0.7]{gui_pictures/screenga001.png}
\end{figure}

In the second example, the species is haploid. Individuals have been genotyped at three autosomal microsatellite loci and one mitochondrial DNA sequence locus. The species being haploid (deduced from the presence of autosomal haploid loci), no indication of the sex-ratio appears in the title line.

\begin{figure}[h]
\includegraphics[scale=0.6]{gui_pictures/screenga002.png}
\end{figure}

In the third example, the species is diploid and has been genotyped at a large number of SNP autosomal loci. The first line provides the species sex-ratio. The second line indicates what's in the different columns : individual name in column 1, individual sex in column 2, population name in column 3 and one column per SNP locus. Columns are sparated by one or more spaces. SNP are coded 0, 1 or 2 according to the number of reference alleles. Only the top left part of the data file is represented below :

\begin{figure}[h]
\includegraphics[scale=0.7]{gui_pictures/screenga003.png}
\end{figure}



\clearpage
\section{Cluster version}
The process of simulating data sets is generally a time consuming part of the ABC approach. Typically, one to several millions  data sets are needed to build up a reference table and this process can last several hours to several days. For those  that can have access to a computer grid cluster, two additional programs are available. They both run under Linux. Source files are written in Delphi pascal and can be compiled with Free Pascal Compiler (http://www.freepascal.org/download.var) using the command line :\\
\texttt{>fpc -Sd -m Delphi <filename>}
  \begin{description}
  \item[\texttt{diyabc\_sim}] :\\
  This program simulates a given number of data sets according to the information given in a \texttt{reftableHeader} file. The simulated data sets are output in a \texttt{reftable} file. This program requires 3 parameters : the name of the reftableHeader file, a string that will be affixed to the name of the reftable file (to distinguish it from other reftable files) and the number of desired simulated data sets. This programs has two input files : the reftableHeader file and the data file. The data file must match the one written at the beginning of the reftableHeader file. The data file is just needed to evaluate sample sizes and number of loci as well as determining missing data. As an example, supposing a reftableHeader file named \texttt{mydataanalysis.reftableHeader}, a command such as :\\
  \texttt{>diyabc\_sim mydataanalysis 01 5000}\\
  will produce a reference table file named \texttt{mydataanalysis01.reftable}  containing 5,000 simulated data sets according to the observed data set and analysis summarized in   \texttt{mydataanalysis.reftableHeader}. 
  
  \item[\texttt{diyabc\_cat}]:\\
  This program pools all reftable files of a directory into a single reftable file. This programs requires 2 parameters : the name of the reftableHeader file used to generate all the reftable files to be pooled and the name of the output reftable. \texttt{diyabc\_cat} reads the reftableHeader file (which has to be in the same directory), scans all available reftable files present in the directory and select all those with a header that matches the reftableHeader file.  The selected files are then copied to a single reftable file. Following with the example above, a command such as :\\
  \texttt{>diyabc\_cat mydataanalysis mda}\\
  will pool all reftable files with header matching \texttt{mydataanalysis.reftableHeader} in an output file named \texttt{mda.reftable}. 
  \end{description}
  
  The cluster is just used to produce a reftable file ready to be analyzed by DIYABC. All treatments are thus performed (under Windows) with the DIYABC programs. A typical session will include :\\
  \begin{enumerate}
  \item Using DIYABC, input data file name, provide scenario(s), provide priors for historico-demographic and mutationnal parameters, provide motif lengths and ranges, select summary statistics, simulate a few datasets (in one wants to check that the ouput is OK, but this last step is not compulsary). 
  \item Transfer the reftableHeader file and the data file on the cluster and using \texttt{diyabc\_sim} and \texttt{diyabc\_cat} as explained above, create a large reftable file.
  \item Transfer the reftable file back to the Windows directory where the analysis began and rerun DIYABC to perform comparison of scenarios, estimation of parameters or computation of bias/precision as needed.  
  \end{enumerate}
  For those who have access to a cluster with Sun's Grid Engine software, they can use the following script to create multiple reference tables (see next page). However, this script will not exempt the user to  manually run \texttt{diyabc\_cat} to concatenate all reference tables in a single one. The script is run with the following four parameters :\\
  
  \texttt{>ScriptABC file.dat file.reftableHeader n\_jobs n\_iter\_per\_job}
\clearpage
\begin{figure}
\caption{Bash script that can be run with Sun Grid Engine software}
\includegraphics{gui_pictures/script_SGE.pdf}
\end{figure}

\clearpage
\subsection{A note about the random number generator used in DIYABC}

The random number generator used in DIYABC is of the \emph{Multiply With Carry} type invented by G. Marsaglia (see http://en.wikipedia.org/wiki/Multiply-with-carry for detailed explanations and references; see also http://www.rlmueller.net/MWC32.htm for a more condensed description).\\
The major interests of this random generator are :
\begin{itemize}
\item  it is very simple to program 
\item it is also very fast
\item there is a large choice of random number sequences among which to choose
\item the period of each sequence is a simple function of two parameters that define the sequence
\end{itemize} 

Four parameters define a unique sequence of pseudo random numbers. They are called the \emph{modulo} ($M$), the \emph{multiply}($m$), the \emph{initial x} ($x_0$) and the \emph{initial carry} ($c_0$). Only the last two parameters change when drawing a number, according to the following equations : \\

$x_i$ =$ [(m * x_{i-1} + c_{i-1}] $ \textsf{modulo} $M$\\  

$c_i$ = $[(m * x_{i-1} + c_{i-1}]$  \textsf{div}\footnote{\textsf{div} is the integer division, i.e. the result ($c_i$) is the integer part of the ratio.} $M$\\

In DIYABC, we chose a 24-bit generator, meaning that the \emph{modulo} parameter is fixed to $M$=$2^{24}$=16777216.\\
The \emph{multiply} parameter, which defines the sequence, is chosen among numbers that garanties a period larger than $10^{15}$. \\
The last two parameters, \emph{initial x and carry} are drawn at random, using the Delphi \textsf{random} function. \\

The multithreaded sections of the codes which require random draws are such that each thread uses its own random generator (defined by its own \emph{multiply} parameter). This solution spares writing conflicts from different threads since each call to a random generator involves writing new values of its $x$ and $carry$ parameters. 
The current version of DIYABC contains 10,000 possible values for this \emph{multiply} parameter (over the 43,701 possible). Since the number of threads is set equal to the number of cores, that should be enough for the moment.\\

\textbf{Important note for users of the cluster version}\\
Concerning the program \texttt{diyabc\_sim}, the Delphi \textsf{random} function has been replaced by reading bytes from the \texttt{/dev/random} file. However, when several jobs are sent to the same computer, the latter being multicore, there is a risk that these jobs read the same bytes and hence that they draw the same sequence of random numbers. To avoid this situation, the program uses the second parameter which is different among jobs to define the \emph{multiply} parameter of the sequence. Consequently, if the cluster contains multicore computers, it is strongly advised to use successive integers as second parameters for  \texttt{diyabc\_sim}. Furthermore, \texttt{diyabc\_sim} can use multiple threads. If one knows the number of cores of the computer on which the job will run, one can  add a fourth parameter equal to this core number. In that case, all cores will use the same \emph{multiply} parameter (defined by the second parameter if the latter is an integer), but different values of $x_0$ and $c_0$, because all the latter are now red sequentially from \texttt{/dev/random}.
 


  
\begin{thebibliography}{a}
\bibitem[Beaumont \emph{et al.}, 2002]{B2002} Beaumont, M. A., W. Zhang and D. J. Balding, 2002. Approximate Bayesian Computation in Population Genetics. \emph{Genetics} \textbf{162}, 2025-2035.
\bibitem[Beaumont, 2008]{B2008}Beaumont, M.A., 2008. Joint determination of topology, divergence time, and immigration in population trees. In Simulation, Genetics, and Human Prehistory, eds. S. Matsumura, P. Forster,  C. Renfrew. McDonald Institute Press, University of Cambridge (\emph{in press}).
\bibitem[Begg and Gray, 1984]{BG1984} Begg, C.B. and R. Gray, 1984. Calculation of polychotomous logistic regression parameters using individualized regressions. \emph{Biometrika},
\textbf{71}, 11-18.
\bibitem[Belkhir \emph{et al.}, 1996-2004]{BB1996} Belkhir K., Borsa P., Chikhi L., Raufaste N. and F. Bonhomme, 1996-2004 GENETIX 4.05, logiciel sous Windows TM pour la g�n�tique des populations. Laboratoire G�nome, Populations, Interactions, CNRS UMR 5171, Universit� de Montpellier II, Montpellier (France).
\bibitem[Bertorelle and Excoffier, 1998]{BE1998} Bertorelle, G. and L. Excoffier, 1998. Inferring admixture proportion from molecular data. \emph{Mol. Biol. Evol.} \textbf{15}, 1298-1311.
\bibitem[Choisy  \emph{et al.}, 2004]{CF2004} Choisy, M., P. Franck and J.M. Cornuet, 2004. Estimating admixture proportions with microsatellites : comparison of methods based on simulated data. \emph{Mol. Ecol.} \textbf{13}, 955-968.
\bibitem[Chakraborty and Jin, 1993]{CJ1993}Chakraborty R and L Jin, 1993. A unified approach to study hypervariable polymorphisms: statistical considerations of determining relatedness and population distances. EXS. \emph{67}, 153�175.
\bibitem[Cornuet \emph{et al.}, 2006]{C2006} Cornuet, J. M.,
M. A. Beaumont, A. Estoup and M. Solignac, 2006. Inference on
microsatellite mutation processes in the invasive mite, \emph{Varroa
destructor}, using reversible jump Markov chain Monte Carlo.
\emph{Theoret. Pop. Biol.} \textbf{69}, 129-144.
\bibitem[Cornuet \emph{et al.}, 2010]{C2010}Cornuet J.M., V. Ravign\'e and A. Estoup, 2010. Inference on population history and model checking using DNA sequence and microsatellite data with the sofware DIYABC (v1.0). \emph{submitted}. 
\bibitem[Cornuet \emph{et al.}, 2008]{C2008}Cornuet J.M., F. Santos, M.A. Beaumont, C.P. Robert, J.M. Marin, D.J. Balding, T. Guillemaud and A. Estoup, 2008. Infering population history with DIYABC: a user-friendly approach to Approximate Bayesian Computations. \emph{Bioinformatics}, \textbf{24} (23), 2713-2719.

\bibitem[Estoup \emph{et al.}, 1993]{E1993} Estoup, A., M. Solignac, M. Harry and J.M. Cornuet, 1993. Characterization of $(GT)_n$ and $(CT)_n$ microsatellites in two insect species: \emph{Apis mellifera} and \emph{Bombus terrestris}. \emph{Nucl. Ac. Res.}, \textbf{21}, 1427-1431.
\bibitem[Estoup \emph{et al.}, 2001]{E2001} Estoup, A., I. J. Wilson, C. Sullivan, J. M. Cornuet and C. Moritz, 2001 Inferring population history from microsatellite et enzyme data in serially introduced cane toads, \emph{Bufo marinus}. \emph{Genetics}, \textbf{159}, 1671-1687.
\bibitem[Estoup \emph{et al.}, 2002]{E2002}Estoup, A., P. Jarne and J.M. Cornuet, 2002.
 Homoplasy and mutation model at microsatellite loci and their consequences for population
 genetics analysis. \emph{Mol. Ecol.}, \textbf{11}, 1591-1604.
\bibitem[Estoup and Clegg, 2003]{EC2003} Estoup, A. and S. M. Clegg, 2003. Bayesian inferences on the recent islet colonization history by the bird Zosterops lateralis lateralis. \emph{Mol. Ecol.} \textbf{12}: 657-674.
\bibitem[Estoup \emph{et al.}, 2004]{EB2004}Estoup, A., M.A. Beaumont, F. Sennedot, C. Moritz and J.M. Cornuet, 2004. Genetic analysis of complex demographic scenarios : spatially expanding populations of the cane toad, \emph{Bufo marinus}. \emph{Evolution},\textbf{58}, 2021-2036.
\bibitem[Estoup \emph{et al.}, 2012]{EL2012}Estoup, A., E. Lombaert, J.M. Marin, T. Guillemaud, P. Pudlo, C.P. Robert and J.M. Cornuet, 2012. Estimation of demo-genetic model probabilities with Approximate Bayesian Computation using linear discriminant analysis on summary statistics. \emph{Molecular Ecology Resources}, \textbf{12}, 846-855.

\bibitem[Excoffier \emph{et al.}, 2005]{Ex2005} Excoffier, L., A. Estoup and J.M. Cornuet, 2005.
 Bayesian analysis of an admixture model with mutations and arbitrarily linked markers. \emph{Genetics} \textbf{169}, 1727-1738.
\bibitem[Fagundes \emph{et al.}, 2007]{FR2007} \textsc{Fagundes, N.J.R., N. Ray, M.A. Beaumont, S. Neuenschwander, F. Salzano, S.L. Bonatto and L. Excoffier}, 2007. Statistical evaluation of alternative models of human evolution. \emph{Proc. Natl. Acad. Sc.}, \textbf{104} : 17614-17619.
\bibitem[Fu and Chakraborty, 1998]{F1998} Fu, Y.X. and
Chakraborty, R., 1998. Simultaneous estimation of all the parameters
of a stepwise mutation model. \emph{Genetics}, \textbf{150},
487-497.
\bibitem[Garza and Williamson, 2001]{GW2001} Garza JC and E Williamson, 2001. Detection of reduction in population size using data from microsatellite DNA. \emph{Mol. Ecol.}   \textbf{10},305-318.
\bibitem[Gelman \emph{et al.}, 1995]{GCSR1995} Gelman, A., J.B. Carlin, H.S. Stern and D.B. Rubin, 1995. \emph{Bayesian Data Analysis}. Chapman et Hall, London, 526p. 
\bibitem[Golstein \emph{et al.}, 1995]{GL1995} Goldstein DB, Linares AR, Cavalli-Sforza LL, and Feldman MW, 1995. An evaluation of genetic distances for use with microsatellite loci. \emph{Genetics} \textbf{139}, 463-471.
 \bibitem[Goudet, 1995]{G1995} Goudet, J. ,1995. FSTAT (Version 1.2): A computer program to calculate F- statistics. \emph{J. Hered.} \textbf{86},  485-486.
 \bibitem[Griffiths and Tavar\'e, 1994]{GT1994} Griffiths, R.C. and S. Tavar\'e, 1994. Simulating probability distributions in the coalescent. \emph{Theor. Pop. Biol.} \textbf{46}, 131-159.
\bibitem[Guillemaud  \emph{et al.}, 2010]{G2010}Guillemaud T., M.A. Beaumont, M. Ciosi, J.M. Cornuet and A. Estoup, 2010. Inferring introduction routes of invasive species using approximate Bayesian computation on microsatellite data. \emph{Heredity}, \textbf{104}, 88-99.
\bibitem[Haag-Liautard \emph{et al.}, 2008]{HL2008} Haag-Liautard C., N. Coffey, D. Houle, M. Lynch, B. Charlesworth and P.D. Keightley, 2008. Direct estimation of the mitochondrial DNA mutation rate in Drosophila melanogaster. \emph{Plos Biol}, \textbf{6}, e204.
 \bibitem[Hamilton \emph{et al.}, 2005]{HS2005} Hamilton, G., M. Stoneking and L. Excoffier, 2005. Molecular analysis reveals tighter social regulation of immigration in patrilocal populations than in matrilocal populations. \emph{Proc. Natl. Acad. Sci. USA}, \textbf{102}, 7476-7480.
\bibitem[Hasegawa  \emph{et al.}, 1985]{HKY1985} Hasegawa, M., Kishino, H and Yano, T., 1985. Dating the human-ape splitting by a molecular clock of mitochondrial DNA. \emph{Journal of Molecular Evolution} 22:160-174.
\bibitem[Hudson \emph{et al.}, 1992]{H1992} Hudson,R. R., M. Slatkin and W.P. Maddison, 1992. Estimation of levels of gene flow fom DNA sequence data. \emph{Genetics}, 132, 583-589.
\bibitem[Ihaka and Gentleman, 1996]{IG1996}Ihaka R. and R. Gentleman, 1996. $R$: a language for data analysis and graphics. \emph{J.  Comput. Graph. Stat.}, \textbf{5}, 299-314
\bibitem[Ingvarsson, 2008]{I2008} Ingvarsson P.K., 2008. Multilocus patterns of nucleotide polymorphism and the demographic history of \emph{Populus tremula. Genetics}, 180: 329-340.
\bibitem[Jukes and Cantor, 1969]{JK1969}Jukes, TH and Cantor, CR., 1969. Evolution of protein molecules. Pp. 21-123 in H. N. Munro, ed. \emph{Mammalian protein metabolism}. Academic Press, New York.
\bibitem[Kimura, 1980]{K1980}Kimura, M., 1980. A simple method for estimating evolutionary rate of base substitution through comparative studies of nucleotide sequences. \emph{Journal of Molecular Evolution} 16:111-120.
\bibitem[Lombaert \emph{et al.}, 2010]{L2010}Lombaert E., T. Guillemaud, J.M. Cornuet, T. Malausa, B. Facon and A. Estoup, 2010.  Bridgehead effect in the worldwide invasion of the biocontrol harlequin ladybird. \emph{PLoS ONE}, http://dx.plos.org/10.1371/journal.pone.0009743.
\bibitem[Matsumoto and Nishimura, 2000]{DCMT} Matsumoto M and T Nishimura, 2000. Dynamic Creation of Pseudorandom Number Generators. \emph{Monte Carlo and Quasi-Monte Carlo Methods 1998}, Springer, pp 56--69.
\bibitem[Miller \emph{et al.}, 2005]{ME2005} Miller N, A. Estoup, S. Toepfer, D Bourguet, L. Lapchin, S. Derridj, K.S. Kim, P Reynaud, F. Furlan and T. Guillemaud,  2005. Multiple Transatlantic Introductions of the Western Corn Rootworm. \emph{Science}, \textbf{310}, p. 992
\bibitem[Nei, 1972]{N1972} Nei M., 1972. Genetic distance between populations. \emph{Am. Nat.} 106:283-292
\bibitem[Nei, 1987]{N1987} Nei M., 1987. \emph{Molecular Evolutionary Genetics}. Columbia University Press, New York, 512 pp.
\bibitem[Ohta and Kimura, 1973]{O1973} Ohta, T. and Kimura, M.,
1973. A model of mutation appropriate to estimate the number of
electrophoretically detectable alleles in a finite population.
\bibitem[Pascual \emph{et al.}, 2007]{PC2007}Pascual, M., M.P. Chapuis, F. Mestres, J. Balany\'a, R.B. Huey, G.W. Gilchrist, L. Serra and A. Estoup, 2007. Introduction history of \emph{Drosophila subobscura} in the New World : a microsatellite based survey using ABC methods. \emph{Mol. Ecol.}, \textbf{16}, 3069-3083.
\bibitem[Pollock DD \emph{et al.}, 1998]{PDD1998}Pollock DD, Bergman A, Feldman MW, Goldstein DB, 1998. Microsatellite behavior with range constraints: parameter
estimation and improved distances for use in phylogenetic reconstruction. \emph{Theoretical Population Biology}, \textbf{53}, 256–271.
\bibitem[Pritchard \emph{et al.}, 1999]{P1999} Pritchard, J., M. Seielstad,
A. Perez-Lezaun and M. Feldman, 1999. Population growth of human Y
chromosomes: a study of Y chromosome microsatellites. \emph{Mol.
Biol. Evol.} \textbf{16}, 1791-1798.
\bibitem[Rannala and Moutain, 1997]{RM1997}Rannala, B., and J. L. Mountain, 1997. Detecting immigration by using multilocus genotypes. \emph{Pro. Nat. Acad. Sci. USA} \textbf{94}, 9197-9201.
\bibitem[Raymond and Rousset, 1995]{RR1995} Raymond M., and F. Rousset, 1995. Genepop (version 1.2), population genetics software for exact tests and ecumenicism. \emph{J. Hered.}, \textbf{86}, 248-249
\bibitem[Schug  \emph{et al.}, 1997]{SMA1997} Schug M.D., T.F. Mackay and C.F. Aquadro, 1997. Low mutation rates of microsatellite loci in \emph{Drosophila melanogaster}. \emph{Nat Genet.} \textbf{15}, 99-102.
\bibitem[Stephens and Donnelly, 2000]{SD2000} Stephens, M. and P. Donnelly,
 2000, Inference in molecular population genetics (with discussion).
 J. R. Stat. Soc. B \textbf{62}, 605-655.
 \bibitem[Tajima, 1989]{TA1989}Tajima, F., 1989. Statistical method for testing the neutral mutationhypothesis by DNA polymorphism. \emph{Genetics} 
123: 585-595
 \bibitem[Tamura and Nei, 1993]{TN1993} Tamura, K., and M. Nei., 1993. Estimation of the number of nucleotide substitutions in the control region of mitochondrial DNA in humans and chimpanzees. \emph{Molecular Biology and Evolution} 10:512-526.
\bibitem[V\'azquez  \emph{et al.}, 2000]{VPAD2000} V\'azquez F.J., T. P�rez, J. Albornoz and A. Dom\'inguez, 2000.Estimation of microsatellite mutation rates in \emph{Drosophila melanogaster}. \emph{Genet Res.}, \textbf{76}, 323-6.
\bibitem[Weir and Cockerham, 1984]{WC1984}Weir BS and CC Cockerham , 1984. Estimating F-statistics for the analysis of population structure. \emph{Evolution} \textbf{38}: 1358-1370.
\bibitem[Whittaker \emph{et al.},
2003]{W2003} Whittaker, J.C., Harbord, R.M., Boxall, N., Mackay, I.,
Dawson, G. and Sibly, R.M. 2003. Likelihood-based estimation of
microsatellite mutation rates, \emph{Genetics}, \textbf{164},
781-787.
\end{thebibliography}

\end{document}