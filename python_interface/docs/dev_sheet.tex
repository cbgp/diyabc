\documentclass[12pt,a4paper]{article}
\usepackage[utf8]{inputenc}
\usepackage[francais]{babel}
\usepackage[T1]{fontenc}
\usepackage{graphicx}
\usepackage{fancyhdr}
\usepackage{geometry}
\usepackage{hyperref}
\author{Veyssier Julien}
%\institute{Université de Montpellier 2}

\title{Documentation sur le développement de \\
DIYABC en python}
\date\today
%\pagestyle{headings}
\pagestyle{fancy}
\fancyhf{}
\fancyhead[L]{\thepage}
\fancyhead[R]{DIYACB python doc | Veyssier}
%\fancyhead[LE,RO]{\thesection}

\geometry{hmargin=2.5cm,vmargin=2cm}

\begin{document}

\maketitle
%\begin{center}powered by \LaTeX\end{center}
\newpage

\tableofcontents

\newpage
 

\section{Introduction}
Ce projet est la refonte de DIYABC, initialement programmé en delphi, en PyQt.
	\subsection{Objectifs}
        Modulariser le code de sorte à permettre facilement l'ajout de nouveaux blocs, tant graphiques que fonctionnels.

	\subsection{Choix d'outils}
        La librairie Qt a été choisie pour sa portabilité, sa flexibilité et sa facilité d'utilisation. Le langage python 
        a été choisi pour sa clarté, sa réutilisabilité et son grand nombre de librairies. QtCreator et/ou QtDesigner sont utilisés pour
        la production des objets graphiques.

\section{Conception}
    \subsection{Partie graphique}
        La partie purement graphique est découplée du fonctionnement. Des fichiers .ui sont produits à partir d'un IDE. Toute entité graphique
        étant succeptible d'être présente plusieurs fois dans l'application est produite indépendament dans un .ui et une classe associée.
        
        L'ui principal est celui de la fenêtre principale.
        Un ui peut n'être qu'un widget. Un fichier source python produit à partir d'un .ui avec ``pyuic'' contient une classe non graphique
        qui crée tous les objets graphiques à l'intérieur de l'objet que l'on veut créer.

        Exemple : Création d'une scroll area.
        \begin{itemize}
            \item Création d'une ui pour QScrollArea
            \item Génération du code python avec pyuic4
            \item Instanciation d'un objet QScrollArea
            \item Appel de la methode setupUi de la classe générée avec l'objet QScrollArea en paramètre
        \end{itemize}
    \subsection{Coeur de l'application}
        La classe principale de l'application dérive de QMainWindow


\section{Documentation}

\section{Construction du projet}

\section{Utilisation}

\section{Limites et améliorations}


\end{document}
