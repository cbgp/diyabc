\documentclass[12pt,a4paper]{article}
\usepackage[utf8]{inputenc}
\usepackage[francais]{babel}
\usepackage[T1]{fontenc}
\usepackage{graphicx}
\usepackage{fancyhdr}
\usepackage{geometry}
\usepackage{hyperref}
\author{Veyssier Julien}
%\institute{Université de Montpellier 2}

\title{Documentation sur le développement de \\
DIYABC en python}
\date\today
%\pagestyle{headings}
\pagestyle{fancy}
\fancyhf{}
\fancyhead[L]{\thepage}
\fancyhead[R]{DIYACB python doc | Veyssier}
%\fancyhead[LE,RO]{\thesection}

\geometry{hmargin=2.5cm,vmargin=2cm}

\begin{document}

\maketitle
%\begin{center}powered by \LaTeX\end{center}
\newpage

\tableofcontents

\newpage
 

\section{Introduction}
Ce projet est la refonte de DIYABC, initialement programmé en delphi, en PyQt.
	\subsection{Objectifs}
        Modulariser le code de sorte à permettre facilement l'ajout de nouveaux blocs, tant graphiques que fonctionnels.

	\subsection{Choix d'outils}
        La librairie Qt a été choisie pour sa portabilité, sa flexibilité et sa facilité d'utilisation. Le langage python 
        a été choisi pour sa clarté, sa réutilisabilité et son grand nombre de librairies. QtCreator est utilis\'e pour
        la production des objets graphiques.

\section{Conception}
    \subsection{Partie graphique}
        La partie graphique statique est découplée du fonctionnement. Des fichiers .ui sont produits à partir d'un IDE. Toute entité graphique
        étant succeptible d'être présente plusieurs fois dans l'application est produite indépendament dans un .ui et une classe associée.
        
        L'ui principal est celui de la fenêtre principale.
        Un ui peut n'être qu'un widget. Un fichier source python produit à partir d'un .ui avec ``pyuic'' contient une classe non graphique
        qui crée tous les objets graphiques à l'intérieur de l'objet que l'on veut créer.

        Exemple : Création d'une scroll area.
        \begin{itemize}
            \item Création d'une ui pour QScrollArea
            \item Génération du code python avec pyuic4 :\begin{verbatim} pyuic4 mon.ui > monui.py \end{verbatim}
            \item Instanciation d'un objet QScrollArea (on peut créer une classe qui dérive de QScrollArea ou bien instancier un vrai QScrollArea)
            \item Appel de la methode setupUi de la classe générée avec l'objet QScrollArea en paramètre
                \begin{verbatim}
                from monui.py import Ui_QScrollArea
                maScrollArea = QScrollArea()
                ui = Ui_QScrollArea()
                ui.setupUi(maScrollArea)
                \end{verbatim}
        \end{itemize}
    \subsection{Coeur de l'application}
        La classe principale de l'application dérive de QMainWindow. Elle est instanci\'ee dans le main du fichier qui la contient.


    \subsection{Proposition d'architectures}
        On pourrait dissocier complètement la partie graphique du fonctionnement, c'est à dire avoir une partie du programme
        qui ne manipule que des objets conceptuels (projet, historical model...) qui appelle son système d'affichage à souhait.

        On pourrait faire une classe abstraite Displayer dérivée en QtDisplayer et CliDisplayer.


\section{Documentation}

\section{Construction du projet}
    Pour générer un exécutable compatible avec un système, il faut : 
    \begin{itemize}
    \item Se trouver sur ce système
    \item Posséder python et toutes les librairies nécessaires au programme
    \item Disposer de pyinstaller
    \end{itemize}
    
    \subsection{Windows}
        Sous windows, plusieurs solutions sont envisageables. La première, plus simple pour l'utilisateur et le
        packageur : PyInstaller. L'ex\'ecutable produit est ind\'ependant et l\'eger. \newline

        La deuxième, Inno setup compiler, pour cr\'eer un installateur qui effectue une copie de fichier et certaines actions.
        Deux strat\'egies sont possibles. On peut copier en un bloc
        python contenant ses d\'ependances et diyabc. Le problème est le suivant, python.exe refuse de se lancer s'il n'a pas \'et\'e
        install\'e avec l'installeur officiel Python. Donc il faut tout de même d\'eclencher l'installation de Python en incluant
        l'installeur dans le setup de diyabc. Sinon on peut d\'eclencher l'installeur de Python, PyQt, Numpy, PyQwt pendant le setup de
        diyabc ce qui marche à coup sur mais est lourd pour l'utilisateur.\newline
        
        La troisème, py2exe est satisfaisante mais impose l'installation à part de certaines dll.\newline

        Pour ces trois solutions j'utilise python2.6 et les librairies ``impos\'ees'' par PyQwt5 (http://pyqwt.sourceforge.net/download.html).
        Il faut installer les versions suivantes des logiciels :
        \begin{itemize}
            \item python-2.6.2.msi
            \item numpy-1.3.0-win32-superpack-python2.6.exe
            \item PyQt-Py2.6-gpl-4.5.4-1.exe
            \item PyQwt5.2.0-Python2.6-PyQt4.5.4-NumPy1.3.0-1.exe
        \end{itemize}
        \subsubsection{pyinstaller}
        \paragraph{Anciennes versions}
        La version 1.4 nécessite une version de python <= à 2.5. L'utilisation de pyinstaller est assez simple. Il faut, 
        lancer ``python Configure.py'' pour que pyinstaller se configure par rapport au système sur lequel il est. Ensuite il faut
        lancer ``python Makespec.py --onefile path\_to\_python\_main\_source.py'' qui génère le specfile dans un dossier du même nom que
        notre source python. Ensuite ``python Build.py specfile.spec'' construit l'exécutable qui se trouvera dans le même dossier que le spec file
        dans un dossier nommé ``dist''.
        \paragraph{Solution}
        La version svn est de loin la meilleure solution. Elle marche sans aucun problème sous windows. Son utilisation est simplifi\'ee.
        Il suffit d'appeler le script pyinstaller.py avec le programme cible en paramètre. Les trois \'etapes (configure, makespec et build)
        sont ex\'ecut\'ees automatiquement si besoin. Dans le fichier .spec produit, on peut ajouter l'option icon pour personnaliser l'icone 
        de l'ex\'ecutable.
		dev doc et icones
		Utilisation : \begin{verbatim} python pyinstaller.py --onefile -w --icon=path\to\icon.ico \end{verbatim}

        \subsection{Inno setup compiler}
        http://www.jrsoftware.org/isdl.php

        La cr\'eation du fichier de config (.iss) est assist\'ee et plutôt ais\'ee. On donne les fichiers à inclure, le dossier de destination
        et quelques options. Pour plus de d\'etails, consulter le fichier .iss inclut dans le d\'epot qui est comment\'e.
        \subsection{Py2exe}
        Le site de py2exe contient un tuto simple auquel il faut ajouter quelques d\'etails pour que ça fonctionne parfaitement. Le
        fonctionnement est simple. On \'ecrit un setup.py puis on ex\'ecute \begin{verbatim} python setup.py py2exe \end{verbatim} Cela Cr\'ee un dossier 
        Build et Dist. Dans ce dernier se trouve l\'executable ainsi
        Voici un exemple de setup.py incluant la copie des donn\'ees n\'ecessaires au programme ainsi que la r\'esolution du problème
        li\'e à la librairie SIP:
        \begin{verbatim}
    from distutils.core import setup
    import py2exe
    from glob import glob
    import os

    data_files = [("docs", glob(r'docs\*.*')),("docs/accueil_pictures",
                    glob(r'docs/accueil_pictures/*.*'))]

    setup(console=[{"script":"diyabc.py"}],data_files=data_files, 
                options={"py2exe":{"includes":["sip"]}})
        \end{verbatim}
        L'option ``window'' plutôt que ``console'' ne fonctionne pas.
    \subsection{Linux}
        \subsubsection{Installer PyQt4}
        PyQt4 et PyQwt5 sont packagés pour toutes les distributions. Aucun problème de ce côté.
        \subsubsection{pyinstaller}
        Pyinstaller semble ne pas fonctionner parfaitement sous linux. L'exécutable produit ne marche que
        sur la machine qui l'a créé. Je suis d'avis qu'il faut plutôt faire un paquet debian et/ou rpm
        et distribuer le programme de cette façon.
        \subsection{Script de génération du paquet Debian}
        Ce script se situe dans python\_interface/docs/project_builders/debian . Il utilise le template se trouvant dans le même
        répertoire. Dans un premier temps, il modifie les fichiers d'information du paquet pour coller avec la version. Ensuite, il 
        copie les sources nécessaires dans l'arborescence du paquet. Il écrit enfin le contenu du script de lancement qui sera dans
        /usr/local/bin.

    \subsection{MacOs 10.6 Snow Leopard}
        Sous MacOs, on pourrait cr\'eer un pkg qui installe python et les d\'ependances de DIYABC. C'est un peu lourd
        pour l'utilisateur. Py2app ne fonctionne pas, après avoir compil\'e un apptemplate pour i386\_x64
        il subsiste un problème dont l'origine m'est inconnue. Au lancement
        du .app, ``DIYABC error'' apparait et je n'ai aucun moyen de savoir pourquoi. J'ai donc retenu pyinstaller qui fonctionne parfaitement.

        Sur la machine qui va g\'en\'erer le .app, il faut avoir un environnement de d\'eveloppement complet, à savoir
        python2.6, PyQt4 et PyQwt5.

        \subsubsection{Installer PyQt4}

        \paragraph{Problèmes}

        PyQt n'est pas disponible en binaire sur MacOs. Il faut donc avoir Qt et gcc pour le compiler. 
        Gcc n'est que très difficilement installable sans Xcode. Il faut donc installer Xcode.
        En suivant à peu près les instructions données à cette adresse : \newline
        http://deanezra.com/2010/05/setting-up-pythonqt-pyqt4-on-mac-os-x-snow-leopard/\newline
        , on arrive à bout de cette installation qui est tout de même un gros morceau pour pas grand chose.
    
        Pour la version 4.8.3 de PyQt, je n'ai trouvé d'autre solution que de commenter les 3 lignes qui concernent
        ``designer'' pour que la compilation fonctionne. En cas d'erreurs incompréhensibles, bien penser à faire un ``make clean''
        pour effacer les .o déjà compilés qui peuvent provenir d'une compilation précédente mal paramétrée.

        \paragraph{Solution}

        La solution est d'utiliser les macports pour installer PyQwt (py26-pyqwt), ce qui installera tout le n\'ecessaire. Il faut tout de même
        faire attention à numpy qui pose un problème de version de lib vector. La solution est de d\'esinstaller numpy et de l'installer avec l'option
        ``-atlas''. 
        
        \subsubsection{pyinstaller}

        Les version 1.4 et 1.5 ne fonctionnent pas. Par contre,
        pyinstaller svn est tout à fait op\'erationnel avec python2.6. Il faut tout de même effectuer une s\'erie d'op\'erations avant et après
        le build pour que le .app soit op\'erationnel.Se r\'er\'erer à \newline
        http ://diotavelli.net/PyQtWiki/PyInstallerOnMacOSX \newline
        Protocole : 
        \begin{itemize}
            \item Lancer une première fois ``python pyinstaller.py diyabc.py''
            \item Ajouter à la fin de diyabc.spec :
        \begin{verbatim}
    import sys 
    if sys.platform.startswith(``darwin''): 
        app = BUNDLE(exe, 
        appname='DIYABC', 
        version='1.0')
        \end{verbatim}
            \item Relancer pyinstaller sur le specfile: ``python pyinstaller diyabc/diyabc.spec''
            \item Changer une valeur dans le fichier Info.list
            \item Copier le contenu du dossier dist/nom\_du\_projet dans le dossier MacOs du .app
            \item Copier /Library/Frameworks/QtGui.framework/Versions/4/Resources/qt\_menu.nib dans le dossier Ressources du .app
            \item Copier l'icone dans le dossier Ressources avec le nom ``App.icns''.
        \end{itemize}

\section{Utilisation}

\section{Limites et améliorations}


\end{document}
