\documentclass[12pt,a4paper]{article}
\usepackage[utf8]{inputenc}
\usepackage[francais]{babel}
\usepackage[T1]{fontenc}
\usepackage{graphicx}
\usepackage{fancyhdr}
\usepackage{geometry}
\usepackage{hyperref}
\author{Veyssier Julien}
%\institute{Université de Montpellier 2}

\title{DIYABC 2 documentation}
\date\today
%\pagestyle{headings}
\pagestyle{fancy}
\fancyhf{}
\fancyhead[L]{\thepage}
\fancyhead[R]{DIYABC 2  doc | CBGP}
%\fancyhead[LE,RO]{\thesection}

\geometry{hmargin=2.5cm,vmargin=2cm}

\begin{document}

\maketitle
%\begin{center}powered by \LaTeX\end{center}
\newpage

\tableofcontents

\newpage
 

\section{Introduction}

\section{Documentation}
        \paragraph{plop}
        \label{doc_rmScButton}
        Supprime le scenario

        \paragraph{plop}
        \label{doc_setSumButton}
        set summary stats for this group

        \paragraph{plop}
        \label{doc_rmCondButton}
        remove a condition

        \paragraph{plop}
        \label{doc_setCondButton}
        set a condition !

        \paragraph{plop}
        \label{doc_scplainTextEdit}
        contenu du scenario \newline exemple : \newline N N N

        \paragraph{plop}
        \label{doc_nbGroupLabel}
        nb group quoi

        \paragraph{plopavant}
        \label{doc_versionLabel}
        This is my default description without tag!!!

        \paragraph{plopapres}
        plop
        \label{doc_openProjectButton}
        A project is a unit of work materialized by a specific and unique directory. A project is defined by at least one observed data set and one reference table header file. These files are located in the \emph{Project directory} which name includes an identifier, the date of creation and a number (between 1 and 100).\\

        The header file, always named \texttt{header.txt}, contains all information necessary to compute a reference table associated with the data : i.e. the scenarios, the scenario parameter priors, the characteristics of loci, the loci parameter priors and the summary statistics to compute.
        As soon as the first records of the reference table have been saved in the reference table file,  always named \texttt{reftable.bin} and also included in the project directory, the project is ``locked''. This means that the header file can not be changed anymore. If one needs to change a scenario or a parameter prior, or a summary statistics, a new project needs to be defined. This is to guarantee that all subsequent actions performed on the project are in coherence with the current data and header files. It is of course strongly advised NOT to move files among projects.
        Incidentally, the \texttt{header.txt} file is only built when the project has been saved, the information progressively input by the user being saved in a series of temporary files.\\

        Once a sufficiently large reference table has been simulated, analyses can be performed. Their different output files are copied to the \emph{analysis} directory included in the project directory, and containing as many directories as analyses performed. Hence, it is now much easier to know with certainty the conditions of each analysis.    


        \paragraph{plop}
        \label{doc_versionLabel+++plop+++plap}
        Version of DIYABC. This number is incremented each time a new build is done.
        plop

        \subsection{exemple}
        \label{doc_nbMicrosatLabel}
        nbMicrosatLabel doc
            \paragraph{ploup}
            \label{doc_nbSequencesLabel}
            nbSequencesLabel doc

            \subsubsection{subsub}
            \label{doc_versionLabel+++plip}
            deuxieme pour version doc

\end{document}
