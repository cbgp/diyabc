\documentclass[12pt,a4paper]{article}
\usepackage[utf8]{inputenc}
\usepackage[francais]{babel}
\usepackage[T1]{fontenc}
\usepackage{graphicx}
\usepackage{fancyhdr}
\usepackage{geometry}
\usepackage{hyperref}
\author{Veyssier Julien}
%\institute{Université de Montpellier 2}

\title{Documentation sur le développement de \\
DIYABC version 2 en PyQt4}
\date\today
%\pagestyle{headings}
\pagestyle{fancy}
\fancyhf{}
\fancyhead[L]{\thepage}
\fancyhead[R]{DIYABC 2 python doc | Veyssier}
%\fancyhead[LE,RO]{\thesection}

%\geometry{hmargin=2.5cm,vmargin=2cm}

\begin{document}

\maketitle
%\begin{center}powered by \LaTeX\end{center}
\newpage

\tableofcontents

\newpage
 

\section{Introduction}
Ce projet est la refonte de DIYABC, initialement programmé en delphi, en PyQt.
	\subsection{Objectifs}
        Modulariser le code de sorte à permettre facilement l'ajout de nouveaux
        blocs, tant graphiques que fonctionnels.

	\subsection{Choix d'outils}
        La librairie Qt a été choisie pour sa portabilité, sa flexibilité et sa
        facilité d'utilisation. Le langage python a été choisi pour sa clarté,
        sa réutilisabilité et son grand nombre de librairies. QtCreator est
        utilis\'e pour la production des objets graphiques.

\section{Conception}
    \subsection{Documentation}
    \subsubsection{Doxygen}
    L'intégralité du code est documenté façon python docString. Ces docstrings
    sont donc interprêtables par beaucoup de générateurs de documentation
    spécifiques à python et par doxygen.  Je génère une documentation html du
    code avec Doxygen. Un fichier de configuration de doxygen nommé doxyconfig
    est versionné. Il se trouve à la racine des sources python, c'est à dire le
    dossier python\_interface. En tête des fichiers, des commentaires type
    javadoc sont présent pour que les options comme @author soient pris en
    compte.\\

    \subsubsection{Dans l'interface}
    Une aide à l'utilisateur est intégrée à l'interface. Elle est construite à
    partir d'un fichier xml lui même généré à partir des sources \LaTeX de la
    notice de DIYABC. La génération de cette documentation est décrite en 
    \ref{howtoLatexml}
    Pendant l'utilisation de l'interface, cette aide est
    consultable en effectuant un clic sur le bouton ``What's this'', situé en
    haut à droite de la fenêtre principale, puis sur l'objet pour lequel on veut
    obtenir de l'aide.

    Pour documenter un objet graphique, il faut connaitre son objectname et
    insérer dans le source \LaTeX de la notice 
    un bloc de la forme : 
    \begin{verbatim}\label{doc_objectName}
La description de l'objet (bouton, label...)\end{verbatim}
    La description s'arrête au prochain délimiteur de section comme un
    paragraphe, une section, une sous section \ldots \\

    Tout objet pour lequel aucune documentation n'a été spécifiée aura totu de
    même son objectname dans l'aide ``what's this'' si l'option ``Show objectname in what's
    this'' a été activé.


    \subsection{Partie graphique}
    \subsubsection{Pyuic}
        La partie graphique statique est partiellement découplée du
        fonctionnement. Des
        fichiers .ui sont produits à partir d'un IDE. Toute entité graphique
        étant succeptible d'être présente plusieurs fois dans l'application est
        produite indépendament dans un .ui et une classe associée.\\
        
        L'ui principal est celui de la fenêtre principale.  Un ui peut n'être
        qu'un widget. Un fichier source python produit à partir d'un .ui avec
        ``pyuic'' contient une classe non graphique qui crée tous les objets
        graphiques à l'intérieur de l'objet que l'on veut créer.\\

        Exemple : Création d'une scroll area.\\
        \begin{itemize}
            \item Création d'un fichier .ui pour QScrollArea avec QtCreator par
                exemple
            \item Génération du code python avec pyuic4 :\begin{verbatim} pyuic4 mon.ui > monui.py \end{verbatim}
            \item Instanciation d'un objet QScrollArea (on peut créer une classe qui dérive de QScrollArea ou bien instancier un vrai QScrollArea)
            \item Appel de la methode setupUi de la classe générée dans monui.py avec l'objet QScrollArea en paramètre
                \begin{verbatim}
        from monui import Ui_QScrollArea
        maScrollArea = QScrollArea()
        ui = Ui_QScrollArea()
        ui.setupUi(maScrollArea)
\end{verbatim}
        \end{itemize}
    \subsubsection{Loaduic}
    Une autre technique consiste à charger le fichier .ui pour obtenir deux
    classes (form et base) desquelles on va faire hériter la classe que l'on
    crée. Pour Diyabc qui est une QMainWindow, cela se présente comme cela :
    \begin{verbatim}
    from PyQt4 import uic
    formDiyabc,baseDiyabc = uic.loadUiType('uis/diyabc.ui')
    class Diyabc(formDiyabc,baseDiyabc):
        def __init__(self,app,parent=None,projects=None,logfile=None):
            self.ui=self
            self.ui.setupUi(self)
\end{verbatim}
    L'appel de setupUi a le même effet que dans l'autre solution. L'avantage de
    cette solution est d'économiser l'étape où l'on utilise le programme pyuic4
    à la main.
    \subsection{Coeur de l'application}
        La classe principale de l'application dérive de QMainWindow. Elle est
        instanci\'ee dans le main du fichier qui la contient. La QMainWindow
        contient une zone à onglet. Chaque onglet est un projet. Un projet est
        un ensemble de plusieurs informations :\\

        \begin{itemize}
            \item Un fichier de données
            \item Un modèle historique
            \item Des données génétiques
            \item Une table de référence
            \item Des analyses\\
        \end{itemize}

        Un projet peut être sauvegardé et chargé par la suite. La sauvegarde
        consiste en l'écriture de plusieurs fichiers de sauvegarde qui se
        trouvent dans le dossier du projet.\\

        \begin{itemize}
            \item conf.hist : modèle historique
            \item conf.gen : données génétiques
            \item conf.analysis : sérialisation des analyses
            \item conf.th : table header (dernière partie du reftable header)
            \item conf.tmp : entête du reftable header\\
        \end{itemize}

        Ces fichiers sont lu lors du chargement d'un projet pour rétablir son
        état. Pendant la manipulation d'un projet, un fichier de verrou (.lock)
        est créé dans le dossier du projet. Il permet de s'assurer que l'on ne
        manipule pas le même projet avec deux instances de DIYABC. Si
        l'interface a été arrêtée brutalement, le verrou n'est pas enlevé. Un
        message d'avertissement expliquant la situation est donc présenté au
        chargement du projet. On peut outrepasser le verrou si on est sur de ce
        que l'on fait.

    \subsection{Le modèle historique}
        La conception du modèle historique n'est pas très complexe. Les gestions
        des scénarios, de conditions et des paramètres sont similaires, du moins
        en ce qui concerne la GUI. Les méthodes addSc, addCondition et
        addParamGui ajoutent graphiquement ces entités et les mémorisent dans
        des attributs de la classe SetHistoricalModel .

    \subsection{Les données génétiques}
        La classe SetGenData contient un tableau de locus et une liste de
        groupes de locus. Pour chaque groupe, elle instancie deux modèles
        mutationnels et deux ensembles de summary statistics (microsats et
        sequences). Lorsque l'utilisateur veut définir le modèle mutationnel
        d'un groupe contenant des locus, le modèle mutationnel correspondant
        (microsats ou sequences), déjà instancié, est affiché.

    \subsection{Les préférences}
        Les préférences sont stockées dans le dossier personnel de
        l'utilisateur dans le fichier ~/.diyabc/config.cfg .
        Les valeurs sont stockées sous la forme clé
        : valeur selon un format de fichier de configuration assez répandu.
        Elles sont divisées en plusieurs catégories qui correspondent au onglets
        de la fenêtre de préférences.
        Les préférences sont
        appliquées dès leur modification dans la fenêtre. Un clic sur annuler
        recharge les préférences à partir du fichier de config, c'est à dire
        avant les modifications effectuées.
        

    \subsection{La génération de la table de référence}
        \paragraph{En local}
        Elle s'effectue dans un QThread qui scrute l'avancement dans le fichier
        progress.txt . Pour stopper l'exécutable de calcul, il faut créer un
        ficher .stop dans le dossier du projet.\\

        Un signal est prévu en cas de problème comme la fin prématurée de
        l'exécutable de calcul. La gui l'indique alors dans une popup
        d'information.  Il en est de même pour le calcul des analyses.


        \paragraph{Sur le cluster}
        J'ai implémenté un fonctionnalité permettant à l'interface de lancer une
        génération sur un cluster qui marche avec SGE ou un système compatible.
        L'interface crée une archive tar, se connecte (par socket) à un serveur
        écrit en python qui se trouve sur un des noeuds du cluster. Elle
        transfere l'archive par cette connexion.  Le serveur, une fois
        l'archive réceptionnée, extrait son contenu et exécute le script
        launch.sh qui s'occupe de lancer les qsub. Chaque job lance l'exécutable
        de calcul en arrière plan et transfère le fichier d'avancement local au
        noeud maitre à intervalle de temps régulier. Le serveur fait
        régulièrement la somme des avancements et la transmet à intervalle
        régulier à l'interface par le réseau afin qu'elle se mette à jour.


    \subsection{Les analyses}
        Nous n'avons pas jugé nécessaire de prévoir l'exécution des analyses sur
        le cluster. Il faudrait transférer la table de référence et le temps de
        calcul n'est souvent pas très long, le gain serait peut-être donc
        négatif.\\

        L'écran principal de définition d'une analyse se trouve ans le fichier
        defineNewAnalysis.py . En fonction du type d'analyse sélectionné,
        l'écran suivant est instancié. Pour certains cas, comme l'écran de
        sélection de scénario, l'écran qui doit succéder est passé en paramètre
        au constructeur de l'écran. Pour les autres cas, la classe qui
        implémente l'écran instancie l'écran suivant directement en fonction de
        l'analyse dont il est question. Chaque écran ajoute les informations qui
        lui sont propres à l'analyse. Le dernier écran est chargé de donner
        l'analyse à la classe projet pour qu'elle l'ajoute à la GUI.\\

        Le lancement des analyse est effectué dans un QThread qui scrute
        l'avancement dans le fichier prévu à cet effet. A chaque avancement, il
        emet un signal capté par le thread principal qui met à jour la GUI. Une
        fois l'analyse terminée, c'est la méthode qui capte le signal
        d'avancement qui appelle la méthode qui termine le thread d'analyse et
        qui range les résultats dans un dossier du nom de l'analyse.

    \subsection{Les détails}
        \paragraph{Les logs}
        Les logs sont gérés par une classe (Teelogger) se situant dans output.py. Une instance de Teelogger
        se substitue à stdout et une autre à stderr pour intercepter tout message et effectuer : 
        \begin{itemize}
            \item Une écriture dans un fichier de log (propre à l'instance de DIYABC)
            \item Un affichage dans la fenêtre de log accessible par le menu Help
            \item Un affichage en couleur sur la bonne sortie (out ou err) \\
        \end{itemize}

        Les logs sont produits de plusieurs façon. 
        \begin{itemize}
            \item Par la fonction log présente dans output.py
            \item Par l'instruction print de Python
            \item Par le déclenchement d'une exception arrivant au sommet sans être attrapée\\
        \end{itemize}
        
        La fonction log permet de préciser le niveau de log que l'on souhaite utiliser. J'ai pensé mes niveau comme cela :
        \begin{itemize}
            \item 1 : signalement des actions de l'utilisateur
            \item 2 : signalement des appels de fonctions majeures (celles de Project principalement)
            \item 3 : affichage de valeurs
            \item 4 : détail
        \end{itemize}

        Pour des raisons de lisibilité, les logs sont épurés sur la sortie standart et dans le programme. Les logs réels
        se trouvent dans le fichier de log, c'est à dire dans le dossier de configuration de DIYABC.

        La ligne de log contient la date, l'heure, le niveau de log, la fonction appelante de niveau 1 et 2 ainsi que les paramètres
        donnés à la fonction appelante. 
        
        \paragraph{logrotate}
        Une fonction logRotate est appelée au lancement de l'application. Si le dossier passé en paramètre dépasse une taille H,
        les fichiers vieux de plus de N jours sont supprimés.

        \paragraph{La console intégrée}
        Si on donne l'argument -cmd au lancement de diyabc.py, on active la console intégrée. Elle permet à tout moment d'exécuter du code python
        dans le contexte de l'application qui tourne. L'instance principale et unique de Diyabc est un attribut de la console. Elle s'utilise de la
        manière suivante : \newline On commence la ligne par le mot ``do'' puis on écrit du python. Par exemple : 

        \begin{verbatim}do print len(self.diy.project_list)\end{verbatim}
        
        Affichera le nombre de projets ouverts par l'interface.

        \paragraph{gconf}
        Par défaut, dans la configuration de gnome, les icônes dans les menus sont désactivées. Au lancement de diyabc.py,
        si le système est un linux*, gconftool-2 est appelé pour activer cette option.

        \paragraph{drag and drop}
        On peut faire glisser un dossier sur la fenêtre principale de l'interface pour ouvrir un projet. Pour récupérer 
        cet évènement, on redéfinit les méthodes dropEvent et dragEnterEvent pour la classe Diyabc. Dans dropEvent, on
        récupère l'information donnée par l'objet déposé. Si ces informations contiennent des url, on essaie de les ouvrir 
        comme des projets.

\section{Tests}
    Les tests sont tous dans le fichier unit\_test.py . Ils utilisent le module
    python unittest.
    \subsection{Conception}
    \paragraph{Test général}
    Les opérations de haut niveau sur un projet sont testées dans la méthode
    testGeneral . Attention, ce test nécessite l'existence d'un projet. 
    Un projet existant étant versionné, ce test est opérationnel mais dans le
    cas où ce projet n'existe pas ou bien qu'il a été rendu incohérent par une
    modification à la main, on peut toujours changer les sources du test afin de
    définir correctement les chemins qui pointent statiquement sur un projet
    existant.
    \paragraph{Test des clics}
    La méthode testAllClicks simule une intéraction avec l'interface graphique.
    Un arbre des clics possibles est généré pour ensuite former une liste de
    toutes les séquences de clic possibles. L'arbre de clics permet de
    connaitre, pour un bouton donné, la liste des boutons candidats (les fils)
    pour le clic suivant. Il est evidemment possible de trouver des cycles dans
    les séquences générées. Un paramètre permet de fixer une borne maximum sur
    le nombre de fois que l'on clique sur un même bouton dans une séquence. \\

    Ce test n'utilise pas d'assertion, il a pour but d'identifier une suite
    d'opérations provoquant l'arrêt prématuré de l'interface.

    \paragraph{Test du modèle historique}
    Le fonctionnement général et la validation du modèle historique sont testés 
    dans la méthode testHistoricalModel. Cette méthode teste des exemples de valeurs
    représentatifs des cas possibles, pour chaque paramètre, avec chaque loi, pour
    chaque valeur.

    \subsection{Lancement}
    Pour effectuer les tests, il suffit de lancer le script python unit\_test.py
    qui contient une procédure principale. Il est conseillé de mettre la
    constante ``debug'' à True dans output.py pour éviter les popups
    d'information qui sont bloquants.\\

    Le premier échec sur une assertion ou plantage provoque l'arrêt des tests.\\

    A la fin des tests, si tout s'est bien passé, un résumé est affiché.
    L'erreur de segmentation à la toute fin des tests semble inévitable. Elle
    survient après tous les affichages et n'est donc pas gênante.


\section{Construction du projet}

    \subsection{Introduction}
    Pour construire le projet en vue de le distribuer, on dispose de trois
    scripts, un pour chaque système, qui vont générer un exécutable, une
    application ou un paquet. Cet objet généré est transportable et distribuable
    tel quel avec un dossier contenant des données. Ces scripts se trouvent dans
    le dossier python\_interface/docs/project\_builders/ .\\

    Il est également possible d'utiliser buildbot pour générer les ``packages''. 
    Buildbot permet de gérer des machines esclaves à partir d'une machine maitre.
    Buildbot s'occupe de transférer les sources sur les machines esclaves et permet
    de définir les actions à exécuter sur ces différentes machines pour construire
    le projet. On controle buildbot avec une interface web lancée par la machine maitre.
    Le seul fichier de configuration important est celui de la machine maitre où l'on définit
    les esclaves et leur processus de construction. Ce fichier se trouve aussi dans
    python\_interface/docs/project\_builders/ .

    Pour générer un package sous GNU/Linux, MacOS ou Windows, il faut : 
    \begin{itemize}
    \item Se trouver sur le système en question
    \item Posséder python et toutes les librairies nécessaires au programme
    \item Disposer de pyinstaller\\
    \end{itemize}

    J'ai aussi implémenté un script qui génère un paquet .deb.
    Pas de dépendance manuelle particulière pour générer un paquet sous Debian ou
    Ubuntu.\\

    \paragraph{La mise en place de buildbot} se fait naturellement en deux temps,
    l'installation et la configuration. Installer buildbot et buildslave sur la machine maitre
    sous GNU/Linux se fait par le système de paquets. Sous windows, j'ai suivi la procédure
    indiquée par le site officiel de buildbot à savoir installer pywin32, twisted, setuptools, zope.interface
    et enfin buildbot par easy\_install . Sous MacOS, il se trouve dans les macports et dans fink. Rien n'est
    à configurer sur les esclaves. La config du master est dans le dépôt. On crée le maitre comme ceci :

    buildbot create-master master

    Cela a pour effet de créer le dossier master dans le dossier courant.

    On crée les esclaves comme ceci :

    buildslave create-slave myslavefolder localhost:9989 myslavename pass
    
    On démarre le maitre comme ceci :

    buildbot start master

    Et les esclaves comme ceci :

    buildslave start slave



    \paragraph{La génération de l'aide interne} est très simple.
    \label{howtoLatexml}
    On part de la notice écrite en Latex qui doit être taguée. On peut ensuite générer 
    un fichier xml ou un fichier html qui sera utilisé par l'interface pour 
    sa documentation interne.
    Pour n'importe lequel de ces systèmes (Mac, Windows, GNU/Linux), générer 
    un xml à partir de la notice \LaTeX se fait
    avec ``latexml'' qui est présent dans les dépôts Debian et Ubuntu. \\
    \begin{verbatim}latexml notice.tex > documentation.xml\end{verbatim}

    Il faut que le fichier documentation.xml soit placé dans le dossier python\_interface/docs/ .

    Pour générer la documentation en html, il faut disposer de latex2html. Un 
    script ( python\_interface/docs/gen\_html\_doc.sh ) permet de manipuler 
    correctement les options de latex2html. Il se lance sans paramètre 
    et copie le nécessaire au bon endroit.

    \paragraph{La compilation} de l'exécutable de calcul s'effectue avec un makefile situé
    à la racine du dépôt. Ce makefile prend un paramètre facultatif nommé CCVERSION qui est
    le suffixe à ajouter à gcc et g++ pour compiler.

    
    \subsection{Windows}
        Sous windows, plusieurs solutions sont envisageables. La première, plus
        simple pour l'utilisateur et le packageur : PyInstaller. L'ex\'ecutable
        produit est ind\'ependant et l\'eger. \newline

        La deuxième, Inno setup compiler, pour cr\'eer un installateur qui
        effectue une copie de fichier et certaines actions.  Deux strat\'egies
        sont possibles. On peut copier en un bloc python contenant ses
        d\'ependances et diyabc. Le problème est le suivant, python.exe refuse
        de se lancer s'il n'a pas \'et\'e install\'e avec l'installeur officiel
        Python. Donc il faut tout de même d\'eclencher l'installation de Python
        en incluant l'installeur dans le setup de diyabc. Sinon on peut
        d\'eclencher l'installeur de Python, PyQt, Numpy, PyQwt pendant le setup
        de diyabc ce qui marche à coup sur mais est lourd pour
        l'utilisateur.\newline
        
        La troisème, py2exe est satisfaisante mais impose l'installation à part
        de certaines dll.\newline

        Pour ces trois solutions j'utilise python2.6 et les librairies
        ``impos\'ees'' par PyQwt5 (http://pyqwt.sourceforge.net/download.html).
        Il faut installer les versions suivantes des logiciels :
        \begin{itemize}
            \item python-2.6.2.msi
            \item numpy-1.3.0-win32-superpack-python2.6.exe
            \item PyQt-Py2.6-gpl-4.5.4-1.exe
            \item PyQwt5.2.0-Python2.6-PyQt4.5.4-NumPy1.3.0-1.exe
        \end{itemize}

        \subsubsection{Compilation du programme de calcul}
        La compilation de programmes écrits en C++ pose quelques problèmes sous
        windows. J'ai essayé les compilateurs suivant : 
        digital mars, Borland C++ et GCC (MinGW). Le seul avec lequel j'obtiens
        des résultats satisfaisants est GCC. 

        Voici la marche à suivre pour compiler le programme en C++ sous windows 7.\\

        Tout d'abord il faut avoir MinGW installé. Cela se fait simplement avec mingw-get
        que l'on peut récupérer sur sourceforge. Ensuite, dans un terminal il faut 
        mettre à jour la liste de paquets mingw-get :

        mingw-get update

        Puis installer les paquets suivants :
        gcc-core
        gcc-g++
        binutils
        mingw-runtime
        w32api
        mingw32-make
        mpc
        mpfr
        gmp
        expat

        Les binaires vont se loger dans mingw-get/bin/ .

        Ensuite il faut avoir ce dossier bin dans le path de son émulateur de terminal.

        \begin{itemize}
            \item Se rendre dans le dossier diyabc sur le bureau avec ``git
                bash''
            \item Effectuer un ``git checkout .'' si nécessaire
            \item Effectuer ``git pull --rebase origin master''
            \item Se rendre dans src-JMC-C++
            \item Exécuter ``g++ -fopenmp -O3 general.cpp -o general''
            \item Ou faire un make au sommet du dépôt
            \item Récupérer libstdc++-6.dll et libgcc\_s\_dw2-1.dll qui sont
                dans c:/MinGW/bin . Ces deux dll doivent se trouver à côté de
                l'exécutable pour qu'il fonctionne sur un autre système.
        \end{itemize}
        \subsubsection{pyinstaller}

        \paragraph{Anciennes versions}

        La version 1.4 nécessite une version de python <= à 2.5. L'utilisation
        de pyinstaller est assez simple. Il faut, lancer ``python Configure.py''
        pour que pyinstaller se configure par rapport au système sur lequel il
        est. Ensuite il faut lancer ``python Makespec.py --onefile
        path\_to\_python\_main\_source.py'' qui génère le specfile dans un
        dossier du même nom que notre source python. Ensuite ``python Build.py
        specfile.spec'' construit l'exécutable qui se trouvera dans le même
        dossier que le spec file dans un dossier nommé ``dist''.

        \paragraph{Solution fonctionnelle}

        La version svn est de loin la meilleure solution. Elle marche sans aucun
        problème sous windows. Son utilisation est simplifi\'ee.  Il suffit
        d'appeler le script pyinstaller.py avec le programme cible en paramètre.
        Les trois \'etapes (configure, makespec et build) sont ex\'ecut\'ees
        automatiquement si besoin. Dans le fichier .spec produit, on peut
        ajouter l'option icon pour personnaliser l'icone de l'ex\'ecutable.
		Utilisation : \begin{verbatim} python pyinstaller.py --onefile -w --icon=path\to\icon.ico main.py \end{verbatim}

        \subsubsection{Inno setup compiler}
        http://www.jrsoftware.org/isdl.php

        La cr\'eation du fichier de config (.iss) est assist\'ee et plutôt
        ais\'ee. On donne les fichiers à inclure, le dossier de destination et
        quelques options. Pour plus de d\'etails, consulter le fichier .iss
        inclut dans le d\'epot qui est comment\'e.

        \subsubsection{Py2exe}
        Le site de py2exe contient un tuto simple auquel il faut ajouter
        quelques d\'etails pour que ça fonctionne parfaitement. Le
        fonctionnement est simple. On \'ecrit un setup.py puis on ex\'ecute
        \begin{verbatim} python setup.py py2exe \end{verbatim} Cela Cr\'ee un
        dossier Build et Dist. Dans ce dernier se trouve l\'executable ainsi
        Voici un exemple de setup.py incluant la copie des donn\'ees
        n\'ecessaires au programme ainsi que la r\'esolution du problème li\'e à
        la librairie SIP:
        \begin{verbatim}
    from distutils.core import setup
    import py2exe
    from glob import glob
    import os

    data_files = [("docs", glob(r'docs\*.*')),("docs/accueil_pictures",
                    glob(r'docs/accueil_pictures/*.*'))]

    setup(console=[{"script":"diyabc.py"}],data_files=data_files, 
                options={"py2exe":{"includes":["sip"]}})
        \end{verbatim}
        L'option ``window'' plutôt que ``console'' ne fonctionne pas.
        \subsubsection{Solution choisie}
        Sous windows, j'utilise pyinstaller (révision 1355). Après avoir renoncé
        à l'écriture d'un script batch, j'ai découvert que git installait un
        bash tout à fait fonctionnel. J'ai donc écrit le script de génération
        d'un exécutable en bash.

        La génération d'un package directement distribuable (interface+binaire) est possible
        avec le script package-win-i386.sh qui met à jour le dépôt, compile les sources C++,
        et utilise pyinstaller pour générer un .exe de l'interface PyQt. S'il est exécuté sous windows
        7, il génère un binaire et un .exe qui fonctionnent sous XP, win7, win8 i386 ou x64.

        \paragraph{Utilisation}

        Ce script est : python\_interface/docs/project\_builders/windows\_generation.sh . Une 
        exécution sans paramètre affiche les paramètres à fournir : \\

        \begin{itemize}
            \item path\_to\_pyinstaller.py : chemin vers le script python de pyinstaller
            \item path\_to\_icon.ico : chemin vers le fichier ico qui sera l'icone du fichier .exe
            \item output\_path : dossier cible (existant ou non) dans lequel se trouvera l'executable et les données du programme
            \item path\_to\_main.py : chemin vers le script python principal (diyabc.py dans notre cas)\\
        \end{itemize}

        Après exécution de ce script de génération, un dossier nommé
        diyabc-VERSION se trouve dans le dossier spécifié en output. A
        l'intérieur de ce dossier se trouve l'exécutable ainsi que le dossier
        docs contenant des images et la documentation.

        \paragraph{Fonctionnement}

        Ce script récupère le numéro de version dans le fichier version.txt qui
        se trouve avec les sources python de l'interface. Il crée ensuite un
        dossier temporaire où il copie tout le nécessaire à la construction de
        l'exécutable. Il modifie la variable VERSION dans les sources
        (diyabc.py).  Il lance la génération avec pyinstaller puis copie les
        données à côté du .exe .

    \subsection{Linux}
        \subsubsection{Compilation du programme de calcul}
        Installer g++ suffit.\\

        Attention, depuis le changement de façon de gérer les nombres aléatoires, 
        il faut passer par un makefile pour compiler.

        Il est nécessaire de générer l'exécutable de calcul en 32 et 64 bits.
        Chacun se fait sur le système correspondant ou avec l'option -m32 
        pour produire du 32bits sur un système 64bits.

        \subsubsection{Installer PyQt4}

        PyQt4 et PyQwt5 sont packagés pour toutes les distributions. Aucun
        problème de ce côté.
        \subsubsection{pyinstaller}

        Pyinstaller ne fonctionnait pas parfaitement sous linux lorsque le projet
        était à la version 1355. 6 mois plus tard, la version 1743 fonctionne.
        J'ai donc aussi fait un script pour générer un bundle sous linux. Il se
        trouve ici :\\
        python\_interface/docs/project\_builders/linux\_generation.sh . \\

        La syntaxe est la suivante :\\ 
        linux\_generation.sh  path\_to\_pyinstaller.py  [output\_path
        path\_to\_main.py]

        \subsubsection{Script de génération du paquet Debian}

        Ce script se situe dans python\_interface/docs/project\_builders/debian/ .
        \paragraph{Utilisation}

        Une exécution du script sans paramètre affiche la liste des paramètres à fournir :\\

        \begin{itemize}
            \item directory\_of\_main\_source : chemin vers le dossier contenant le script python principal (python\_interface dans notre cas)
            \item main\_flag : flag qui détermine si le paquet généré comprendra ou pas la version dans son nom. S'il vaut MAIN, le paquet généré
                se nommera diyabc\ldots.deb, s'il vaut autre chose, le paquet se nommera diyabc-VERSION\ldots.deb .\\
        \end{itemize}

        Après exécution de ce script, le paquet debian se trouve dans le dossier courant.

        \paragraph{Fonctionnement}
        Ce script utilise le template se trouvant dans le même répertoire. Dans
        un premier temps, il modifie les fichiers d'information du paquet pour
        coller avec la version. Ensuite, il copie les sources nécessaires dans
        l'arborescence du paquet, modifie les fichiers nécessaires aux menus
        gnome et autres Il écrit enfin le contenu du script de lancement qui
        sera dans /usr/local/bin.

        \paragraph{Dépôt debian}
        Peut-être auront nous par la suite envie de mettre en place un dépôt
        debian pour distribuer plus facilement DIYABC aux utilisateur de Debian,
        d'Ubuntu et autres dérivés de Debian. Pour mettre en place un dépôt, il
        suffit d'avoir un serveur web opérationnel. Un dépôt n'est en fait
        simplement qu'un ensemble de dossier et de fichiers respectant un
        certain standart. Il existe plusieurs outils facilitant la création de
        cette arborescence. J'ai utilisé reprepro pour tester. En résumé, il
        suffit de créer un dossier qui sera la racine de notre dépôt :\\

        \begin{verbatim}mkdir debian \end{verbatim}

        Puis d'y créer deux répertoires conf et incoming :\\

        \begin{verbatim}mkdir debian/conf
mkdir debian/incoming \end{verbatim}

        Il faut ensuite éditer le fichier conf/distributions qui va contenir les
        types de distributions que l'on veut servir. Notre paquet étant
        fonctionnel sur toute distribution Linux, nous ne mettrons qu'une
        distribution dans ce fichier et tout le monde s'adressera au dépôt comme
        s'il possédait cette distribution. Le contenu du fichier distributions
        est de la forme : \\

        \begin{verbatim}Origin: CBGP
Label: INRA
Suite: stable
Codename: squeeze
Version: 1.0
Architectures: i386 amd64
Components: main
Description: Repository of CBGP softwares \end{verbatim}

        Une fois qu'on a un paquet prêt à être ajouté au dépôt, on utilise
        reprepro pour éviter de créer à la main toute l'arborescence dans le
        dépôt. L'ajout d'un paquet s'effectue de la manière suivante :\\

        \begin{verbatim}reprepro -Vb path/to/my/repo includedeb stable path/to/my/package.deb \end{verbatim}

        Voici deux sources de documentation pour aller plus loin avec reprepro
        :\newline
        \url{http://www.isalo.org/wiki.debian-fr/index.php?title=Reprepro}
        et\newline
        \url{http://doc.ubuntu-fr.org/tutoriel/comment_creer_depot} . On peut
        vouloir signer un dépôt avec une clé GPG ou encore faire un mirroir d'un
        autre dépôt.

    \subsection{MacOS 10.6 Snow Leopard}
        Sous MacOS, on pourrait cr\'eer un pkg qui installe python et les
        d\'ependances de DIYABC. C'est un peu lourd pour l'utilisateur. Py2app
        ne fonctionne pas, après avoir compil\'e un apptemplate pour i386\_x64
        il subsiste un problème dont l'origine m'est inconnue. Au lancement du
        .app, ``DIYABC error'' apparait et je n'ai aucun moyen de savoir
        pourquoi. J'ai donc retenu pyinstaller qui fonctionne parfaitement.\\

        Sur la machine qui va g\'en\'erer le .app, il faut avoir un
        environnement de d\'eveloppement complet, à savoir python2.6, PyQt4 et
        PyQwt5.

        \subsubsection{Compilation du programme de calcul}
        Sous macOS 10.6, le gcc/g++ fourni avec xcode est trop vieux pour supporter
        openMP. Il faut avoir au minimum la 4.7. Pour cela je conseille 
        l'installation de macports ou de fink, un autre gestionnaire de paquets pour MacOS,
        afin d'installer gcc47 qui supporte openmp.

        \paragraph{Installation des macports}
        Installation des macports : installer avec le dmg, puis mettre à jour
        avec un sync en file:// ou en http:// sur le fichier ports.tar.gz qui se trouve ici :
        \\
        www.macports.org/files/ports.tar.gz 
        \\
        Pour cela il faut ajouter une ligne à la fin
        du fichier \\
        /opt/local/etc/macports/sources.conf :\\

        http://www.macports.org/files/ports.tar.gz [default]\\

        \paragraph{Installation de fink}
        Dans l'archive de fink se trouve un script d'installation automatisée qui
        fonctionne à merveille. Ensuite une synchronisation et une installation de
        gcc47 et il devient possible de faire un make au sommet du dépôt.

        \subsubsection{Installer PyQt4}

        \paragraph{Problèmes}

        PyQt n'est pas disponible en binaire sur MacOS. Il faut donc avoir Qt et
        gcc pour le compiler.  Gcc n'est que très difficilement installable sans
        Xcode. Il faut donc installer Xcode.  En suivant à peu près les
        instructions données à cette adresse : \newline
        http://deanezra.com/2010/05/setting-up-pythonqt-pyqt4-on-mac-os-x-snow-leopard/\newline
        , on arrive à bout de cette installation qui est tout de même un gros
        morceau pour pas grand chose. Il est tout de même plus simple d'utiliser
        les macports ou fink.\\
    
        \paragraph{Solution}

        La solution est d'utiliser les macports pour installer PyQwt
        (py26-pyqwt), ce qui installera tout le n\'ecessaire. Il faut tout de
        même faire attention à numpy qui pose un problème de version de lib
        vector. La solution est de d\'esinstaller numpy et de l'installer avec
        l'option ``-atlas''. 

        \subsubsection{pyinstaller}

        Les version 1.4 et 1.5 ne fonctionnent pas. Par contre, pyinstaller svn
        1355 est tout à fait op\'erationnel avec python2.6. Il faut tout de même
        effectuer une s\'erie d'op\'erations avant et après le build pour que le
        .app soit op\'erationnel.Se r\'er\'erer à \newline http
        ://diotavelli.net/PyQtWiki/PyInstallerOnMacOSX \newline Protocole : \\

        \label{mac_pyinstaller}
        \begin{itemize}
            \item Lancer une première fois ``python pyinstaller.py diyabc.py''
            \item Ajouter à la fin de diyabc.spec :
        \begin{verbatim}
    import sys 
    if sys.platform.startswith(``darwin''): 
        app = BUNDLE(exe, 
        appname='DIYABC', 
        version='1.0')
        \end{verbatim}
            \item Relancer pyinstaller sur le specfile: ``python pyinstaller diyabc/diyabc.spec''
            \item Changer une valeur dans le fichier Info.list
            \item Copier le contenu du dossier dist/nom\_du\_projet dans le dossier MacOs du .app
            \item Copier /Library/Frameworks/QtGui.framework/Versions/4/Resources/qt\_menu.nib dans le dossier Ressources du .app
            \item Copier l'icone dans le dossier Ressources avec le nom ``App.icns''.\\
        \end{itemize}

        \subsubsection{Solution choisie}
        \paragraph{Utilisation}

        Comme sous windows, j'ai écrit un script bash qui utilise pyinstaller
        (révision 1355). Ce script se trouve dans
        python\_interface/docs/project\_builders/ .  Une exécution sans
        paramètre affiche les paramètres à fournir : \\

        \begin{itemize}
            \item path\_to\_pyinstaller.py : chemin vers le script python de pyinstaller
            \item path\_to\_icon.icns : chemin vers le fichier icns qui sera l'icone de l'application (.app)
            \item output\_path : dossier cible (existant ou non) dans lequel se trouvera l'application et les données du programme
            \item path\_to\_main.py : chemin vers le script python principal (diyabc.py dans notre cas)\\
        \end{itemize}

        Un script englobe la compilation, la génération du .app et du .dmg : package-mac-i386.sh ou
        package-mac-x86\_64.sh . Il est à ajuster légèrement en fonction des chemins que l'on souhaite définir
        sur la machine utilisée. Le i386 marche sur la machine virtuelle osxqt, il compile en statique
        le binaire, il produit un .app qui fonctionne sur les macos 32 et 64. Le script x32\_64 ne fonctionne pas
        car pyinstaller ne fonctionne pas sur macos 64 bits. Il reste néanmoins intéressant de compiler le binaire
        sur cette machine pour l'obtenir en x86\_64.

        \paragraph{Fonctionnement}

        Comme les autres, ce script crée un dossier temporaire où il copie tout
        le nécessaire à la construction du .app .  Il récupère le numero de
        version dans le fichier version.txt . Il modifie les sources en
        conséquence. Il lance pyinstaller une première fois pour générer le
        .spec . Il modifie le .spec pour produire un .app puis lance à nouveau
        pyinstaller pour générer le .app .  Ensuite il effectue les opérations
        décrites dans la section \ref{mac_pyinstaller} afin d'obtenir un .app
        fonctionnel.\\

        Après exécution de ce script de génération, à l'intérieur du dossier
        d'output se trouve l'application ainsi que le dossier docs contenant des
        images et la documentation.

\section{Améliorations à apporter}
    La partie sur le lancement des calculs sur le cluster est à améliorer. Elle
    n'a pas été testée ni lancé un grand nombre de fois et la finition est à
    revoir.\\
    
    Il faudrait aussi effectuer le transfert de la table de référence
    générée une fois le calcul effectué sur le cluster.\\
    
    L'interruption de
    connexion n'est pas gérée, les calculs continuent de tourner sur le cluster
    même en cas d'arrêt de l'interface mais on ne peut pas se connecter à
    nouveau pour voir l'avancement.


\end{document}
