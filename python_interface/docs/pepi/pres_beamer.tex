\documentclass{beamer}
\usepackage[utf8]{inputenc}
\usepackage[francais]{babel}
\usepackage[T1]{fontenc}
%\usepackage{multimedia}
\usepackage{graphicx}
\usepackage{hyperref}
\usepackage{algorithm}
\usepackage{algpseudocode}
\floatname{algorithm}{Algorithme} 

%\usetheme[width=100pt]{PaloAlto}
\usetheme[compress]{Ilmenau}
\setbeamertemplate{frametitle}[default][center]

%\setbeamercovered{transparent}

\title{Proactive : Un middleware open source pour le calcul parallèle}

\author{Veyssier Julien}
\institute{CBGP - INRA}
\date\today
\setbeamertemplate{navigation symbols}{}
%\logo{}
\begin{document}

\begin{frame}
\titlepage
\end{frame}

\begin{frame}
\tableofcontents
\end{frame}

\section[Introduction]{Introduction}
\begin{frame}
	\tableofcontents[currentsection]
\end{frame}

\begin{frame}{Contexte}
	\begin{columns}
	\begin{column}[l]{0.5\linewidth}
        \begin{columns}
        \begin{column}[l]{0.5\linewidth}
        \begin{figure}
            %[!bh]
            \centering
            \includegraphics[scale=0.32]{cbgp.png}
            %\caption{Architecture autour d'un CDN}
        \end{figure}
        \end{column}
        \begin{column}[r]{0.5\linewidth}
        \begin{figure}
            %[!bh]
            \centering
            \includegraphics[scale=0.32]{cocci.png}
            %\caption{Architecture autour d'un CDN}
        \end{figure}
        \end{column}
        \end{columns}
        \vspace{1cm}
	\begin{block}{Objectifs}<2->
			\begin{itemize}
                    %decouplage .. .
                \item Refonte de DIYABC
				\item Implémentation de l'interface graphique
                \item Interfaçage avec des outils de calcul parallèle
			\end{itemize}
					
		\end{block}

	\end{column}
	\begin{column}[r]{0.5\linewidth}
	\begin{exampleblock}{Activités du laboratoire}<1->
			\begin{itemize}
				\item Génétique des population
                \item Systématique
                \item Écologie
			\end{itemize}
					
		\end{exampleblock}
	\begin{block}{Moyens matériels}<3->
			\begin{itemize}
                \item Cluster local (SGE)
				\item Machines personnelles des chercheurs
                \item Postes de travail de l'administration
			\end{itemize}
					
		\end{block}
	\end{column}
	\end{columns}
\end{frame}

\begin{frame}{État de l'art}
    Constat : architectures trop rigides et fastidieuses à maintenir
	\begin{columns}
	\begin{column}[l]{0.3\linewidth}
    \begin{block}{Projets}
        \begin{itemize}
            \item<2-> Diane
            \item<3-> Ganga
            \item<4-> Wisdom
            \item<5-> OurGrid
            \item<6-> Proactive
        \end{itemize}
    \end{block}
	\end{column}
	\begin{column}[r]{0.6\linewidth}
    \begin{exampleblock}{Particularité}
        \begin{itemize}
            \item<2-> Peu d'outils annexe
            \item<3-> Sous-partie de DIANE
            \item<4-> Orienté vers les grilles de calcul
            \item<5-> Programmes non fournis
            \item<6-> Découplé et polyvalent
        \end{itemize}
    \end{exampleblock}
	\end{column}
	\end{columns}
\end{frame}

\begin{frame}{Proactive}
    Exploration des possibilités pour une éventuelle intégration au CBGP
\end{frame}

\section[Fonctionnement]{Fonctionnement de Proactive}
\begin{frame}{Caractéristiques}
    Langage de programmation $\Longrightarrow$ Java + Bash/Batch
    Portable
    Supporte un environnement hétérogène%on y reviendra plus tard

\end{frame}
\begin{frame}{Architecture}
	\begin{columns}
	\begin{column}[l]{0.5\linewidth}
        \begin{figure}
            %[!bh]
            \centering
            \includegraphics[scale=0.4]{arch.png}
            %\caption{Architecture autour d'un CDN}
        \end{figure}
	\end{column}
	\begin{column}[r]{0.5\linewidth}
        \begin{alertblock}{Ressource management}
            Gestion des noeuds de calcul
        \end{alertblock}
        \begin{block}{Routeur}
             Gestion des communications entre les composants
        \end{block}
        \begin{exampleblock}{Scheduler}
             Gestion de la politique d'ordonnancement
        \end{exampleblock}
        \begin{alertblock}{Client}
            Interaction avec l'utilisateur
        \end{alertblock}
        
	\end{column}
	\end{columns}
\end{frame}

\subsection{Composants}
\begin{frame}{Programmes}
    \begin{block}
        
    \begin{itemize}
            % et leurs clients respectifs
        \item Noeud de calcul $\Longrightarrow$ accepte et traite des t\^aches
        \item Gestionnaire de ressources $\Longrightarrow$ recense et contrôle les noeuds de calcul
        \item Ordonnanceur $\Longrightarrow$ accepte des t\^aches et les affecte à des noeuds de calcul
        \item Serveur de données $\Longrightarrow$ permet la transmission des données liées aux calculs
    \end{itemize}
    \end{block}
\end{frame}

\begin{frame}{Noeud de calcul}
    % a lancer sur chaque machine ressource
    % reçoit les taches, les execute dans un env temporaire qui sera nettoyé
\end{frame}

\begin{frame}{Gestionnaire de ressources}
    % attend la connexion des noeuds de calcul et les requêtes de l'ordonnanceur
    
\end{frame}

\begin{frame}{Serveur de données}
    % permet aux noeuds de calcul de ``monter'' un espace pour y lire et y écrire

    % Peut être lancé par un administrateur pour fournir un espace permanent commun à tous les utilisateurs (faisable au CBGP, petite taille)
    % peut être lancé par chaque client pour créer l'espace sur sa propre machine
    
\end{frame}

\begin{frame}{Ordonnanceur}
    % est l'intermédiaire entre le client et le gestionnaire de ressources
    % Il gère l'authentification des clients et leurs droits d'accès aux ressources 
    % et sait communiquer avec le rm pour affecter les taches aux noeuds de calcul
    
\end{frame}

\subsection{Architecture}

\begin{frame}{Communication réseau}
    \vspace{-1.9cm}
    \begin{figure}
        %[!bh]
        \hspace*{-1.8cm}
        \centering
        \includegraphics[scale=0.52]{netmap.pdf}
        %\caption{Communication réelle entre les entités logicielles de Proactive}
    \end{figure}
    
\end{frame}

\begin{frame}{Communication dans le réseau virtuel de Proactive}
    \vspace{-2.9cm}
    \begin{figure}
        %[!bh]
        \hspace*{-1.2cm}
        \centering
        \includegraphics[scale=0.62]{netmap_abs.pdf}
        %\caption{Principe de communication dans le réseau Proactive}
    \end{figure}
    
\end{frame}

\subsection{Définition d'une tache}
\begin{frame}
	\begin{columns}
	\begin{column}[l]{0.5\linewidth}
        \begin{block}{Qu'est-ce qu'une t\^ache ?}
            \begin{itemize}
                \item Un identifiant
                \item Un serveur de données
                \item Une liste d'opération à effectuer %option dependances, retry..
            \end{itemize}
        \end{block}
        \begin{exampleblock}{Définition en Java}
            \begin{itemize}
                \item Directement dans un programme en Java qui se connecte à l'ordonnanceur
                \item Gestion fine
            \end{itemize}
        \end{exampleblock}
	\end{column}
	\begin{column}[r]{0.5\linewidth}
        % en java, par xml, ou par liste de commandes
        \begin{exampleblock}{Définition par XML}
            \begin{itemize}
                \item Fichier XML qui spécifie chaque paramètre
                \item Plus simple mais moins complète que Java
            \end{itemize}
        \end{exampleblock}
        \begin{exampleblock}{Liste de commandes}% natives obligatoirement
            \begin{itemize}
                \item Liste de commandes qui seront exécutées directement sur un noeud de calcul
                \item Gestion basique
            \end{itemize}
        \end{exampleblock}
	\end{column}
	\end{columns}
\end{frame}

\subsection{Outils graphiques}
\begin{frame}
    tools
\end{frame}


\section[Applications]{Applications possibles}
\begin{frame}
	\tableofcontents[currentsection]
\end{frame}

\begin{frame}
	% constat, propo, exemple

		\begin{block}{Travail effectué}<1->
		\begin{itemize}
			\item Constat%diff entre topos
			\item Propositions
			\item Étude d'un exemple d'application
		\end{itemize}
		\end{block}

		\begin{exampleblock}{Perspectives}<2->
		\begin{itemize}
			\item Établir une norme sur les échanges de messages
				%pour favoriser les
				%reflexions sur ce genre d'opti?
			\item Cache s\'electif (images, degr\'e de volatilit\'e)
			\item Élaboration de protocoles applicatifs pour la
				communication entre routeurs (Gossiping)
			\item Création d'une architecture spécialisée pour les CDN
			\item Proposition d'extension du protocole IP proposant
				des primitives de plus haut niveau dans les
				routeurs
		\end{itemize}
		\end{exampleblock}
	% 
\end{frame}

\begin{frame}
	\begin{center}	{\huge Merci de votre attention}\end{center}
	\end{frame}
\begin{frame}
	\begin{center}	{\huge Questions}\end{center}
	\end{frame}

\end{document}
