\documentclass{beamer}
\usepackage[utf8]{inputenc}
\usepackage[francais]{babel}
\usepackage[T1]{fontenc}
%\usepackage{multimedia}
\usepackage{graphicx}
\usepackage{hyperref}
\usepackage{algorithm}
\usepackage{algpseudocode}
\floatname{algorithm}{Algorithme} 

%\usetheme[width=100pt]{PaloAlto}
\usetheme{Ilmenau}

%\setbeamercovered{transparent}

\title{Proactive : an Open Source Middleware for parallel computing}

\author{Veyssier Julien}
\institute{CBGP- INRA}
\date\today
\setbeamertemplate{navigation symbols}{}
%\logo{}
\begin{document}

\begin{frame}
\titlepage
\end{frame}

\begin{frame}
\tableofcontents
\end{frame}

\section{Contexte et objectif}
% p2p, CDN -> routage -> amelioration, topologies diff, routage logique et IP

% resumer mon travail. 
%biblio,
%extraction des points qui pèchent,
%pistes d'opti,
%creuser une d'entre elles
\begin{frame}
	\tableofcontents[currentsection]
\end{frame}

\begin{frame}{Les réseaux de distribution de contenu}
	\begin{columns}
	\begin{column}[l]{0.5\linewidth}
	\begin{block}{}
			\begin{itemize}
				\item CDN = Content Distribution Network
				\item Akamai, Squirrel\ldots %DHT
				\item Objectifs :
					\begin{itemize}
						\item fiabilité
						\item performance
						\item passage à l'échelle
					\end{itemize}
			\end{itemize}
					
		\end{block}
	\end{column}
	\begin{column}[r]{0.5\linewidth}
        plop
	\end{column}
	\end{columns}
\end{frame}

\begin{frame}{CDN et P2P}
    plap
\end{frame}



\begin{frame}
	%TODO schema qui illustre lincoherence entre topologies
	\begin{columns}
		\begin{column}[l]{0.65\linewidth}
	\begin{block}{Concepts}<1->
		\begin{itemize}
			\item CDN
			\item P2P (Peer to Peer) % passage à lechelle ->
			            		%good for CDN
			\item Routage logique
			\item Routage physique
		\end{itemize}
	\end{block}
	\begin{alertblock}{Objectif}<2->
		Optimiser le routage en se concentrant sur les différences
		entre topologie logique et physique
	\end{alertblock}
	\end{column}

		\begin{column}[r]{0.35\linewidth}
            blurp
\end{column}
\end{columns}
\end{frame}

\section[Conclusion]{Conclusion et perspectives}
% temps de reponse minimise dans les conditions decrites

% faire des protocoles de niveau appli mais reserves aux routeurs

% jusqu'ou peuvent aller les archi dispo?
% necessite d'en élaborer une specialement pour les CDN? 
% Propositions d'integration de primitives dans les protocoles de type IP? 
% Generalisation de ce type d'opti?
%proposer une norme sur les echanges de mess ds les CDN pour favoriser les
%reflexions sur ce genre d'opti?
\begin{frame}
	\tableofcontents[currentsection]
\end{frame}

\begin{frame}
	% constat, propo, exemple

		\begin{block}{Travail effectué}<1->
		\begin{itemize}
			\item Constat%diff entre topos
			\item Propositions
			\item Étude d'un exemple d'application
		\end{itemize}
		\end{block}

		\begin{exampleblock}{Perspectives}<2->
		\begin{itemize}
			\item Établir une norme sur les échanges de messages
				%pour favoriser les
				%reflexions sur ce genre d'opti?
			\item Cache s\'electif (images, degr\'e de volatilit\'e)
			\item Élaboration de protocoles applicatifs pour la
				communication entre routeurs (Gossiping)
			\item Création d'une architecture spécialisée pour les CDN
			\item Proposition d'extension du protocole IP proposant
				des primitives de plus haut niveau dans les
				routeurs
		\end{itemize}
		\end{exampleblock}
	% 
\end{frame}

\begin{frame}
	\begin{center}	{\huge Merci de votre attention}\end{center}
	\end{frame}
\begin{frame}
	\begin{center}	{\huge Questions}\end{center}
	\end{frame}

\end{document}
